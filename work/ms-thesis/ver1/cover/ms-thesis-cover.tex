

\title{FINANCIAL TIME SERIES PREDICTION USING KALMAN FILTERS AND HIDDEN MARKOV MODELS}
\turkcebaslik{HMM VE KALMAN FÄ°LTRELERÄ° KULLANARAK FÄ°NANSAL ZAMAN SERÄ°SÄ° TAHMÄ°NÄ°}

\degree{B.E., Computer Engineering, Stevens Institute of Technology, 1996}

\author{Burak Bayramlı}

\program{Systems and Control Engineering}

\supervisor{Prof. Fikret Gürgen}

\examineri{Prof. Levent Akın}

\examinerii{Prof. Ethem Alpaydın}

\dateofapproval{..............................}


\begin{document}

\pagenumbering{roman}
\makemstitle        % For M.S. theses

\makeapprovalpage

\begin{acknowledgements}
I would like to thank my thesis advisor Dr. Gürgen for his help, direction and
patience during my work on this thesis. 

Thanks to Dr. Ethem Alpaydın for first suggesting Systems and Control
Engineering department as a good master's goal, and second for all the knowledge
I gained in his {\em Pattern Recognition} class. Once in Systems and Control
Engineering department, I was able to take immensely interesting classes such as
{\em Nonlinear Dynamics} and {\em Simulation} both of which helped greatly in
researching this thesis. I am also grateful to Dr. Goldsman for the all
knowledge he was willing to part with in his {\em Simulation} class - I think
his lectures on pseudorandom generation will stay with me forever. It was also
in his class I first saw Brownian Motion which is very important in financial
time series analysis.

I extend warmest thanks to Dr. Levent Akın for his excellent {\em Autonomous
Robotics} class and his suggestion that I look into {\em Probabilistic
Robotics}, a book and term popularized by Sebastian Thrun - it was while
flipping through the pages of this book that I got the idea of using Kalman
Filters for stock price prediction.

I would also like to thank my parents Ercan and Füsun Bayramlı for their support
without whom none of this would have been possible.
\end{acknowledgements}

\begin{abstract}
Financial markets are challenging targets for analysis and prediction. The
existence of many factors that contribute to a final price of security at time
$t$, makes the task of predicting a future price a very hard task indeed. In the
past, many statistical and non-statistical models have been utilized that
attempted to perform this daunting task - the aim of this thesis is going to be
trying to demonstrate Hidden Markov Models and Kalman Filter methods can be used
for predicting a future price. We also devised a mixture predictor using HMM and
KF which we called KMM.

For this purpose, we hypothized HMM, KF and KMM based models that are trained on
historical data can generate future data, thereby predicting this securities'
price in the future. For data generation, we used Monte Carlo simulation to
smooth over the irregular patterns. ``Carrying the model forward in time'' is
achieved by a combination of Viterbi algorithm and rolling the dice on hidden
state transitions, in KF case, we follow the time transition equation.

Another goal of this thesis was to determine how HMM, KF and KMM methods stood
in comparison to other conventional methods. We picked polynomial regression,
Neural Networks (ANN) and plain ``random guess'' as our comparison
criteria. Random guess was expected to be the lowest performer in our tests,
every method mentioned should have surpassed random guess results by wide
margin. We were glad to see that this was indeed the case.
\end{abstract}

\begin{ozet}
Finans piyasaları, analiz ve fiyat tahmini yapmak açısından zor bir
alandır. Tahmin yapılmasını zorlaştıran faktör, herhangi bir gündeki fiyatın
ortaya çıkmasında çok fazla etkinin olmasıdır. Tahmin ve analiz için şimdiye
kadar pek çok istatistiki ve diğer yöntemler bu amaç için seferber edilmiştir -
biz de bu yolda bir adım atmaya uğraştık.

Bu bağlamda Gizli Markov Modelleri ve Kalman Filtreleri adlı iki kuvvetli
metotun, geçmiş senet verisini modelleyebilerek geleceğe dönük tahmin
yapabileceğini varsay\-dık, ve iki metodu birleştirerek KMM adlı üçüncü bir
metodun performansına baktık. Fiyat tahmin sayılarıni üretirken Monte Carlo
metotu kullanarak uç noktalarda olan tahminleri ortalama içinde düzeltmeye ve
böylece daha iyi bir tahmine ulaşmaya çabaladık. Model eğitildikten sonra onu
ileriye dönük tahminlerde kullanmak için ise, GMM durumunda Viterbi algoritması
ile geçiş olasılıklarını rasgele sayı üreterek takip eden bir algoritma, Kalman
Filtreleri durumunda ise en son gelinen zaman noktasından bir sonrakine geçiş
için bizim değiştirdiğimiz zaman güncelleme denklemleri, ve yine rasgele sayı
üreten bir algoritma kullandık.

Bu tezin bir diğer amacı tarif ettiğimiz metotların diğer klasik metotlara
kıyasla nasıl başarı göstereceğini ortaya koymaktı. Bunun için Yapay Sinir
Ağları, polinom regresyon ve en son pür rasgele sayı üreterek tahmin yapmaya
uğraşan üç diğer metotu test senaryolarımıza dahil ettik. Bu metotların arasında
rasgele üretim en alt seviyede performans göstermesini beklenilen metot idi,
diğer tüm metotlar farklı derecelerde bu metotu geçmeliydi, ki deneyler
sonucunda durumun böyle olduğunu gözlemledik.
\end{ozet}

\tableofcontents
\listoffigures
\listoftables

\begin{symabbreviations}
% The title will be typeset as "LIST OF SYMBOLS/ABBREVIATIONS".
% If you prefer it as "LIST OF SYMBOLS" or "LIST OF ABBREVIATIONS"
% use the environments "symbols" or "abbreviations", respectively.
%
% Use a separate \sym command for each symbols definition.
% First Latin symbols in alphabetical order

\sym{$a_{ij}$}{HMM transition probability $p(q_t^j=1 | q_t^i=1)$ of hidden states}
\sym{$A$}{HMM hidden transitions matrix. $a_{ij}$ accesses elements of $A$}
\sym{$A$}{Linear multiplier for KF time transition from time $t$ to $t+1$}
\sym{$C$}{A vector holding all means of emission Gaussians for HMM}
\sym{$C$}{Linear multiplier for KF that translates $x_t$  to $y_t$}
\sym{$h_t$}{Sum of errors going back $p$ days for ARCH / GARCH / EGARCH modeling.}
\sym{$K_t$}{Kalman gain matrix at time $t$} 
\sym{$P_t$}{Price of a security at time $t$}
\sym{$R_t$}{Rate of return for financial time series data}
\sym{$R$}{Variance of the emission Gaussian for HMM}
\sym{$P_{t}^{t}$}{Conditional variance of $x$  at time $t$ given output
emissions $y_0,...,y_t$}
\sym{$P_{t,t-1}$}{Joint covariance necessary for KF lag-one covariance smoothing
calculations} 
\sym{$q_t$}{HMM's hidden state symbol at time $t$}
\sym{$T$}{Number of training data points} 
\sym{$v_t$}{White noise - Gaussian distribution with zero mean and covariance $Q$}
\sym{$\hat{x}_{t}^t$}{Conditional expectation of $x$ at time $t$ given output
emissions $y_0,...,y_t$} 
\sym{$x_t$}{Random variable representing hidden state at time $t$ for KF
(Gaussian distribution).} 
\sym{$y$}{All of the measured/observed data for HMM, KF and KMM}
\sym{$w_t$}{White noise - Gaussian distribution with zero mean and covariance $R$}

\\\\

\sym{$\alpha(q_t)$}{Probability of emitting partial sequence of outputs
$y_0,...,y_t$ and ending up in state $q_t$ \cite{jordan} for HMM}
\sym{$\beta(q_t)$}{Probability of emitting partial sequence of outputs
$y_{t+1},..,y_T$ given that system starts in state $q_t$ for HMM}
\sym{$\eta$}{Distribution representing output emissions for HMM}
\sym{$\epsilon$}{Error}
\sym{$\gamma_{t}$}{HMM posterior probability $p(q_t|y)$ at time $t$}
\sym{$\pi$}{HMM initial probability distribution}
\sym{$\xi_t$}{HMM cooccurence probabilities of $q_t$ and $q_{t+1}$}

\\\\

\sym{ANN}{Artificial Neural Network}
\sym{ARCH}{Autoregressive conditional heteroskedasticity}
\sym{DJI}{Dow Jones Industrial}
\sym{EM}{Expectation Maximization}
\sym{GARCH}{General ARCH}
\sym{HMM}{Hidden Markov Models}
\sym{HME}{Hidden Markov Experts}
\sym{iid}{Independent Identically Distributed}
\sym{ISE}{Istanbul Stock Exchange}
\sym{KF}{Kalman Filters} 
\sym{KMM}{A hybrid method combining KF and HMM} 
\sym{MC}{Markov Chains} 
\sym{MLP}{Multilayer perceptron} 
\sym{RTS}{Rauch-Tung-Striebel} 
\sym{SSM}{State space model}


\end{symabbreviations}

