\documentclass[11pt]{article} 

\usepackage{palatino,eulervm}
\usepackage{cancel}
\usepackage[latin5]{inputenc}
\usepackage{color}
\usepackage{latexsym}
\usepackage{amsfonts}
\usepackage[hyphens]{url}
\usepackage[bookmarks=false]{hyperref}
\makeatletter \renewcommand\verbatim@font{\footnotesize\ttfamily} \makeatother
% pseudocode
\usepackage{algorithm}
\newcommand{\code}[1]{\mbox{\texttt{#1}}}
\newcounter{lineno}
\newenvironment{pseudocode}{\begin{small}\begin{tabbing}\textbf{mm}\=mm\=mm\=mm\=mm\=mm\=mm\=mm\=mm\=\kill}{\end{tabbing}\end{small}}
\newcommand{\codename}{\setcounter{lineno}{0}\>}
\newcommand{\codeline}{\>\stepcounter{lineno}\textbf{\arabic{lineno}}\'\>}
\newcommand{\codeskip}{\>\>}

\begin{document}
\title{3D Visual Tracking with Particle and Kalman Filters}
\author{Burak Bayraml�}
\date{June 29, 2010}

\maketitle

\begin{abstract}
One of the most visually demonstrable and straightforward uses of filtering is
in the field of Computer Vision. In this document we will try to outline the
issues encountered while designing and implementing a particle and kalman filter
based tracking system.
\end{abstract}

\section{Introduction}

Kalman and particle filters are examples of recursive filters; using Kalman
filters for example,  the state equations are in the form of 

\begin{eqnarray*}
x_{t+1} = A x_t + Q\\
y_t = Hx_t + R
\end{eqnarray*}

is utilized where $x_t$ and $y_t$ are distributed as Gaussians. The beauty of
recursive filtering is that the incoming data is for $y_t$, and based on this
data, we can "reverse the arrow", and calculate $x_t$, which is the hidden
state. We know the transition equation beforehand (multiplication of $x_t$ by A
with the addition of noise, $Q$) as well as the observation equation,
$y_t$. Together, these two equations could represent an object moving following
a certain movement (a robot moving, a plane flying) and our observation error
which could arise due to sensor sensitivity, unknown calibration parameters or
other external conditions.

We need to emphasize however, that knowing transition and observation equations
"exactly" does not mean that the mathematical motivation for filtering is
weak. For our example we had to determine a 3D location based on successive
pixel readings, that is 2D data readings. As anyone who dealt with depth
calculation from 2D images can attest to, we lose data when we transition from
3D to 2D; by looking at a single image, we can never determine where an object
really is. We could be looking at an image of the Empire State building 1000
meters away, or a miniature of it, right in front of our nose. This is why 3D
the "reverse" calculation requires a succession of images; we match a pixel on
all images in order to calculate its 2D displacement, using that and the
viewer's known physical 3D displacement, we can try to calculate (triangulate) a
3D location.

There are other problems. Using pure graphical methods, such as Multiple View
Geometry for 3D calculation is not an easy task, somewhat error prone, and still
does not give us true scale of an object. Filtering not only can take into
account more than 2 images, but it can also account for noise, uncertainty of
our transition and observation model, all based on an initial assumption which
can encode scale assumptions in it as well. Then, every new measurement makes
the estimation for $x_t$ better, plus this method is suited perfectly for online
applications -- all history is encoded and reflected in $x_t$, historical values
themselves are not kept beyond a single frame.

We implemented a tracking solution using both Kalman and particle filters. A
chessboard plane was simply moved on a flat surface (table) on constant speed
toward our camera while we tracked a single reference point on this image. State
$x_t$ was used to represent the 3D location of our chessboard plane. The
transition equation of the Kalman filters only needed to account for constant
velocity, along one axis. Matrix A looked like

\begin{eqnarray*}
A= \left[
\begin{array}{cccc}
1 & 0 & 0 & 0 \\
0 & 1 & 0 & 0 \\
0 & 0 & 1 & d \\
0 & 0 & 0 & 1
\end{array}
\right]
\end{eqnarray*}

Note that matrix A is 4x4, not 3x3. We used homogenous coordinates to represent
3D location, which made $x_t$ a 4x1 vector, therefore the transition equation
captured in A had to be 4x4 so we could multiply it with $x_t$. 

A is the identity matrix I with one difference; its (3,4)'th item is a constant,
d, that is our displacement. To verify if this $A$ can be used for calculating
displacement, one could perform a sample multiplication using $x_t =
[\begin{array}{cccc}a1 & a2 & a3 & a4 \end{array}]$ and see $Ax_t$, or
$x_{t+1}$, becomes $[\begin{array}{cccc}a1 & a2 & a3+d & a4\end{array}]$. We can
see displacement d is added to the z-coordinate, depth. For our testing we
picked d = -0.5, meaning for each frame chessboard plane moved 0.5 cm toward the
camera. Each transition for $x_t$ adds to uncertainty, and that is reflected in
$Q$.

Observation equation $y_t$ handles the calculation where a 3D coordinate is
projected onto 2D (pixel measurements) on screen with added noise. For standard
pinhole camera model, camera matrix P is responsible for 2D projection. P is
unique for each camera, and a process called camera calibration can determine
camera matrix P, and various methods for calibration exist in literature. OpenCV
library contained a function for calibration which we tested, but we were not
happy with the results. At the end, we calibrated the camera manually; by simply
deciding on a 3D real world location (middle point at the end of our flat
surface / table) and tested various P values while at the same time drawing an
imaginery chessboard image on screen (shown as a green square on
Figure~\ref{calib}) using this projection matrix P. This was repeated until the
imaginary board was located at the desired location.

%cb-kf-00.eps

We also tested if manual, virtual displacements in centimeters from the test
point was reflected in the projection accurately. Projection matrices are 3x3,
Kalman filter also adds a 4th row [0 0 0] to this P turning it into an H matrix
in 4x3 dimensions.

The reason for using chessboard image was so \verb!cvFindChessboardCorners!
and \verb!cvDrawChessboardCorners! OpenCV functions could be used. Using
these two methods, a chessboard image can be detected and marked (on
screen, real-time) with great accuracy and speed. The chessboard image had
9 squares on it, giving us 9 points of which, we only used the 5th, middle
point. At each frame \verb!cvFindChessboardCorners! detected the 2D pixel
locations, and we passed these values over to the Kalman filter that
recursively updated its hidden state.

We tested two scenarios for tracking, one starting from 36cm to the left, the
other from 30 cm to the right to the midend point of the flat surface, in each
case moving toward the camera in constant speed. In each case initial condition
for the filter is the middle end point, somewhat away from both starting
locations, therefore the uncertainty $Q$ of the filter at time = 0 had to be
large. Hence we used $Q = I*150cm$ for the Kalman filter. 

\subsection{Particle Filters}

Another method for recursive estimation is called particle filters. Particle
filters are able to represent a distribution using "particles" which are weights
in an array that are assigned to possible values of $x_t$ which are elements in
an array. These values and their associated weights together form a discrete
distribution that can represent any distribution at an accuracy allowed by the
number of particles. Bayesian filtering through particles is achived by first
applying a transition where uncertainty increases, then we apply the observation
function to ``update'' the distribution by processing an error function. The
error function for each particle is

\begin{eqnarray*}
w^{[i]}=\frac{1}{1 + (y^{[i]}-p^{[i]})^2}
\end{eqnarray*}

where $y^{[i]}$ is an observation value and $p^{[i]}$ is the ``guess'' that was
calculated by applying the transition function to previous estimate, akin to
$Ax_t + Q$ transition of the Kalman filter. That transition for particle filters
is calculated by sampling from a uniform distribution and adding the results to
$x$, for each particle. We sample from $Unif(-0.1, -1)$ for z-coordinate
addition, and from $Unif(-40,40)$ for x-coordinate addition. In other words, we
add forward motion between 0.1 and 1 centimeters, and an uncertainty bubble 80
cm wide, horizontally, in both directions. 

The reason we add '$1$' to $(y^{[i]}-p^{[i]})^2$ in the fraction should be
clear; this way the error function can give back a probability-like result. For
small errors, $1+error$ divides the '$1$' in the nominator, giving back a number
close to 1. This is what we want, smaller errors resulting in greater weight,
hence greater probability, larger errors causing the opposite.

The resampling process is also straightforward; it makes sure that particles,
values with greater weight are repeated more than particles with smaller
probabilities. Note however resampling procedure does not create new particle
values, it simply repeats (or skips) existing particles.

\section{Conclusion}

All code was written in Python language, using Scipy and Numpy libraries, and
also OpenCV. The results received from this experiment were satisfactory, it was
witnessed that in both Kalman and particle filter cases tracking worked
successfully. Our example code can be downloaded from our
blog\footnote{http://ascratchpad.blogspot.com/2010/06/3d-tracking-using-kalman-and-particle.html},
and selected screenshots for both approaches can be found below.

\begin{thebibliography}{1}

\bibitem{marsland}
S.~Marsland, \emph{Machine Learning: An Algorithmic Perspective}, CRC Press, 2009.

\bibitem{thrun}
S.~Thrun, W.~Burgard, and D.~Fox, \emph{Probabilistic Robotics}, MIT Press, Cambridge, MA, 2005

\end{thebibliography}


\end{document}


