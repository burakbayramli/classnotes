\documentclass[12pt,fleqn]{article}\usepackage{../../common}
\begin{document}
Ders 19

Ho�geldiniz. Yeni bir konuya yelken a�ma zaman� geldi, tek boyutlu
e�lemeler (kitab�mda 10. b�l�m). Bu teknik bize kaosu, �imdiye kadar
ilgilendi�imiz Lorenz sisteminden daha az �etrefil olan baz� �atalla�malar�
tarif etmekte daha basit bir y�ntem sa�layacak. 1-D e�lemeler �u formda
oluyorlar, 

$$ x_n = f(x_n) $$

Burada $n$ zamanla e�de�er say�labilir, ama s�rekli de�il ayr�ksal
(discrete). Yani bir s�reli�ine diferansiyel denklemler konusunu
terkediyoruz, ve kaosun daha basit modelleri olan 1-d e�lemelere
odaklan�yoruz. 

1-d eslemeler ve temel bilimde 





















\end{document}


























