\documentclass[12pt,fleqn]{article}\usepackage{../../common}
\begin{document}
Ders 16

Bu derste su �ark� denklemlerini h�zl�ca analiz edece�iz, ama hemen
ard�ndan onlar� b�rak�p yak�ndan alakal� kaosta herkesin inceledi�i ama
daha iyi bilinen Lorenz denklem formuna ge�ece�im.

Su �ark� sistemini en son b�rakt���m�z nokta neydi? Bir mucizeyi
gozlemledik (!), sonsuz tane denklem i�inden �� tane denklem kald�, bu ��
ODE geriye kalanlardan tamamen ba��ms�zdi. Tekrar yazarsak, 

$$ \dot{a_1} = \omega b_1 - K a_1 $$

$$ \dot{b_1} = -\omega a_1 + q_1 - Kb_1 $$

$$ \dot{\omega} = \frac{-v\omega}{I} + \frac{\pi gr a_1}{I} $$



















[devam edecek]


\end{document}




















