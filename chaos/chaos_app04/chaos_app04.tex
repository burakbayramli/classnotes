\documentclass[12pt,fleqn]{article}\usepackage{../../common}
\begin{document}
Enflasyon


$$ 
\frac{\dot{\lambda}}{\lambda} =
\left( 
  \left( \frac{I_{fn}(\pi_r)}{v}  - \delta_{Kr} \right) -
  (\alpha + \beta)
\right)
$$

$$ 
\frac{\dot{\omega}}{\omega} = 
\left( 
(w_{fn}(\lambda) - \alpha) + \frac{1}{\tau_p} \left( 1-\frac{1}{1-s} \cdot \omega\right)
\right)
$$


$$ 
\frac{\dot{d}}{d} = 
\frac{\left( \frac{I_{fn}(\pi_r)}{v} \right) - \pi_s }{d} - 
\left[ 
\left( \frac{I_{fn}(\pi_r)}{v} -\delta_{Kr} \right) + 
\frac{1}{\tau_p} \left(1 - \frac{1}{1-s} \omega\right)
\right]
$$





\newpage









$$ 
\frac{\dot{\lambda}}{\lambda} =
\left( 
  \underbrace{ \left( \frac{I_{fn}(\pi_r)}{v}  - \delta_{Kr} \right)}_{A} - 
  \underbrace{ (\alpha + \beta) }_{B}
\right)
$$

A: Growth Rate

B: Labor productivity and population growth

``Employment rises if the growth rate exceeds population growth and labor productivity.''


$$ 
\frac{\dot{\omega}}{\omega} = 
\left( 
(\underbrace{w_{fn}(\lambda)}_{A} - \underbrace{\alpha}_{B}) + \frac{1}{\tau_p}
\underbrace{\left( 1-\frac{1}{1-s} \cdot \omega \right)}_{C}
\right)
$$

A: Wage share 

B: Labor productivity

C: Inflation

``Wage share rises if money wage demands greater than labor productivity and
the inflation rate''



$$ 
\frac{\dot{d}}{d} = 
\underbrace{\frac{\left( \frac{I_{fn}(\pi_r)}{v} \right) - \pi_s }{d}}_{A} - 
\left[ 
\underbrace{\left( \frac{I_{fn}(\pi_r)}{v} -\delta_{Kr} \right) }_{B}+ 
\underbrace{\frac{1}{\tau_p} \left(1 - \frac{1}{1-s} \omega\right)}_{C}
\right]
$$

A: Debt growth rate

B: Real growth rate

C: Inflation

``Debt ratio will rise if debt growth rate is greater than real growth rate
plus inlation''



\newpage


$$ 
\frac{\dot{\lambda}}{\lambda} =
  g - (\alpha + \beta)
$$

$$ 
\frac{\dot{\omega}}{\omega} = 
\left( 
(w_{fn}(\lambda) - \alpha) + \frac{1}{\tau_p} \left( 1-\frac{1}{1-s} \cdot \omega\right)
\right)
$$


$$ 
\frac{\dot{d}}{d} = 
\frac{\left( \frac{I_{fn}(\pi_r)}{v} \right) - \pi_s }{d} - 
\left[ 
g + \frac{1}{\tau_p} \left(1 - \frac{1}{1-s} \omega\right)
\right]
$$


$$ g = \frac{I_{fn}(\pi_r)}{v} - \delta_{Kr} $$

$g$ is the real growth rate, which is determined by the rate of investment
and depreciation.


\newpage



$$ 
\frac{\dot{\lambda}}{\lambda} =
\left( 
  \underbrace{ g }_{A} - 
  \underbrace{ (\alpha + \beta) }_{B}
\right)
$$

A: Growth Rate

B: Labor productivity and population growth

``Employment rises if the growth rate exceeds population growth and labor productivity.''


$$ 
\frac{\dot{\omega}}{\omega} = 
\left( 
(\underbrace{w_{fn}(\lambda)}_{A} - \underbrace{\alpha}_{B}) + \frac{1}{\tau_p}
\underbrace{\left( 1-\frac{1}{1-s} \cdot \omega \right)}_{C}
\right)
$$

A: Wage share 

B: Labor productivity

C: Inflation

``Wage share rises if money wage demands greater than labor productivity and
the inflation rate''



$$ 
\frac{\dot{d}}{d} = 
\underbrace{\frac{\left( \frac{I_{fn}(\pi_r)}{v} \right) - \pi_s }{d}}_{A} - 
\left[ 
\underbrace{g }_{B}+ 
\underbrace{\frac{1}{\tau_p} \left(1 - \frac{1}{1-s} \omega\right)}_{C}
\right]
$$

A: Debt growth rate

B: Real growth rate

C: Inflation

``Debt ratio will rise if debt growth rate is greater than real growth rate plus inlation''








[devam edecek]



[1] Keen, {\em A monetary Minsky model of the Great Moderation and the Great Recession}

[2] {\em Greenwich-Kingston PhD students lecture: the logic  maths of modelling Minsky (2)} \url{youtu.be/0Do05hV_Iqo}

[3] Jelonek, {\em Numerical techniques in MATLAB: differential equations and non-linear dynamics} \url{https://warwick.ac.uk/fac/soc/economics/current/modules/rm/notes1/research_methods_matlab_3.pdf}

\end{document}


















