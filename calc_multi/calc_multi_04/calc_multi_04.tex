\documentclass[12pt,fleqn]{article}\usepackage{../../common}
\begin{document}
Ders 4

Düzlemin formülüne bakalım.

$$ ax + by + cz = d $$

Bu formül $x,y,z$ noktalarının bir düzlem üzerinde olma şartını tarif
ediyor. 

Şu problemlere bir göz atalım. Diyelim ki 

1) Orijinden, yani $(0,0,0)$ noktasından geçen ve normal vektörü $\vec{N} =
< 1,5,10 >$ olan bir düzlem yaratmak istiyoruz. Yani alttaki gibi bir şekil:

\includegraphics[height=5cm]{4_1.png}

Herhangi bir nokta $P = (x,y,z)$ ne zaman bu düzlem üzerindedir? Eğer orijinden
$P$'ye giden vektör - $\vec{OP}$ vectörü -, düzlem normali ile doksan derece açı
oluşturuyorsa. Bunun nedeni, eğer $P$ noktası düzlem üzerinde ise, orijinden $P$
noktasına giden vektör düzleme paralel olacaktır. Düzlemin normal vektörünün de
düzleme kesinlikle dik olduğunu bildiğimiz için, $P$ noktası düzlem üzerinde
olduğu koşul için $\vec{OP} \cdot \vec{N} = 0$ ifadesini rahatlıkla yazabiliriz.

Dikkat edelim, $x,y,z$ kordinatlarını, tek başlarına kullanır kullanmaz, aslında
$\vec{OP}$'nin orijinden başlamasını şart koşmuş oluyorum, çünkü $x,y,z$
kordinatları sadece $(0,0,0)$ noktasına referansla anlamlılar.

Her neyse, bu çarpımı normal için verdiğimiz örnek sayılar için yaparsak,
sonuç $x+5y+10 = 0$ olacaktır.

2) Şimdi düzlem $P_0 = (2,1,-1)$ noktasından geçsin (orijinden değil), ve normal
yine aynı olsun, $\vec{N} = < 1,5,10 >$. Bu durumu zihnimizde canlandırmak için
yeni bir düzlemi hayal etmemiz lazım, ve $P$ noktası bu yeni düzlem üzerinde
olacak.

\includegraphics[height=6cm]{4_2.png}

$P$ ne zaman düzlem üzerinde? Bu soruyu incelemeden önce bir konuya dikkat
etmeliyiz. Resimden de farkedebileceğiniz gibi, normalleri paralel olan
düzlemler birbirlerine paraleldir. Bu yüzden, $P_0$ noktasından geçen diğer
düzlem üzerinde çizeceğimiz bir vektör -ki resimde bu vektör $\vec{P_0P}$
vektörüdür- orijinden geçen düzlemin normaline diktir. Şimdi sorumuza tekrar
dönecek olursak, eğer

$$ \vec{P_0P} \cdot \vec{N} = 0 $$

ise. O zaman 

$$ < x-2, y-1, z+1 > \cdot < 1,5,10 > = 0 $$

$$ x+5y + 10z = -3 $$

$x,y,z$ üzerinde çıkartma işlemini niye yaptım? Çünkü vektörün bir ucu hala
$x,y,z$ içeren $P$ noktasında, diğer ucu başlangıç noktası olan $P_0$'da.

İlk problemdeki sonuçtakiyle aradaki tek fark eşitliğin sağındaki -3 değeri. Bir
benzerlik ise her iki durumda da $x,y,z$ katsayılarının normal vektörün
değerlerine tekabül ediyor olması. Bu düzlemler hakkında önemli bir püf noktası,
eğer orjinden geçiyorlarsa eşitliğin sağ tarafı sıfır, başka bir yerden
geçiyorlarsa, başka bir değer. Peki bu -3 değerini daha hızlı bir şekilde
bulamaz mıydık? Bulabilirdik. Çünkü eşitliğin sol tarafının katsayılarını hızlı
bir şekilde bulabiliyoruz, orası tamam. Ayrıca düzlemdeki bir noktanın
kordinatlarını da biliyoruz, bu nokta düzlemin içinden geçmesini şart koştuğumuz
$P_0$ noktası. O zaman bu kordinatı $x,y,z$ terimlerini içeren formüle koyarsak,
eşitliğin sağ tarafını hemen hesaplarız.

$$ x+5y + 10z = 1(2) + 5(1) + 10(-1) = 2 + 5 -10 = -3$$

Bu arada bir düzlemin tek bir formülü yoktur, sonsuz tane denklemi
vardır. Mesela her şeyi 2 ile çarpsaydım

$$ 2x+10y+20z = -6 $$

olurdu, ve bu formülde aynı denklemin formülü olurdu. Bunun çokluğun sebebi
normal vektörlerin herhangi bir ``boyda'' olabilmesi, diklik için yön yeterli
olduğu için, farklı boylar ama değişmeyen yön hala aynı düzlemi tanımlıyor.

Düzlemi tanımlamak için normal vektör en önemlisi. Bir önceki derste düzlem
üzerindeki noktalar, onların ortaya çıkardığı iki vektör o vektörlerin çapraz
çarpımı üzerinden nasıl normal vektör hesaplanabileceğini görmüştük.

Soru:

Vektör $< 1,2,-1 >$ ve düzlem $x+y+3z = 5$ birbirine

\begin{itemize}
   \item Paralel
   \item Dik
   \item Hiçbiri
\end{itemize}

Cevaplayin. 

Vektörü ve düzlemin normal vektörünü çarptık. $< 1,2,-1 >\cdot< 1,1,3 > =
0$. 

Dogru cevap: "Paralel".

Şimdi bir lineer denklem sistemini inceleyelim.

$$ x + z = 1  $$

$$ x + y = 2 $$

$$ x + 2y + 3z = 3 $$

İlk iki denkleme bakalım. Bu denklem belli, özel iki $x,z$ noktasından
bahsediyor. İkinci denklemi de gözönüne alınca, aynı $x,z$ noktalarının ikinci
denklem için de geçerli olması gerekir.

\includegraphics[height=4cm]{4_3.png}

İlk iki denklemleri ayrı düzlemler olarak düşünürsek, çözüm olacak $x,y,z$ iki
düzlemin kesiştiği yerdedir. Peki üçüncü denklem, yani üçüncü düzlem ne yapar?

\includegraphics[height=4cm]{4_4.png}

O da iki düzlemin kesimindeki çizgiyi keser. Kesişimin kesimi bir
noktadır. O nokta da, üstteki lineer sistemin çözümü olan noktadır. 

Soru: 

Eğer 3 x 3 boyutlarındaki bir lineer sistemin çözümü bir nokta değilse,
nedir? 

\begin{itemize} 
\item Çözüm yoktur
\item İki nokta (2 çözüm)
\item Bir çizgi ($\infty$ tane çözüm)
\item Bir tetrahedron
\item Bir düzlem
\item Bilmiyorum
\end{itemize}

Diyelim ki ilk iki düzlemin kesişmesi bir çizgi ortaya çıkardı, ama bu çizgi
üçüncü düzlem ile paralel. O zaman çözüm yok demektir. Fakat şu da doğru
olabilir, belki bu çizgi üçüncü denklemin ``üzerindedir''. Bu durum cebirsel
olarak iki denklemin ortaya çıkardığı bir denklemin üçüncü denklemin katı
olmasıdır. Bu durumda üçüncü denklem bize hiçbir yeni bilgi sağlamamıştır. Bu
durumda sonsuz tane çözüm vardır, kesişmeden ortaya çıkan çizgi üzerindeki
``her'' nokta bir çözümdür, ve sonsuza kadar uzayan bir çizgi üzerinde sonsuz
tane nokta vardır.

O zaman doğru cevap "Çözüm yoktur", "Bir çizgi ($\infty$ tane çözüm)" ve "Bir 
düzlem".

5 niye doğru? Aynen iki denklemden ortaya çıkan denklemin üçüncü denklemin bir
katı olması gibi, her üç denklem ayrı ayrı birbirinin katı olabilir. O zaman bu
denklemler aslında aynı düzlemdirler. Çözüm bu tek düzlemdir, ve sonsuz
tanedir. Yani size aynı denklemi üç kere vermişim demektir, bu pek ilginç bir
sistem sayılmaz, ama yine de bu bir lineer sistemdir.


Peki $x + y + z = ..$ gibi bir denklemin sağındaki sıfır olmayan değerlerin
geometrik anlamı nedir? Cevap: Daha önce gördüğümüz $x + y + z = 0$ orijinden
geçer. Sağ taraf sıfır değilse, sıfırdan geçen aynı düzleme paralel ama ondan
belli miktarda uzakta bir düzlemden bahsediyoruz demektir. Ne kadar uzakta? Her
zaman eşitliğin sağındaki büyüklük kadar değil. O uzaklık için hesabın ayrıca
yapılması lazım. Şimdilik sadece orijinden uzakta olduğunu bilelim.

Şimdi matrislere dönelim. Önceki derste gördüğümüz lineer cebir formülünü
hatırlayalım

$$ AX = B $$

$$ X = A^{-1}B $$

Buradaki problem, bir matrisin her zaman tersini alamayacağımız gerçeği. 

Hatırlarsak 

$$ A^{-1} = \frac{1}{det(A)}adj(A) $$

Bu hesapta eğer determinant sıfır çıkarsa üstteki bölme işlemini yapamayız. Yani
bir önceki derste aslında şunu söylememiştik; bir matris şadece determinanti
sıfır değilse tersine çevirilebilir.

Geometrik olarak çözümün tek nokta olduğu durum, $A$'nin tersine çevirilebilir
olduğu durum. Kesişim çizgisinin üçüncü düzleme paralel olduğu durum ise
determinantın sıfır, yani tersine çevirim yapılamadığı durum.

Homojen Durum:

$AX = 0$ homojen durumdur, eşitliğin sağı sıfırdır, yani üç denklem
örneğinde tüm denklemlerin sağ tarafı sıfırdır. 

Örnek:

$$ x + z = 0 $$

$$ x + y = 0 $$

$$ x + 2y + 3z = 0 $$

Aslında bu denklemin bariz ve hep mevcut bir çözümünü zaten
biliyoruz. $x,y,z$'nin hepsi sıfır. Matematiksel terminolojide bu çözüme ``basit
çözüm (trivial solution)'' denir. Geometrik anlamı nedir? Her denklemin sıfıra
eşit olması, onların temsil ettiği her düzlemin orijinden geçtiği anlamına
gelir. Eh hepsi orijinden geçiyorsa, hepsi orada kesiyorlar da demektir. Basit
çözüm budur.

Burada iki durum daha var. 

1) Eğer $det(A) \ne 0$ o zaman $A$ tersine çevirilebilir, o zaman $X =
A^{-1} \ 0 = 0$. 

Başka hiçbir çözüm yoktur.

2) Eğer $det(A) = 0$ o zaman $det(\vec{N_1},\vec{N_2},\vec{N_3}) = 0$'dır. Her
iki hesap ta sıfır çünkü normal vektörün öğeleri her denklemin $x,y,z$ katsayısı
aynı zamanda, o katsayıları alıp $A$ içine koyuyoruz, bu matrisin determinantını
hesaplamak bir bağlamda normal vektörlerin determinantını hesaplamakla eşdeğer
oluyor. Devam edelim, üç formülü temsil eden üç düzlemin normal vektörleri
determinanti sıfır ise, o zaman $\vec{N_1},\vec{N_2},\vec{N_3}$ aynı
düzlemdedir.

\includegraphics[height=4cm]{4_5.png}

$det(\vec{N_1},\vec{N_2},\vec{N_3})$ hesabının paralelyüzün hacmini
hesapladığını hatırlayalım, ve bu hacim sıfır ise vektörlerin oluşturduğu hacim
sıfırdır, yani paralelyüz tamamen yassı demektir. O zaman vektörler aynı
düzlemdedir.

$\vec{N_1},\vec{N_2},\vec{N_3}$'nin aynı düzlemde ise, bu vektörlere aynı
anda dik olan bir çizgiyi düşünelim şimdi. İddia ediyorum ki o çizgi,
kesişme çizgisidir.

Niye? Çünkü kesişme çizgim tüm normal vektörlere aynı anda dik, yani o
normal vektörlerin temsil ettiği düzlemlerin hepsine aynı anda
paralel. Peki niye parallellik ötesinde, o düzlemlerin ``üzerinde''? Çünkü
çizgi orijinden geçiyor, ve tüm düzlemler de orijinden geçiyor. Bunun
olabilmesi için çizgimiz düzlemlerin üzerinde olmalı.

O zaman elimde $\infty$ tane çözüm vardır. 

Peki bu çözümleri nasıl bulurum? $\vec{N_1} \times \vec{N_2}$ hesabı
$\vec{N_1},\vec{N_2}$'ye dik bir üçüncü vektör hesaplar, bunu biliyoruz, bu
vektör de $\vec{N_3}$'e aynı şekilde dik olmalıdır çünkü bu üç vektörün
aynı düzlemde olduğunu biliyoruz. Bu basit olmayan çözümdür.

Üsttekiler homojen durum içindi. Şimdi homojen olmayan, genel duruma duruma
bakalım.

Genel Durum (General Case):

Eğer $det(A) \ne 0$ işe özgün (unique) bir çözüm vardır, $X = A^{-1}B$. 

Eğer $det(A) = 0$ işe, ya hiç çözüm yoktur ya da sonsuz tane çözüm
vardır. Tek bir çözüm olması mümkün değildir. 

\end{document}



