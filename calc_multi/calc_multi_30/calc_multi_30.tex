\documentclass[12pt,fleqn]{article}\usepackage{../../common}
\begin{document}
Ders 30

Çizgi Entegralleri (Line Integrals)

Diyelim ki bir vektör alanı $F = P i + Qj + Rk$ var, ve bu alan bir kuvvet
alanını temsil ediyor olabilir. Aynı uzayda bir eğri $C$ var, ve bu alan içinde
yapılan iş $W = \int_C \vec{F} \cdot \ud \vec{r}$ ile hesaplanabilir. Bu tanıdık
bir formülasyon tabii bir ek var, üçüncü kordinat $z$. Sonsuz küçük $\vec{r}$,

$$
\ud\vec{r} = < \ud x, \ud y, \ud z >
$$

olarak gösterilebilir, ve $\vec{F}$ ile noktasal çarpım yapınca tabii ki

$$
\int_C \vec{F} \cdot \ud \vec{r} =
\int_C P \ud x + Q \ud y + R \ud z
$$

sonucu elde edilir. Bu hala çizgisel entegral. Üstteki gerekli değerleri
soktuktan sonra temiz bir formüle dönüşecek. Metot düzlemde gördüğümüz durumla
aynı, eğriyi parametrize etmenin bir yolunu bulacağız, yani $x,y,z$
değişkenlerini tek bir değişken üzerinden göstereceğiz, sonra entegrasyonu
o tek değişken bağlamında yapacağız.

Örnek

$$
\vec{F} = < yz, xz, xy >
$$

$$
C: x=t^3, y=t^2, z=t, 0 \le t \le 1
$$

$\ud\vec{r}$ elde etmek için $C$ öğelerinin $t$'ye göre türevini alırız,

$$
\ud x = 3t^2, \ud y = 2t \ud t, \ud z = \ud t
$$

Nihai entegral hesabı için 

$$
\int_C \vec{F} \cdot \ud \vec{r} =
\int_C yz \ud x + xz \ud y + xy \ud z
$$

Biraz önce bulduğumuz $\ud x$, $\ud y$, $\ud z$ değerlerini üste koyarsak,

$$
= \int_C t^3 3t^2 \ud t + t^4 2t \ud t + t^5 \ud t
$$

$$
= \int_{0}^{1} 6 t^5 \ud t = t^6 \big\vert_{0}^{1} = 1
$$

Yani klasik yaklasimimi daha yuksek boyutta uyguladik.

Eger egri geometrik bir tarif uzerinden verilmis ise onu nasil parametrize
edecegimize kendimizin karar vermesi lazim. Parametrize etmek icin en iyi
degisken nedir? Ustte gordugumuz gibi bu parametre bir zamanimsi $t$ degiskeni
olabilir, ya da, kordinatlardan biri olabilir, yani $x,y,z$ degiskenlerinden
biri. Mesela ustteki ornekte $z$ de kullanabilirdim, o zaman egri $x=z^3$,
$y=z^2$ olurdu (tabii ki $z=z$). Acilari da kullanilabilir, ustteki ornekte
degil ama eger hareket bir cember, elips etrafinda olsaydi bunu yapabilirdim.






[devam edecek]

\end{document}
