\documentclass[12pt,fleqn]{article}\usepackage{../../common}
\begin{document}
Ders 28

Önceki derste bir yüzey içinden olan akış hesabını gördük. Bu bir çift
entegraldi,

$$
\int \int_S \vec{F} \cdot \hat{n} \ud S
$$

ki $\hat{n}$ yüzeye olan birim normal, $\ud S$ ise yüzeydeki alan öğesiydi.
Gördük ki farklı yüzeyler için farklı $\hat{n}$ ve farklı alan öğe formülü $\ud
S$ olabiliyordu. Bulmamız gereken yüzeyin ufak bir parçası için $\hat{n} \ud
S$'in ne olacağını bulmak.

Diyelim ki yüzeyin $xy$ düzlemine olan yansımasındaki / ``gölgesindeki'' ufak
bir dikdörtgeni alıyoruz, ki bu dikdörtgenin kenarları $\Delta x$ ve $\Delta y$,
ve onun yüzey $S$'deki karşılığına bakıyoruz.

\includegraphics[width=20em]{calc_multi_28_01.png}

Yani soru yüzeydeki o ufak parçanın alanı ve normal vektörün ne olduğu. Dikkat
edersek eğer yeterince ufak ise yüzeydeki o ufak parça bir paralelograma
benzeyecek. Kabaca tabii, belki biraz kavisi vs olacak ama yaklaşık olarak
bir düz paralelogram. Ve hatırlarsak uzayda bir paralelogramın alanını
çapraz çarpım ile nasıl hesaplayacağımızı gördük.

\includegraphics[width=10em]{calc_multi_28_02.png}

O zaman, eğer yeşil okla gösterilen o iki kenarın vektörünü bulabilirsek, alanı
hesaplayabileceğiz demektir çünkü çapraz çarpım sonucu olan vektörün büyüklüğü,
paralelogram alanına eşittir. Daha da iyisi çapraz çarpım sonucunu yönü bize
aradığımız bir diğer vektör, yüzey normalini de verecektir. Önceki derste işte
bu sebeple $\hat{n} \ud S$ hesabını bulmanın bazen daha rahat olduğunu
söylemiştim.

Vektörleri, ufak alanları daha iyi göstermek için bir resim,

\includegraphics[width=20em]{calc_multi_28_03.png}

$\vec{u}$ ve $\vec{v}$'yi bulalım, eğer onları bulabilirsek,

$$
\pm \vec{u} \times \vec{v} = \Delta S \cdot \hat{n}
$$

hesabını yapabiliriz.

Resme bakılınca $\vec{u}$ başlangıcı $x,y,f(x,y)$ bitişi $x+\Delta
x,y,f(x+\Delta x,y)$ noktasında.

Fakat $f(x+\Delta x,y)$ aslında bizde kısmi türev çağrışımı yapmıyor mu?
Evet. O zaman onu yaklaşık olarak şöyle temsil edebiliriz,
$f(x,y) + \Delta x f_x$. Demek ki soyle soylenebilir,

$$
\vec{u} \approx < \Delta x, 0, f_x \Delta x >
$$

Peki $\vec{v}$? Benzer sekilde,

$$
\vec{v} \approx < 0, \Delta y, f_y \Delta y >
$$

Paralelogramın iki kenarını bulmuş oldum. Şimdi çapraz çarpım yapalım. Ondan
önce biraz basitleştirerek,

$$
\vec{u} \approx < 1, 0, f_x  > \Delta x
$$

$$
\vec{v} \approx < 0, 1, f_y  > \Delta y
$$

Şimdi çapraz çarpım,

$$
\hat{n} \Delta S =
\vec{u} \times \vec{v} =
\left[\begin{array}{ccc}
i & j & k \\
1 & 0 & f_x \\
0 & 1 & f_y
\end{array}\right]
\Delta x \Delta y
$$

$$
= < -f_x, -f_y, 1 > \Delta x \Delta y
$$

Eğer diktörtgeni sonsuz küçültürsek, limite giderken daha önce gösterdiğimiz

$$
\hat{n} \ud S = \pm < -f_x, -f_y, 1 > \ud x \ud y
$$

formülünü elde ederiz. İşaret $\pm$ çünkü yukarı mı aşağı mı giden normal
vektörü seçmek bize kalmış.

Bu formül önemli bir formül, çünkü genel bir $f(x,y)$'yi baz alıyor, bu
sebeple pek çok alanda kullanım bulabilir, hatırlamamız iyi olur.

Örnek

Dikey bir vektör alanı düşünelim, $\vec{F} = z \hat{k}$. Bu $\vec{F}$'in
paraboloid $z = x^2 + y^2$'in $xy$'deki birim diske tekabül eden alanı
içinden olan akışını hesaplayın.

Bahsedilen alanı biraz daha iyi tarif etmek gerekirse,

\includegraphics[width=10em]{calc_multi_28_04.png}

Yani paraboloidin alttan yukarı giden sonsuz yüzeyi değil, sadece $xy$'de
yansıması görülen disk içine düşen kısmı ile ilgileniyoruz. 

$\int \int_S \vec{F} \cdot \hat{n} \ud S$ hesabını yapalım, $\vec{F}$ nedir?
$< 0, 0, z >$. $\hat{n}$ nedir? Daha doğrusu $\hat{n} \ud S$ nedir?
$< -f_x, -f_y, 1 > \ud x \ud y$ demiştik, $< -2x, -2y, 1 > \ud x \ud y$.

$$
\int \int < 0, 0, z > \cdot < -2x, -2y, 1 > \ud x \ud y
$$

$$
= \int \int_S z \ud x \ud y
$$

$z$'de kurtulmamız lazım, bunun için $z = x^2 + y^2$ formülünü kullanabiliriz,
yerine koyalım,

$$
= \int \int_S (x^2 + y^2) \ud x \ud y
$$

Dikkat edelim, çözümde sadece yüzeyde olanlara baktığımız gerçeğini
kullanıyoruz, $z$ yerine üstteki formülü koyabildik çünkü bu eşitlik yüzeyde
doğru.

Entegralin $x,y$ değişken menzillerine gelelim, bu değişkenler hangi değerler
arasında? Biraz önce bahsettik, birim disk içinde kalmamız gerekiyor.
Kutupsal kordinata geçebilirim.

$$
\int_{0}^{2\pi} \int_{1}^{0} r^2 \cdot r \ud r \ud \theta = \pi / 2
$$

Daha genel durumlara bakalım şimdi. Diyelim ki elimde öyle bir yüzey var ki
$z$'yi $x,y$ değişkenlerinin bir fonksiyonu olarak ifade edemiyorum. Fakat
bir şekilde parametrize edebiliyorum, yani yüzeyimin parametrik denklemleri
elimde var, bu demektir ki $x,y,z$ değişkenlerini diğer iki parametrik değişken
üzerinden temsil edebiliyorum. Bu tür bir daha genel problemi nasıl çözerdik?
Diyelim ki bu parametrik tanim verilmis,

$$
S = \left\{ \begin{array}{l}
x = x(u,v) \\ y = y(u,v) \\ z = z(u,v)
\end{array} \right.
$$

Bir yüzeyin temel özelliklerinden birisi üzerinde hareket edebileceğim sadece
iki tane bağımsız yön olduğu için o yüzeyin $x,y,z$ değişkenlerini iki tane
bağımsız değişken üzerinden tanımlayabilmem [üç boyutta bir eğriyi tek $t$
üzerinden tanımlayabilmemiz gibi].
















[devam edecek]

\end{document}
