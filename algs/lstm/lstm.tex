\documentclass[12pt,fleqn]{article}\usepackage{../../common}
\begin{document}
Uzun Kisa-Vade Hafiza Aglari (Long Short-Term Memory Networks, LSTM)

RNN'lerin biraz daha farkl� bir �e�idi LSTM. RNN'lerde i�i�e ge�en
fonksiyonlar sebebiyle hatalar ya �ok b�y�y�p ya da �ok
k���lebiliyor. Ger�i bu normal derin YSA'lerde de problem olabilirdi, fakat
RNN'lerde bu durum daha belirgin, ��nk� N ad�m geriye gitmek demek N kadar
i�i�e ge�en fonksiyon demektir, ve s�ral� veriyi tahmin i�in N'in b�y�k
olmas� gerekebilir. LSTM hatay� sabit bir seviyede tutarak yeni bir
yakla��m getirmi�. P�r RNN'lerde de baz� ��z�mler getirildi tabii.




















\end{document}

