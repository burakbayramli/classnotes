\documentclass[12pt,fleqn]{article}\usepackage{../../common}
\begin{document}
Temel Fizik 3, Basınç, Parçacıklar, Çarpışma, Hareket

Elastik Çarpışma (Elastic Collision)

$m_1,m_2$ kütlesine sahip $v_1,v_2$ hızında iki küre arasında mükemmel bir
elastik çarpışma olduğunu düşünelim, yani çarpışma öncesi ve sonrası enerji
kaybı yok, bu durumda, sistemin toplam momentumu da önce ve sonra aynı
olacaktır, 

$$
m_1 \vec{v}_1 + m_2 \vec{v}_2 = m_1 \vec{v}_1' + m_2 \vec{v}_2' 
$$

ki $\vec{v}_1,\vec{v}_2,\vec{v}_1',\vec{v}_2'$ hız vektörleri,
$\vec{v}_1',\vec{v}_2'$ çarpışma sonrası hız vektörleri.

\includegraphics[width=20em]{phy_005_basics_06.png}

Eğer momentum muhafaza ediliyorsa, birinci topun kaybettiği ya da kazandığı
momentum ikinci topa eklenecek ya da ondan çıkartılacaktır.

$$
m_1 \vec{v}_1 = m_1 \vec{v}_1' - \Delta \vec{p}
$$

$$
m_2 \vec{v}_2 = m_2 \vec{v}_2' + \Delta \vec{p}
$$

Üstteki idealize ortamda momentum transferi sadece çarpışma çizgisi üzerinde
olabilir, bu çizgi, ya da vektör yönü eğer iki topun arasında teğet bir düzlem
düşünsek ona dik olan bir vektör olacaktır, ona $n$ diyelim. O zaman, ve $p$
vektörünün büyüklüğünü $P$ ile gösterirsek,

$$
\vec{v}_1' = \vec{v}_1 - (P / m_1) \vec{n}
\mlabel{1}
$$

$$
\vec{v}_2' = \vec{v}_2 + (P / m_2) \vec{n}
\mlabel{2}
$$

Eğer $P$ skalar büyüklüğünü bulabilirsek, çarpışma sonrası yeni hızı elde
edebiliriz. 

Üstteki resme bakınca görüyoruz ki $v_1$ ve $v_2$ her biri iki tane ayrı
vektörün toplamı olarak temsil edilebilir, bu vektörlerden biri çarpışma,
momentum transfer çizgisine dik, diğeri ona paralel. Bu bilgi ile,
$v_1,v_1',v_2,v_2'$ şöyle temsil edilebilir,

$$
\vec{v}_1 = a_1 \vec{n} + b_1 \vec{q}, \qquad \vec{v}_2 = a_2 \vec{n} + b_2 \vec{q}
\mlabel{3}
$$

$$
\vec{v}_1' = a_1' \vec{n} + b_1' \vec{q}, \qquad v_2' = a_2' \vec{n} + b_2' \vec{q}
\mlabel{4}
$$

$a_1,a_2,b_1,b_2$ tek sayı değerleridir. 

(1) formülüne (3a)'yı sokarsak,

$$
v_1' = a_1 \vec{n} + b_1 \vec{q} - (P/m_1) \vec{n}
$$

$$
 = (a_1 - p/m_1) \vec{n} + b_1 \vec{q}
$$


$$
v_2' = a_2 \vec{n} + b_2 \vec{q} + (P/m_2) \vec{n}
$$

$$
= (a_2 + P/m_2) \vec{n} + b_2 \vec{q}
$$

Ve tabii ki form olarak $\vec{v}_1' = a_1' \vec{n} + b_1' \vec{q}$, ve
$\vec{v}_2' = a_2' \vec{n} + b_2' \vec{q}$ olduğunu biliyoruz, o zaman birbirine
tekabül eden kısımlara bakarak

$$
a_1' = a_1 - (P/m_1), \qquad b_1' = b_1
\mlabel{5}
$$

$$
a_2' = a_2 + (P/m_2), \qquad b_2' = b_2
\mlabel{6}
$$

Şimdi $P$ tek sayı değerini bulmak için enerji muhafazası formülünü
kullanabiliriz. Tek boyutta $1/2 m v^2$ şeklinde olan formülü $\frac{1}{2} m
\cdot \vec{v}\cdot\vec{v}$ olarak değiştirmek lazım. Ya da $\frac{1}{2} m
<\vec{v},\vec{v}>$, ya da $\frac{1}{2} m ||v||^2$.  O zaman

$$
\frac{m_1}{2} ||v_1||^2 + \frac{m_2}{2} ||v_2||^2  =
\frac{m_1}{2} ||v_1'||^2 + \frac{m_2}{2} ||v_2'||^2 
$$

$||v_1||^2$ ve $||v_1'||^2$, vs hesabının kolay bir yolu var, eğer üstteki resme
bakarsak mesela $||v_1||$ büyüklüğü kenarları $a_1$ ve $b_1$ olan bir üçgenin
hipotenüsü olarak görülebilir.

$$
\frac{m_1}{2} (a_1^2+b_1^2) + \frac{m_2}{2} (a_2^2+b_2^2) =
\frac{m_1}{2} (a_1'^2+b_1'^2) + \frac{m_2}{2} (a_2'^2+b_2'^2) 
$$

Daha önce bulduğumuz (5),(6) değerlerini üstteki formüle sokunca,

$$
\frac{m_1}{2} (a_1^2+b_1^2) + \frac{m_2}{2} (a_2^2+b_2^2) =
\frac{m_1}{2} \left( \left(a_1-\frac{P}{m_1} \right)^2 + b_1^2 \right)  +
\frac{m_2}{2} \left( \left(a_2-\frac{P}{m_1} \right)^2 + b_2^2 \right) 
$$

$b_1^2$ ve $b_2^2$ iptal olur. Her şeyi $P$ sol tarafta olacak şekilde tekrar
düzenlersek,

$$
P = \frac{2 m_1 m_2 (a_1-a_2)}{m_1+m_2}
$$


Bu degeri (1) ve (2)'ye sokarsak,


$$
\vec{v}_1' = \vec{v}_1 - \frac{2 m_2 (a_1-a_2)}{m_1+m_2} \vec{n}
$$

$$
\vec{v}_2' = \vec{v}_2 + \frac{2 m_1 (a_1-a_2)}{m_1+m_2} \vec{n}
$$

Üstteki formülü değişik kaynaklarda, mesela [3], biraz farklı formda görüyoruz,
mesela

$$
\vec{v}_1' =
\vec{v}_1 - \frac{2m_2}{m_1+m_2}
\frac{< \vec{v}_1-\vec{v}_2, \vec{x}_1-\vec{x}_2 >}{||\vec{x}_1-\vec{x}_2||^2}
(\vec{x}_1-\vec{x}_2)
$$

$$
\vec{v}_2' =
\vec{v}_2 - \frac{2m_1}{m_1+m_2}
\frac{< \vec{v}_2-\vec{v}_1, \vec{x}_2-\vec{x}_1 >}{||\vec{x}_2-\vec{x}_1||^2}
(\vec{x}_1-\vec{x}_2)
$$

Fakat biraz dikkat edilince mesela $a_1-a_2$'nin $\vec{n}$ yönündeki hız farkı
olduğunu görürüz, yani

$$
a_1-a_2=\frac{< \vec{v}_1-\vec{v}_2,\vec{x}_1-\vec{x}_2 >}{||\vec{x}_1-\vec{x}_2||}
$$

Geri kalanlardan zaten $\vec{n} = \vec{x}_1-\vec{x}_2/||\vec{x}_1-\vec{x}_2||$
ve $m_1,m_2$ değerleri de aynı şekilde iki tarafta uyar.

İki kütlenin eşit olduğu durumlarda (ki moleküler simülasyonlarda bu çok rahat
kabul edilebilir), formül daha da basitleşir [4],

$$
v_1' = v_1 - \left( (v_1-v_2)  \cdot \vec{n} \right) \vec{n}
$$

$$
v_2' = v_2 - \left( (v_2-v_1)  \cdot \vec{n} \right) \vec{n}
$$

ki $\vec{n} = \frac{x_1-x_2}{|x_1-x_2|}$

Basınç (Pressure) ve Parçacık Çarpışması

Bir sıvı içinde duran bir objeye tek uygulanan etki, stres onu
sıkıştıran türden bir etkidir. Diğer bir deyişle bir sıvı içindeki
objenin hissettiği kuvvet onun yüzeyine her zaman diktir.

\includegraphics[width=10em]{phy_005_basics_07.png}

Bir sıvının içindeki objeye uyguladığı basıncı, o objeye uygulanan
birim alanda uygulanan kuvvet olarak temsil edilebiliriz, kuvvet $F$
ve alan $A$ ise,

$$
P \equiv \frac{F}{A}
$$

Eğer belli bir noktadan bahsetmek istersek, diyelim $dA$ sonsuz
ufaklıktaki bir alana uygulanan $dF$ kuvveti,

$$
P = \frac{dF}{dA}
$$

O zaman belli bir alandaki basınç için o alan üzerinden entegral almak gerekir. 

Basıncın birimi $N / m^2$, şaşırtıcı olmasa gerek, kuvvet birimi
Newton, ve alan birimi $m^2$.

İdeal Gazlar Kanunu (İdeal Gas Law)

Peki mikro etkileşimlerden yola çıkarak basınç kavramını türetebilir miyiz
acaba? 19'uncu yüzyıl sonlarına doğru bu başarıldı ve gerçekten temel mekanik
kanunlarının basit bir model üzerinen makro açıklamalar yapabilmesinin çok güzel
bir örneği.

Basıncın gaz moleküllerinin bir yüzeye çarpmasından ortaya çıktığını
hatırlayalım. Bu kuvvet tabii ki Newton kanunundan hareketle,

$$
f = m a = m \frac{\ud v}{\ud t}
$$

Hız $v$'ye molekül içinde olduğu kabin / yüzey duvarına çarptığında ona dik olan
hız diyelim [1]. Bu türevi hesaplamak için, ki birim zamanda hız değişimi
gerekiyor, kenarları $L$ uzunluğunda bir küp içinde tek bir gaz molekül olduğunu
düşünelim.

Basitleştirme amacıyla diyelim ki bu molekül sürekli küp kutu içinde ileri geri
gidip geliyor, bir duvara çarpınca bir süre sonra geri geliyor. Bu molekül bir
duvara çarptığında $v$ hızında çarptığında (yani $mv$ momentumuyla) elastik
olarak geri sekecektir, ve $-v$ ile tam ters yöne geri gitmeye başlayacaktır.

\includegraphics[width=20em]{phy_005_basics_04.png}

O zaman her çarpışma için hız değişimi $2v$, momentum değişimi ise $2mv$
olur.

Tabii aslında eğer daha genel formülize etmek gerekirse bu çarpışma sırasında
$\bar{v}$ hızının duvara dik olan bileşeni $v_x$'yi düşünüyoruz.

\includegraphics[width=20em]{phy_005_basics_05.png}

Yani momentum degisimi

$$
\Delta p_x = (-m v_x) - (m v_x) = - 2 m v_x 
$$

Yani duvara transfer edilen momentum $2 m v_x$. 

Birim zaman $\Delta t$'ye bir molekün iki çarpışma arasında geçen zaman dersek,
ve $v_x$ hızında $2L$ yol katedilmişse, $\Delta t  = 2 L / v_x$ demektir, ve

$$
F = \frac{\Delta p_x}{\Delta t} = \frac{2 m v_x}{2 L / v_x} = \frac{m v_x^2}{L}
$$

Basınç birim alana uygulanan kuvvettir, ve küpün bir kenarının $L^2$ alaninda
olduğunu düşünürsek, 

$$
P = \frac{m v^2}{L^3} = \frac{m v^2}{V}
$$

$V$'yi kutunun hacmi olarak aldık, ve $V = L^3$.

Birden fazla molekülü düşünmek istiyoruz şimdi, mesela bir averaj
üzerinden.. Fakat her molekül hem negatif hem pozitif yönde aşağı yukarı aynı
miktarda hareket yapar (rasgele hareket olduğu için) ve bu tür bir hareket
üzerinden averaj almak bizi sıfır değerine götürür. Bu sebeple ortalamasını
almadan önce hızların karesini almak istiyoruz,

$$
\bar{v^2} = \frac{v_1^2 + v_2^2 + ... + v_N^2 }{N} = \frac{\sum_i v_i^2}{N}
$$

ve ortalama değeri bulmak için $\sqrt{\bar{v^2}}$ kullanıyoruz. Bu hesaba kök
kare ortalaması (root mean square -RMS-) ismi de verilir. Şimdi tüm $N$
moleküller üzerinden bir basınç hesaplamak istersek, $N$ tane molekül, ama belli
bir anda sadece Kartezyen kordinat sisteminde sadece üç yönden sadece biri
yönünde etki var, o zaman $N$ ile çarpıp 3'e bölmek lazım, 

$$
P = \frac{N}{3} \frac{m \bar{v^2}}{V}
$$

Bu formül içinde bir kinetik enerji formülasyonu görülebiliyor, averaj kinetik
enerjiye $\epsilon = m \bar{v^2} / 2$ dersek, üstteki formülü

$$
PV = \frac{N}{3} m \bar{v^2} = \frac{2}{3} N \epsilon
$$

olarak yazabiliriz.

Eğer bu formülü sıcaklık içerek şekilde değiştirmek istiyorsak; biliyoruz ki
sisteme eklenen her Joule enerji ve bir derece sıcaklık değişimi arasındaki
ilişkiyi $k$ sabiti kontrol eder [5, 29-16] bu sabit $k = 1.38 x 10^{23}$ Joule
/ Kelvin'dir, o zaman enerjiden sıcaklığa geçiş için $kT$ kullanabiliriz, hatta
bir $3/2$ eklenerek üstteki 2/3 iptali amaçlanır,

$$
\epsilon = \frac{3}{2} k T
$$

Ve,

$$
PV = \left( \frac{2}{3} N \right) \left( \frac{3}{2} k T \right) = N k T
$$

Devam edelim, $n = N / N_A$ olduğunu da biliyoruz ki $N_A = 6.02 x 10^{23}$,
Avagadro'nun sayısı, $n$ örneklemdeki mol sayısı, $N$ ise örneklemdeki tüm
moleküller [2, sf. 550],

$$
PV = n N_A k T
$$

Tabii bu bizi $R$ denen bir diğer sabite götürüyor, $R = 8.31 J/mol \cdot
K$. Onun $k$ ve $N_A$ ile ilişkisi şöyle,

$$
k = \frac{R}{N_A}
$$

O zaman,

$$
PV = n R T
$$

İdeal gazlar kanununa erişmiş olduk.


Parçacıklar, Çok Parçacıklı Sistemler, Katı Gövde (Rigid Body)

Kütle Merkezi

Küresel, birörnek parçacıklara uygulanan kuvvetin onların merkezine uygulandığı
farzedilebilir, ya da parçacık sanki noktasal bir parçacık imiş gibi işlemler
yapılabilir. Fakat bu faraziyeyi nasıl doğrularız? Ayrıca farklı şekillerde,
homojen bir maddeden oluşmayan objeleri nasıl idare ederiz?

Burada kütle merkezi (center of mass) kavramı yardımcı olur, mesela $N$ tane
$m_i$ kütlesine sahip parçacığın kütle merkezi

$$
r_c =
\sum _{i=1}^{N} \frac{m_i}{m} r_i =
\frac{1}{m} \sum _{i=1}^{N} m_i r_i
$$

ile hesaplanabilir. Bu bir nevi bir ağırlıklı ortalama. $O$ bir referans
noktasıdır, mesela kordinat orijini, $r_i$ ise $O$'dan parçacık konumuna işaret
eden bir vektör, parçacığın yeri [6, sf. 77].

\includegraphics[width=15em]{phy_005_basics_08.png}

Dikkat edersek parçacıklar ayrı ayrı duruyor, ama birazdan göreceğiz ki bu
parçacıklar tek bir katı gövde objenin içinde olsalar da hesap geçerli
oluyor. Hatta gövde katı olmasa bile, mesela titreşen, salınımda olan
moleküllere, öğelere sahip objeler bile belli şartlara uyuyorlarsa aynı kurallar
ile idare edilebiliyorlar. $F_{ij}$ her $i$ parçacığının $j$ parçacığından
hissettiği kuvvet olsun, ve $F_i$ değeri de $i$ parçacığı üzerindeki tüm dış
kuvvetlerin toplamı olsun. Newton'un ikinci kuralını her parçacık için yazarsak,
$i$'nin hissettiği tüm kuvvetlerin toplamı solda kütle çarpı ivmeye eşit
olacaktır,

$$
m_i \ddot{r}_i = \sum _{j \ne i} F_{ij} + F_i
$$

ki $\ddot{r} = \ud^2 / \ud t^2 r$, zamana göre ikinci türev anlamında.

Fakat Newton'un etki-tepki üçüncü kuralına göre her parçacık uyguladığı kuvvet
kadar ters yönde geri bir kuvvet hisseder, yani $F_{ij} = -F_{ji}$, olmalı, o
zaman her iki tarafta tüm $i$'ler üzerinden bir toplama yapınca parçacık
sistemindeki bu ``iç'' kuvvetler birbirini nötralize etmeliler, üstteki
formülden geriye kalanlar,

$$
\sum m_i \ddot{r}_i =  \sum F_i
$$

olur. $\sum F_i$ değeri sisteme uygulanan tüm dış kuvvetler $F_{ext}$ olarak
görülebilir. Sol tarafı $m$ ile hem bölüp, hem çarpalım,

$$
m \frac{\ud^2}{\ud t^2} \underbrace{\sum \frac{m_i}{m} r_i}_{r_c} =
F_{ext}
$$

İşaretli yerde bir $r_c$ elde ettiğimizi görüyoruz, o zaman 

$$
m\ddot{r}_c = F_{ext}
$$

diyebiliriz, yani tüm sistemin kütle merkezi tek bir parça gibi görülebilir,
hesaplarda $m$ ve $F_{ext}$ ile beraber bu şekilde kullanılabilir, ki
o merkezin ivmesi $\ddot{r_c}$ olacaktır.

İşin ilginç tarafı bu sistem ayrı ayrı parçalar olsun, ya da aynı katı gövde
içindeki ``bloklar'' olsun bu hesap yine işliyor olacaktır, çünkü düşünürsek bir
katı obje içindeki blokların, parçacıkların da birbiri üzerinde kuvvetleri de
birbirini nötralize ederler, hatta katı olmayan ama parçaları titreşen obje için
de aynı argüman, ve aynı formül kullanılabilir.

Toplam Açısal Momentum

Toplam lineer momentum oldukca basit,

$$
P = \sum_i p_i = \sum_i m_i \dot{r}_i = M \dot{R}
$$

Yani kütle merkezi üzerinden tüm kütle için bir momentum kullanılabilir. Kütle
merkezinin toplam açısal momentumun hesabında oynadığı rol biraz daha karmaşık
ama bir o kadar önemli [7, sf. 368].

Yine bir katı gövde düşünelim, $N$ tane her biri $m_i$ ağırlığında parçacıktan
oluşuyor olsun, altta resmedildiği gibi, kütle elipsoid şeklinde.

\includegraphics[width=15em]{phy_005_basics_09.png}

$m_i$'nin herhangi bir şekilde seçilen referans / orijin noktası $O$'ya göre
olan yeri $r_i$ ile gösteriliyor. Kütle merkezi (center of mass) CM ile,
CM'nin $O$'ya göre yeri $R$, $m_i$'nin CM'ye izafi olarak yeri $r_i'$. Yani

$$
r_i = R + r_i'
\mlabel{7}
$$

doğru olacak. Şimdi $O$'ya göre parçacık $i$ için açısal momentum hesaplarsak

$$
l_i = r_i \times p_i = r_i \times m_i \dot{r}_i
$$

$l,p,r$ vektörsel değerler.

O zaman $O$ referansina izafi olarak toplam momentum

$$
L = \sum_i l_i = \sum_i r_i \times p_i = \sum_i r_i \times m_i \dot{r}_i
$$

Eğer (7) denklemini üste sokarsak,

$$
L = \sum_i (\dot{R} + \dot{r}_i') \times m_i (\dot{R} + \dot{r}_i' )
$$

Şu kuralı [3] kullanarak

$$
(A + B) \times (C+D) = (A \times B) + (A \times C) + (B \times C) + (B \times D) 
$$

açılımı yapabiliriz, 

$$
 = \sum_i (\dot{R} + \dot{r}_i') \times (m_i \dot{R} + m_i \dot{r}_i' )
$$


$$
= \sum_i (R \times m_i \dot{R}) +
  (R \times m_i \dot{r}_i') +
  (r_i' \times m_i \dot{R}) +
  (r_i\ \times m_i \dot{r}_i')   
$$

$M = \sum_i m_i $ diyelim, ki $M$ tüm kütle,

$$
L = R \times M\dot{R} +
R \times \sum m_i \dot{r}_i' +
(\sum r_i' m_i) \times \dot{R} +
\sum r_i' \times m_i \dot{r}_i'
\mlabel{8}
$$

Bu son denklem üzerinde oldukca fazla basitleştirme mümkün. Mesela üçüncü
terimde parantezler içinde olan formül sıfıra eşit, niye? Kütle merkezi
formülünü yazarsak,

$$
R = \frac{1}{M} \sum m_i r_i
$$

$$
MR = \sum m_i r_i
$$

Aynı şekilde izafi açılımı yapalım,

$$
= m_i (R + r_i') = \sum m_i R + \sum m_i r_i'
$$

Yani

$$
MR = MR + \sum m_i r_i'
$$

elde ederiz, ve bu formülün doğru olması için $\sum m_i r_i' = 0$ olması gerekir.

Yani (8) denkleminin üçüncü terimi sıfır. O terimin türevini alırsak
ikinci terimin elde edileceğini görürüz, o zaman o da sıfır olur. Geriye
kalanlar,

$$
L = R \times P + \sum_i r_i' \times m_i \dot{r}_i' 
$$

Birinci terim kütle merkezinin $O$'ya göre açısal momentumu, ikincisi hareketin
kütle merkezine göre olan açısal momentumu. Üstteki formül sayesinde bir
katı kütlenin açısal momentumunu iki parçaya bölerek düşünmek mümkün oluyor.

Şimdi şu soruyu soralım: katı kütlenin açısal momentum değişimi (dönüşsel
kuvvet, tork) ona uygulanan, ve yine onun CM'sine göre uygulanmış dış torka eşit
midir? Cevap evet olacak ama bu aslında ilk başta çok bariz olmayabiliyor, çünkü
CM ivmeleniyor, acaba öyle mi değil mi? Fakat CM ivmeleniyor olsa bile cevap
değişmiyor.

İspatlamak için CM etrafındaki açısal momentumu formülize edelim [9, sf 37],

$$
L_{CM} = \sum r_i' \times m_i \dot{r}_i' 
$$

CM etrafında uygulanan tork, ustteki açısal momentumun zamansal türevidir, 

$$
\dot{L}_{CM} = \sum \dot{r}_i \times m_i \dot{r}_i' + \sum r_i' \times m_i\ddot{r}_i'
$$

Eşitliğin sağındaki ilk terim sıfır çünkü iki paralel vektörün çapraz çarpımı
sıfırdır. Kalanlarla devam edersek, $r_i = r_i' + R$ olduğu için oradan
$\ddot{r}_i = \ddot{r}_i' + \ddot{R}$ ve oradan $\ddot{r}_i' = \ddot{r}_i -
\ddot{R}$ diyoruz, üstteki ikinci terimde yerine koyunca

$$
= \sum r_i' \times m_i (\ddot{r}_i - \ddot{R})
$$

$$
= \sum (r_i' \times m_i \ddot{r}_i) - (r_i' \times m_i \ddot{R})
$$

$$
= \sum (r_i' \times m_i \ddot{r}_i) - \sum (r_i' \times m_i \ddot{R})
$$

İlk terimde kuvvet tanımı görülüyor,

$$
= \sum (r_i' \times F_i ) - \sum (r_i' \times m_i \ddot{R})
$$

İkinci terimde $m_i$ yer değiştirebilir,

$$
= \sum (r_i' \times F_i ) - \sum (m_i r_i' \times \ddot{R})
$$

$\sum m_i r_i'=0$ olduğunu hatırlarsak ikinci terim kaybolur,

$$
= \sum r_i' \times F_i 
$$

Geriye kalan dışarıdan CM etrafında uygulanan tork tanımı değil midir? Evet.  O
zaman dışarıdan CM etrafında uygulanan tork ile başladığımız ifade
$\dot{L}_{CM}$ yani CM etrafındaki momentum değişimi arasındaki eşitliği
ispatlamış olduk.

Kaynaklar

[1] Chang, {\em Physical Chemistry for the Biosciences},
    \url{https://chem.libretexts.org/@go/page/41408}

[2] Resnick, Fundamentals of Physics, 10th Ed

[3] Wikipedia, {\em Elastic collision}, \url{https://en.wikipedia.org/wiki/Elastic_collision}

[4] Masson, {\em Elastic Collisions in 3D}, \url{https://exploratoria.github.io/exhibits/mechanics/elastic-collisions-in-3d/index.html}

[5] Feynman, {\em Feynman Lectures on Physics, I}

[6] Levi, {\em Classical Mechanics with Calculus of Variations and Optimal Control}

[7] Taylor, {\em Classical Mechanics}

[8] Bayramli, {\em Cok Degiskenli Calculus, Ders 3}

[9] Taylor, {\em Classical Mechanics Problem Solutions}

\end{document}
