\documentclass[12pt,fleqn]{article}\usepackage{../../common}
\begin{document}
Muhafaza Kanunları

Kütle Muhafazası, Süreklilik Formülü (Continuity Equation)

Ufak bir kutu ya da hacim ögesi düşünelim [4, 7 sf. 50, 8 sf. 10, 10 sf. 404, 12
  sf 95, 11 sf. 9], kenarları çok küçük $\Delta x$, $\Delta y$, $\Delta z$
boyutlarında, ve bu kutu uzayda sabitlenmiş, içinden sıvı akıyor. Sabitlenmiş
bir noktada olanlara baktığımız için bu Euler bakış açısı, detaylarını ileride
göreceğiz. Neyse, şimdi sadece $x$ yönündeki kütle değişimine bakalım,

\includegraphics[width=25em]{phy_050_fluid_02.jpg}

Sıvının akış hızı $u(x,y,z) = (u,v,w)$, ve yoğunluğu $\rho$ olsun. Birim zamanda
içeri akan net kütleyi giren eksi çıkan kütle olarak hesaplayacağız. Bu zamanda
$x$ yönünde akış kenarları $\Delta y$, $\Delta z$ ve $u(x,y,z)$ (ufak dik duran
bir pizza kutusu gibi) olan bir kesit düşünebilir. Bu birim zamandaki akış
hacmi.  Onu yoğunluk ile çarpınca kütle elde edilir, aynı şeyi $x + \Delta x$
noktası için de yaparız, ve farklarını alırız,

$$
\rho(x,y,z)u(x,y,z)\Delta y \Delta z -
\rho(x+\Delta x,y,z) u(x+\Delta x,y,z)\Delta y \Delta z
$$

Üstteki formüldeki bir bölüm bir kısmi türevi andırıyor, $\Delta x$ ile bölüp
çarpsak,

$$
= \frac{\rho(x,y,z)u(x,y,z)}{\Delta x}\Delta x \Delta y \Delta z -
\frac{\rho(x+\Delta x,y,z) u(x+\Delta x,y,z)}{\Delta x} \Delta x \Delta y \Delta z
$$

Evet iki bölme işlemini yaklaşıksal kısmi türev olarak görebiliriz,

$$
\approx -\frac{\partial (\rho u) }{\partial x} \Delta x \Delta y \Delta z
$$

Benzer işlemi tüm eksenler için ayrı ayrı yapsak onlar için de kısmi türevler
elde ederdik, o zaman tüm eksenler üzerinden olan değişim, ve farksal hacmi
$\Delta V = \Delta x \Delta y \Delta z$ olarak göstererek,

$$
\left(
-\frac{\partial (\rho u) }{\partial x} 
-\frac{\partial (\rho u) }{\partial y} 
-\frac{\partial (\rho u) }{\partial z} 
\right) \Delta V
$$

Üstteki gradyan vektörsel olarak daha rahat ifade edilebilir,

$$
= -\nabla \cdot (\rho \bar{u} ) \Delta V
$$

Birim zamandaki kütle artışı buna eşit. Üstteki ifadeyi farklı bir açıdan, birim
zamandaki kütle (yoğunluk çarpı hacim) artışı olarak, şöyle de belirtebilirdik,

$$
\frac{\partial }{\partial t} (\rho \Delta V) 
$$

Bu formül iki üstteki formül ile eşit olmalı. Ayrıca her iki tarafta $\Delta V$
var, çıkartılabilir, zaten sabit bir hacim, sonuç,

$$
\frac{\partial \rho}{\partial t}  = -\nabla \cdot (\rho \bar{u} )
$$

Literatürde çoğunlukla şu formda gösterilir,

$$
\frac{\partial \rho}{\partial t}  + \nabla \cdot (\rho \bar{u} ) = 0
\mlabel{1}
$$

Bu denkleme süreklilik formülü (continuity equation) ya da kütle muhafaza kanunu
(mass convervation law) ismi veriliyor.

Bilinen Vektör Calculus eşitliğinden hareketle

$$
\frac{\partial \rho}{\partial t}  +
\bar{u} \cdot \nabla \rho +
\rho \cdot \nabla \bar{u} = 0
$$

Eğer bir sıvı sıkıştırılamaz (incompressible) ise, ki pek çok sıvı dinamiği
simülasyonlarında böyle olduğu kabul edilir, o zaman $\rho = sabit$ demektir,
$\frac{\partial \rho}{\partial t} = 0 $ olur, değişim yok, süreklilik
formülünden geri kalan

$$
\nabla \cdot \bar{u} = 0
\mlabel{2}
$$

olacaktır. $\nabla \cdot$ sembolünün uzaklaşım, ya da $\bdiv$ olduğunu
hatırlayalım, ve uzaklaşım yaklaşık olarak bir bölgeye giren ekşi çıkan akışı
gösterir, ve sıkıştırılamaz durumda uzaklaşımın sıfır olması mantıklı.

$\rho \bar{u}$ büyüklüğüne kütle akışı diyebiliriz, kütlenin ne hızla aktığını
gösterir. Uzaklaşım, akımın bir bölgede nasıl yayıldığını gösteriyorsa (bazı
durumlarda sanki orada bir su / sıvı kaynağı varmış gibi) o zaman süreklilik
formülünün fiziksel olarak şunu söylüyor denebilir; bir sıvının bir bölgedeki
yoğunluk değişimi o bölgeye giren ve çıkan akımların sonucudur.

Euler ve Lagrange

Euler ve Lagrange bakış açısı arasındaki farklarla başlayalım. Bu iki bakış
açısı bir sıvının dinamiğini nasıl incelediğimiz ile alakalı. Eğer bir nehirdeki
kirlilik yoğunluğunu ölçüyorsak mesela, bunu herhangi bir $x,y,z$ noktasında
yapabiliriz, ve diyelim ki kirlilik belli bir yerde hiç değişmiyor, ertesi gün
gelsek aynı yerde aynı ölçümü alıyoruz [2, sf 78]. Bu yere bağımlı Euler açısı.

Fakat farklı yerlerde farklı ölçümler olabilir, mesela nehir boyunca bir kayık
içinde sabit hızda gidersek yoğunluk lineer oranda artıyor. Bu durumda pir paket
sıvıyı takip ettiğimizi düşünebiliriz, o paketin açısından elde edilen ölçümler
Lagrange bakış açısıdır. 

İki bakış açısı arasında gidip gelmenin yolu materyel türev. Böylece Euler
bazındaki değişim kullanılarak Lagrange tarifi yapılabiliyor. Bu önemli çünkü
ölçümler çoğunlukla Euler formatında düşünülür (bir yerde duran ölçüm aleti
idare etmesi ve düşünmesi daha rahat bir kavramdır), ayrıca matematik Euler
ortamında biraz daha kolay manipüle edilebilir hale geliyor [3].

Lagrange ile bir parçacık hayal ediyoruz, onu tanımlamanın bir yolu $t=0$ anında
nerede olduğu. Daha sonra bu başlangıç noktasındaki sıvı paketinin hangi yolu
takip ettiğini $\bar{r}(t)$ ile tarif ediyoruz, ki $\bar{r}(t)$ parametrik bir
eğri olarak düşünülebilir, $r = ( x(t), y(t), z(t) )$. Eğer bir başlangıç
noktasını $a$ olarak tanımlarsak bu başlangıcın ve yol denkleminin bir parçacığı
tarif ettiğini düşünebiliriz,

\includegraphics[width=20em]{phy_050_fluid_01.png}

Şimdi herhangi bir ölçümü düşünelim [4], biraz önce kirlilik örneği verdik, bu
sıcaklık ta olabilirdi, ölçüm $F(t,x,y,z)$ olsun, $t$ anında ve $x,y,z$
noktasında yapılan ölçüm, bu ölçüme Calculus'un Zincirleme Kuralını uygularsak,
değişim oranını materyel türev $D F / Dt$'yi nasıl elde edebileceğimizi
görebiliriz,

$$
\frac{D F}{D t} =
\frac{\partial F}{\partial t} +
\frac{\partial F}{\partial x} \frac{\partial x}{\partial t} + 
\frac{\partial F}{\partial y} \frac{\partial y}{\partial t} + 
\frac{\partial F}{\partial z} \frac{\partial z}{\partial t} 
$$

$(\frac{\partial x}{\partial t}, \frac{\partial y}{\partial t},\frac{\partial
z}{\partial t})$ hız olarak görülebilir, ona $\bar{u} = (u,v,w)$ vektörü diyelim,

$$
\frac{D F}{D t} =
\frac{\partial F}{\partial t} +
\frac{\partial F}{\partial x} u + 
\frac{\partial F}{\partial y} v + 
\frac{\partial F}{\partial z} w 
$$

Ayrica $(\frac{\partial F}{\partial x},\frac{\partial F}{\partial y},\frac{\partial F}{\partial z})$
gradyan vektoru $\nabla F$,

$$
\frac{\partial F}{\partial t} + \bar{u} \cdot \nabla F
$$

Burada $\frac{\partial F}{\partial t}$ ölçülen $F$'nin tek, sabit bir yerde
zamana göre değişimidir. Bu terime yapılan ekler hareket halindeki parçanın ek
olarak göreceği ölçüm değişim oranı olacaktır.

Alınan türev bir operatör olarak görülebilir, 

$$
\frac{D ()}{D t} = \frac{\partial () }{\partial t} + \bar{u} \cdot \nabla ()
$$

Üzerinde operatör uygulanan $()$ içine gider, $F$ için

$$
\frac{D F}{D t} = \frac{\partial F}{\partial t} + \bar{u} \cdot \nabla F
$$

ile önceki formüle eriştik.

Şimdi ilginç bir noktaya geldik, süreklilik denklemi (1)'i, $\rho$ ölçümü
üzerinde materyel türev uygulanmış formu olarak görmek mümkün,

$$
\frac{D \rho}{D t} + \rho \nabla \cdot \bar{u} = 0
$$



Kaynaklar

[1] Aerodynamics for Engineering Students

[2] Storey, {\em Fluid Dynamics}

[3] Lumley, {\em Eulerian and Lagrangian Descriptions in Fluid Mechanics},
    \url{https://www.youtube.com/watch?v=XDrt-uATAY8}

[4] Berloff, {\em Introduction to Geophysical Fluid Dynamics},
    \url{https://wwwf.imperial.ac.uk/~pberloff/gfd_lectures.pdf}

[6] Landau, {\em Computational Physics}
    
[7] Anderson, {\em Computational Fluid Dynamics, the basics with applications}

[8] Liu, {\em Particle Methods for Multi-scale and Multi-physics}

[10] Kreyzig, {\em Advanced Engineering Mathematics, 10th Edition}

[11] {\em Mathematics, Numerics, Derivations, and OpenFOAM}

[12] {\em Introduction to Atmospheric Physics, 2nd Edition}

\end{document}

