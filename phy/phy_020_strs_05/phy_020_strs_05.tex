\documentclass[12pt,fleqn]{article}\usepackage{../../common}
\begin{document}
Materyel Mekaniği - 5

Bükülen bir çubuğun formüllerine bir giriş [1] kaynağında yapıldı. Orada
moment-eğri (moment-curvature) formülü gösterilmişti.

$$
M(x) = \frac{E(x)I(x)}{\rho(x)}
\mlabel{1}
$$

Şimdi bu formülü genişletelim, ve bir ikinci derece türeve eşitleyelim.

\includegraphics[width=10em]{phy_020_strs_05_01.jpg}

Resimde gösterilen semboller $M$ bükme momenti, $\rho$ çubuğun $+y$ tarafındaki
bükülme çemberinin, eğiminin yarıçapı (radıus of curvature).  $v$ ise yine $+y$
kısmındaki yer değişimidir. Çubuğa uygulanan kuvvet dağılımının ne olduğu
önemli değil, sonuçta odaklandığımız çubuğun ufak bir kısmı.

[3] kaynağında bir çemberi (yarıçapını) onun bir eğriye dokunduğu noktadaki
türevler üzerinden temsil etme tekniğini paylaştık. Bu formülü mevcut probleme
uygulayabiliriz.

\includegraphics[width=20em]{phy_020_strs_05_02.jpg}

Üstteki örnekte çemberin yarıçapı $\rho$, türevler ise $\ud v / \ud x$. Formül,

$$
\frac{1}{\rho} =
\frac
{\dfrac{\ud^2 v}{\ud x^2}}
{ \left[ 1 + \left( \dfrac{\ud v}{\ud x}  \right)^2 \right]^{3/2} }
$$

Üstteki problemde eğim çok ufaktır o zaman $\ud v / \ud x$ ufak kabul edilir
(resimdeki eğim eğitim amaçlı abartılmış), demek ki bölendeki kare hesabı
daha da ufalır, geriye sadece 1 kalır, 1 ile bölümü yok sayarız, geriye kalanlar

$$
\frac{1}{\rho} \approx \frac{\ud^2 v}{\ud x^2}
\mlabel{2}
$$

Şimdi (1) formülünü tekrar düzenlersek,

$$
\frac{1}{\rho(x)} = \frac{M(x)}{E(x) I(x)}
$$

diyebilirdik. Bu formülün sol tarafının (2) sol tarafı ile aynı olduğunu
görüyoruz. Demek ki onları eşitleyebiliriz, moment-eğri formülü şu hale gelir,

$$
\frac{\ud^2 v}{\ud x^2} = \frac{M(x)}{E(x) I(x)}
$$

Kaynaklar

[1] Bayramlı, {\em Fizik, Materyel Mekaniği - Hazırlık}

[2] Craig, {\em Mechanics of Materials, Third Edition}

[3] Bayramlı, {\em Çok Değişkenli Calculus, Eğrilik (Curvature)

\end{document}
