\documentclass[12pt,fleqn]{article}\usepackage{../../common}
\begin{document}
Materyel Mekaniği - 6

Dönüş (Rotation)

Alttaki gibi bir kiriş düşünelim,

\includegraphics[width=20em]{phy_020_strs_06_01.jpg}

Daha önce bu tür bir kiriş üzerinde eksenel yöndeki kuvvetler ve yer
değişimlerinin ilişkisini

$$
\left[\begin{array}{c}
f'_{1x} \\ f'_{2x}
\end{array}\right] =
\frac{AE}{L}
\left[\begin{array}{cc}
1 & -1 \\ -1 & 1
\end{array}\right]
\left[\begin{array}{c}
u'_1 \\ u'_2
\end{array}\right]
\mlabel{1}
$$

olarak göstermiştik. Üstte yazılan kirişin yerel, kendisine has kordinat
sistemini baz alıyor. Eğer üstteki değişkenleri global kordinat sistemine
eşlemek, yansıtmak istiyorsak o zaman sistemi görülen $\theta$ kadar döndürmemiz
gerekiyor. Döndürme işlemi genel olarak iki boyuttaki bir $[u, v]$ vektörü için
[1, sf. 85]

$$
\left[\begin{array}{c}
u' \\ v'
\end{array}\right] =
\left[\begin{array}{cc}
C & S \\ -S & C
\end{array}\right]
\left[\begin{array}{c}
u \\ v
\end{array}\right]
\mlabel{2}
$$

ile yapılır, ki $C = \cos\theta$, $S = \sin\theta$.

Fakat unutmayalım tek eksenlikten çıktığımız zaman kirişin her ucunda iki
serbestlik derecesi vardır, her uç $u,v$ yönünde yer değişim yaşayabilir,
bunları $u_1,v_1$ ve $u_2,v_2$ diye gösterebiliriz. O zaman dönüş hesabı

$$
\left[\begin{array}{c}
u'_1 \\ v'_1 \\ u'_2 \\ v'_2
\end{array}\right] =
\left[\begin{array}{cccc}
C & S & 0 & 0 \\
-S & C & 0 & 0 \\
0 & 0 & C & S \\
0 & 0 & -S & C 
\end{array}\right]
\left[\begin{array}{c}
u_1 \\ v_1 \\ u_2 \\ v_2
\end{array}\right]
$$

İlerlemeden önce iki üstteki dönüş matrisi, $T$ diyelim, hakkında ilginç bir
ispatı verelim, ileride lazım olacak. Acaba $T^T = T^{-1}$ ifadesi doğru mudur?
Bu aynı zamanda [1] kitabındaki 3.28 probleminin de cevabı. İspat için $T T^T$
çarpımını yapabiliriz, eğer birim (identity) matrisi elde edersek ispat tamam
demektir.

$$
T = 
\left[\begin{array}{cccc}
C & S & 0 & 0 \\
-S & C & 0 & 0 \\
0 & 0 & C & S \\
0 & 0 & -S & C 
\end{array}\right], \quad
T^T = 
\left[\begin{array}{cccc}
C & -S & 0 & 0 \\
S & C & 0 & 0 \\
0 & 0 & C & -S \\
0 & 0 & S & C 
\end{array}\right]
$$

Çarpımı \verb!sympy! ile yapalım,

\begin{minted}[fontsize=\footnotesize]{python}
from sympy import symbols, pprint, latex
from sympy.matrices import Matrix
C,S = symbols("C,S")
T = Matrix([[C,S,0,0],[-S,C,0,0],[0,0,C,S],[0,0,-S,C]])
Tprime = Matrix([[C,-S,0,0],[S,C,0,0],[0,0,C,-S],[0,0,S,C]])
print (latex(T * Tprime)[:60],'...')
\end{minted}

\begin{verbatim}
\left[\begin{matrix}C^{2} + S^{2} & 0 & 0 & 0\\0 & C^{2} + S ...
\end{verbatim}

\LaTeX\ ile

$$
\left[\begin{matrix}C^{2} + S^{2} & 0 & 0 & 0\\0 & C^{2} + S^{2} & 0 & 0\\0 & 0 & C^{2} + S^{2} & 0\\0 & 0 & 0 & C^{2} + S^{2}\end{matrix}\right]
$$

Hatırlarsak $C = \cos\theta, S = \sin\theta$, bunları yerine koyunca tüm köşegen
boyunca 1 değeri elde edilir, diğer hücrelerde sıfır var, demek ki bir birim
matrisi elde ettik. Bu demektir ki $T T^T = I$, ve bu ifadenin doğru olmasının
tek yolu $T^T = T^{-1}$ olmasıdır.

Bir önemli eşitlik daha, hem yer değişimleri, hem kuvvetler döndürme
matematiğini kullanabilirler. Mesela dönüş matrisi $T$ için

$$
d' = T d
$$

diyebilirdim, ya da kuvvetler için

$$
f' = T f
$$

Bunun bir yan etkisi şudur, yer değişimlerini kuvvetlerle ilintilendiren
sistem

$$
f = k d
$$

ise,

$$
f' = k' d' 
$$

sistem şöyle de gösterilebilir,

$$
T f = k' T d
$$

Eğer üstteki ifadeyi soldan $T^{-1}$ ile çarparsak,

$$
T^{-1} T f = T^{-1} k' T d
$$

$T^{-1} T = I$ olduğu için yokolur, ayrıca biraz önceki ispattan $T^{-1} = T^T$
olduğunu biliyoruz,

$$
f = T^T k' T d
$$

Global direngenlik matrisi $k$ ortadaki $T^T k' T$ büyüklüğüdür.

Devam edelim. Dönüş mekaniğini gördük, şimdi önceki derste işlenen kiriş
parçasına hem eksenel dinamiği hem de biraz önce gördüğümüz dönüş mantığını
ekleyelim.  Altta görülen kiriş parçasının hareketlerini hesaplayabilmek
istiyoruz yani,

\includegraphics[width=15em]{phy_020_strs_06_02.jpg}

Önceki dersten hatırlarsak eksene dik yük alan parçaların mekaniği alttaki
formülle gösterilmişti,

$$
\left[\begin{array}{c}
f_{1y} \\ m_1 \\ f_{2y} \\ m_2
\end{array}\right] =
\frac{EI}{L^3}
\left[\begin{array}{cccc}
12 & 6L & -12 & 6L \\
6L & 4L^2 & -6L & -6L \\
-12 & -6L & 12 & -6L \\
6L & 2L^2 & -6L & 4L^2
\end{array}\right]
\left[\begin{array}{ccc}
v_1 \\ \phi_1 \\ v_2 \\ \phi_2
\end{array}\right]
$$

Bu formüle (1)'deki eksenel mantığı eklersek, yerel kordinatlarda

$$
\left[\begin{array}{c}
f'_{1x} \\ f'_{1y} \\ m'_1 \\ f'_{2x} \\ f'_{2y} \\ m'_2
\end{array}\right] =
\left[\begin{array}{cccccc}
C_1 & 0 & 0 & -C_1 & 0 & 0 \\
0 & 12C_2 & 6 C_2 L & 0 & -12 C_2 & 6 C_2 L \\
0 & 6C_2 L & 4 C_2 L^2 & 0 & -6 C_2 L & 2 C_2 L^2 \\
-C_1 & 0 & 0 & C_1 & 0 & 0 \\
0 & -12C_2 & -6 C_2 L & 0 & 12 C_2 & -6 C_2 L \\
0 & 6 C_2 L & 2 C_2 L^2 & 0 & -6C_2 L & 4C_2 L^2
\end{array}\right]
\left[\begin{array}{c}
u'_1 \\ v'_1 \\ \phi'_1 \\ u'_2 \\ v'_2 \\ \phi'_2
\end{array}\right]
$$

elde edilir, ki $C_1 = \dfrac{AE}{L}$ ve $C_2 = \dfrac{EI}{L^3}$
olmak üzere. Üstte ortada duran matris $k'$ matrisidir.

Şimdi dönüş mekaniğini ekleyelim.

$$
\left[\begin{array}{ccc}
u'_1 \\ v'_1 \\ \phi'_1 \\ u'_2 \\ v'_2 \\ \phi'_2 
\end{array}\right] =
\left[\begin{array}{cccccc}
C & S & 0 & 0 & 0 & 0 \\
-S & C & 0 & 0 & 0 & 0 \\
0 & 0 & 1 & 0 & 0 & 0 \\
0 & 0 & 0 & C & S & 0 \\
0 & 0 & 0 & -S & C & 0 \\
0 & 0 & 0 & 0 & 0 & 1
\end{array}\right]
\left[\begin{array}{ccc}
u_1 \\ v_1 \\ \phi_1 \\ u_2 \\ v_2 \\ \phi_2 
\end{array}\right]
$$

Dikkat edersek dönüşümü sağlayan 2 x 2 boyutundaki altmatris, üstteki matristeki
o iki bölge, daha büyük matriste öyle yerleştirildi ki sadece $u_1,v_1$ ve
$u_2,v_2$ değişkenlerini etkiliyor, onlara tekabül eden bölgelerde duruyor.

Böylece $T$ matrisini bulmuş olduk. Şimdi $k$ matrisini hesaplamak için
$k = T^T k' T$ işlemini yapabiliriz [1, sf. 243].

\begin{minted}[fontsize=\footnotesize]{python}
from sympy import symbols, latex, simplify
from sympy.matrices import Matrix

C,S,C1,C2,L,A,E,I = symbols("C,S,C1,C2,L,A,E,I")
kprime = Matrix([ [C1, 0, 0, -C1, 0, 0],
                  [0, 12*C2, 6*C2*L, 0, -12*C2, 6*C2*L],
                  [0, 6*C2*L, 4*C2*L**2, 0, -6*C2*L, 2*C2*L**2],
                  [-C1, 0, 0, C1, 0, 0],
                  [0, -12*C2, -6*C2*L, 0, 12*C2, -6*C2*L],
                  [0, 6*C2*L, 2*C2*L**2, 0, -6*C2*L, 4*C2*L**2]])

T = Matrix([[C,S,0,0,0,0],[-S,C,0,0,0,0],[0,0,1,0,0,0],
            [0,0,0,C,S,0],[0,0,0,-S,C,0],[0,0,0,0,0,1]])

res = T.transpose()*kprime*T
res = res.subs(C1,A*E/L) 
res = res.subs(C2,E*I/L**3)
res = res / (E/L) # E/L ile onceden bol cunku basitlestirme yapamadi
print (latex(simplify(res))[:70],'...')
\end{minted}

\begin{verbatim}
\left[\begin{matrix}A C^{2} + \frac{12 I S^{2}}{L^{2}} & \frac{C S \le ...
\end{verbatim}

$$
k = 
\frac{E}{L} \times % sonradan biz ekledik cunku bolum yapmistik
\left[\begin{matrix}A C^{2} + \frac{12 I S^{2}}{L^{2}} & \frac{C S \left(A L^{2} - 12 I\right)}{L^{2}} & - \frac{6 I S}{L} & - A C^{2} - \frac{12 I S^{2}}{L^{2}} & \frac{C S \left(- A L^{2} + 12 I\right)}{L^{2}} & - \frac{6 I S}{L}\\\frac{C S \left(A L^{2} - 12 I\right)}{L^{2}} & A S^{2} + \frac{12 C^{2} I}{L^{2}} & \frac{6 C I}{L} & \frac{C S \left(- A L^{2} + 12 I\right)}{L^{2}} & - A S^{2} - \frac{12 C^{2} I}{L^{2}} & \frac{6 C I}{L}\\- \frac{6 I S}{L} & \frac{6 C I}{L} & 4 I & \frac{6 I S}{L} & - \frac{6 C I}{L} & 2 I\\- A C^{2} - \frac{12 I S^{2}}{L^{2}} & \frac{C S \left(- A L^{2} + 12 I\right)}{L^{2}} & \frac{6 I S}{L} & A C^{2} + \frac{12 I S^{2}}{L^{2}} & \frac{C S \left(A L^{2} - 12 I\right)}{L^{2}} & \frac{6 I S}{L}\\\frac{C S \left(- A L^{2} + 12 I\right)}{L^{2}} & - A S^{2} - \frac{12 C^{2} I}{L^{2}} & - \frac{6 C I}{L} & \frac{C S \left(A L^{2} - 12 I\right)}{L^{2}} & A S^{2} + \frac{12 C^{2} I}{L^{2}} & - \frac{6 C I}{L}\\- \frac{6 I S}{L} & \frac{6 C I}{L} & 2 I & \frac{6 I S}{L} & - \frac{6 C I}{L} & 4 I\end{matrix}\right]
$$

Bu sonuç [1]'deki sonuca benziyor, cebirsel olarak eşit. Kod yorumlarında
vurgulandı tekrar bahsedelim, $E/L$ bölümünü \verb!sympy! basitleştirmesi öncesi
sistemde biz uyguladık çünkü cebirsel düzenlemede sisteme yardım etmek istedik,
bu sayede sonuç kitaptaki çıktıya benzemiş oldu.

Üstteki formül / matris düz katı şasi / düz oynamaz çerçeve (rigid plane frame)
formülü olarak bilinir, bu sistem ``katı bir şekilde birbirine bağlanmış bir
grup kiriş parçalarının toplamı'' olarak ta tarif edilebilir, yani kiriş
parçalarının birbirine olan açıları, yük uygulandıktan sonra bağlandıklarında ne
ise o halde kalırlar, deformasyon sonrası değişime uğramazlar. Ayrıca bu tür bir
sistemde moment bir parçadan diğerine, bağlantı noktaları üzerinden transfer
olabilir, yani katı bağlantı noktaları üzerinden bir moment sürekliliği vardır.

Soru

İlk katı düzlem çerçeve analizi olarak alttaki basit sistemi çözün.

\includegraphics[width=15em]{phy_020_strs_06_03.jpg}

Cevap

Sistem düğüm 1 ve 4 üzerinden sabitlenmiş, düğüm 2 üzerinde ve yatay 40 kN
kuvvet uygulanıyor, ayrıca düğüm 3'te pozitif moment 500 N-m var. Üstteki
resimde global kordinat sisteminin yeri gösteriliyor.











[devam edecek]

Kaynaklar

[1] Logan, {\em A First Course in the Finite Element Method, 6th Ed}

\end{document}



