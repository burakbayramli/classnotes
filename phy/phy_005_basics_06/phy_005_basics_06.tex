\documentclass[12pt,fleqn]{article}\usepackage{../../common}
\begin{document}
Hareketin Katı-Gövde Denklemleri - 3

[1, sf. 842]

\begin{minted}[fontsize=\footnotesize]{python}
n = 256
theta0 = 0.1
dtheta0 = 1.0
h = 0.1
c = 1.0

theta_values = np.zeros(n)

for i in range(n):
    K1theta = h * dtheta0
    K1dtheta = -h * c * np.sin(theta0)
    theta1 = theta0 + 0.5 * K1theta
    dtheta1 = dtheta0 + 0.5 * K1dtheta
    K2theta = h * dtheta1
    K2dtheta = -h * c * np.sin(theta1)

    theta1 = theta0 + 0.5 * K2theta
    dtheta1 = dtheta0 + 0.5 * K2dtheta
    K3theta = h * dtheta1
    K3dtheta = -h * c * np.sin(theta1)
    theta1 = theta0 + K3theta
    dtheta1 = dtheta0 + K3dtheta
    K4theta = h * dtheta1
    K4dtheta = -h * c * np.sin(theta1)
    theta1 = theta0 + (K1theta + 2.0 * K2theta + 2.0 * K3theta + K4theta) / 6.0
    dtheta1 = dtheta0 + (K1dtheta + 2.0 * K2dtheta + 2.0 * K3dtheta + K4dtheta) / 6.0
    theta_values[i] = theta1
    theta0 = theta1
    dtheta0 = dtheta1

plt.ylim(-1.20,1.20)
plt.plot(range(n), theta_values)
plt.savefig('phy_005_basics_06_01.jpg')
\end{minted}
















Kaynaklar

[1] Eberly, {\em Game Physics 2nd Ed}

\end{document}
