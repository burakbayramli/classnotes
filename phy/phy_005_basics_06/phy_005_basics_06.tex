\documentclass[12pt,fleqn]{article}\usepackage{../../common}
\begin{document}
Hareketin Katı-Gövde Denklemleri - 2

Daha önce rotasyon matrisi bazlı hareket denklemlerini vermiştik. Bu yazıda
cisim duruşu ve dönüş mekaniğini kuaterniyon kavramı üzerinden hesaplayacağız,
daha önce görülen bazı kavramların da tekrar üzerinden geçeceğiz.

Lineer Hız, Momentum

Cisim yeri ve hızı şöyle alakalı,

$$
\frac{\ud x(t)}{\ud t} = v(t)
$$

Katı gövdenin momentumu

$$
p(t) = m v(t)
$$

Newton'un ikinci kanununa gore,

$$
\frac{\ud p(t)}{\ud t} = F(t)
$$

Yani momentumun zamansal türevi kuvvettir. Hareket hesaplarında $F$ entegre
edilerek $p$ hesaplanır, $p$'yi $m$ ile bölerek $v$ elde ederiz, ve onu entegre
ederek cisim yeri $x$ elde edilir [1, sf. 466].

Açısal Momentum

Lineer momentum kavramı anlaşılması kolay, onu bir tür atalet olarak görüyoruz,
hareket eden bir objenin düz bir çizgi üzerindeki hareketini devam ettirmeye
olan meyili. Tabii ki bu meyil, hareketin devamlılığı o anki hız ve cismin
kütlesiyle orantılı, $mv$ buradan geliyor. Açısal momentum da benzer bir kavram,
tek fark bir eksen etrafında dönmekte olan bir cismin dönmeye devam etme meyili.
O zaman üç boyutta, orijin etrafındaki açısal momentum $\vec{L}$, $m$ kütleli
cismin orijine olan uzaklığı $\vec{r}$ ile $m\vec{v}$ çapraz çarpımıdır [1, sf. 42].

$$
L = r \times p = r \times mv
$$

Aynen kuvvetin momentumun zamansal türevi olması gibi, tork benzer şekilde
açısal momentumun türevidir, bunu görmek için üstteki $L$'nin türevini alalım,

$$
\frac{\ud L}{\ud t} = \frac{\ud (r \times p)}{\ud t}
$$

Eşitliğin sağındaki türev zincirleme kuralı ile şöyle açılır,

$$
= r \times \frac{\ud p}{\ud t} + \frac{\ud r}{\ud t} \times p
$$

$\ud p / \ud t = F$, $\ud r / \ud t = v$ olduğunu biliyoruz, üstte yerine
geçirelim,

$$
= r \times F + v \times p
$$

Fakat $v$ ile $p$ aynı yöne işaret eden vektörler, onların çapraz çarpımı sıfır,
geri kalanlar,

$$
\frac{\ud L}{\ud t} = r \times F = \tau
$$

Bir diger esitlik

$$
L(t) = J(t) w(t)
$$

Bu formül açısal momentum ile açısal hız $w$ (diğer kaynaklarda $\omega$) ile
ilişkilendirir, lineer momentum'daki kütle yerine burada $J$ (diğer kaynaklarda
$I$ diye geçer) var, ki $J$ bir atalet tensoru, objenin şekline, ağırlık
dağılımına göre değişir, ve bu değerin kendisi de obje döndükçe değişime uğrar.

Hesaplarda gereken uygulanan tork $\tau(t)$'un sebep olduğu cisim dönüşünü
bulmak, bunun için $\tau$ entegre edilerek $L$ elde edilir, $L$ değeri $J$ ile
bölünerek (daha doğrusu $J$ bir matris olduğu için onun tersi alıp çarpılarak)
$w$ elde edilir. Kuaterniyon durumunda

$$
\frac{\ud q(t)}{\ud t} = \frac{1}{2} \omega(t) q(t)
$$

entegre edilerek objenin dönüş sonrası yeni işaret ettiği yer bulunur (burada
$\omega$ açısal hızı temsil eden kuaterniyon). Üstteki denklemi türetmek
gerekirse,









[devam edecek]

Kaynaklar

[1] Eberly, {\em Game Physics 2nd Ed}

[2] Bayramlı, {\em Diferansiyel Denklemler Ders 2}

\end{document}

 
