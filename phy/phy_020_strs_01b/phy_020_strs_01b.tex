\documentclass[12pt,fleqn]{article}\usepackage{../../common}
\begin{document}
Materyel Mekaniği - 1 - Problemler

Altta kiriş odaklı bazı örnek problemleri çözeceğiz. Bir kirişe yük
uygulandığında dengenin muhafaza edilmesi için kiriş içinde kuvvetler oluşur.
Bu iç kuvvetler kirişin destek yapısına göre farklı şekillerde ortaya çıkabilir
[1].

\includegraphics[width=25em]{phy_020_strs_01b_01.jpg}

Üstteki soldaki resimde mesela iki boyutta pimli destek dönüşe izin verir,
tekerlekli yatay sağ, sol hareketi ve dönüşü serbest bırakır. Sabit destekte hiç
harekete izin yoktur. Hangi harekete izin verilmediğine göre yük uygulanması
ardından üst sağdaki iç kuvvetler ortaya çıkacaktır, bunlar pimli durumda dikey
ve yatay kuvvetler, tekerlekli durumda dikey kuvvet, sabit durumda ise her üç
mümkün tepkilerdir, yani moment, dikey ve yatay.

Problem 1














[devam edecek]

Kaynaklar 

[1] The Efficient Engineer, {\em Understanding Shear Force and Bending Moment Diagrams},
    \url{https://youtu.be/C-FEVzI8oe8}

\end{document}
