\documentclass[12pt,fleqn]{article}\usepackage{../../common}
\begin{document}
Optimal Kontrol ve Say�sal ��z�mler

Daha �nce analitik olarak ��zd���m�z �ift-entegre edici
(double-integrat�r) problemine say�sal olarak yakla�al�m [1]. Bu problemde
amac tek eksen �zerinde $y$ diyelim, bir objeyi bir konumdan di�erine
hareket ettirmekti. Ana fizik form�lleri $F = ma$'dan hareketle,

$$
m \ddot{y} = f(t)
$$

olabilir, h�z $\dot{y}(t)$, ivme $\ddot{y}(t)$, konum $y(t)$. E�er

$$
x_1(t) = y(t), \quad x_2(t) = \dot{y}(t)
$$

dersek ODE sistemini �u �ekilde tan�mlayabiliriz,

$$
\dot{x}_1(t) = x_2(t)
$$

$$
\dot{x}_2(t) = u(t)
$$

ki $u(t) = f(t)/m$ olacak. 












[devam edecek]

Kaynaklar

[1] Naidu, {\em Optimal Control Systems}

[2] Wang, {\em Solving optimal control problems with MATLAB - Indirect methods}, 
    \url{http://citeseerx.ist.psu.edu/viewdoc/summary?doi=10.1.1.604.8922}


\end{document}
