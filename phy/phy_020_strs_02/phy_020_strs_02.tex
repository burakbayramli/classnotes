\documentclass[12pt,fleqn]{article}\usepackage{../../common}
\begin{document}
Materyel Mekaniği - 2

Alternatif anlatım; Giriş dersinden hatırlarsak kuvvet uygulanan kirişlerdeki
deformasyon maddenin özellikleriyle ilişkilendirilebiliyordu, bunun için birim
alandaki kuvvet ve birim boya tekabül eden uzama gözönüne alınıyordu. Çok boyuta
geçerken ilk olarak birim alan bazlı içsel kuvvetlerin alttaki gibi genel bir
nesneye nasıl uygulanacağını görelim [3, sf. 184].

\includegraphics[width=15em]{phy_020_strs_02_16.jpg}

Bir $O$ noktasına etki eden iç kuvvetleri incelemek için o noktadan geçen bir
düzlem hayal edebiliriz, düzlemi temsil eden ona dik normal vektör $n$ olsun.
Eğer düzlemi ufak parçalara bölsek ve her bölgeye etki eden kuvvetleri ölçsek
oradaki etki eden kuvvetlerin birinden diğerine değişebileceğini görürdük.

\includegraphics[width=15em]{phy_020_strs_02_17.jpg}

Eğer benzer şekilde $O$ merkezli bir ufak kare $\Delta A$ ele alsak orada etki
eden bir $\Delta F$ olacaktır. $\Delta F$'nin düzleme dik olması gerekmez,
herhangi bir açıda duran bir vektör olabilir, cismin altındaki kuvvetler üstten
alta doğru bastıran kuvvetlerden daha büyükse $\Delta F$ onların tek bir
noktadaki bir tür birleşimi olduğu için yukarı doğru gösteriyor olurdu muhakkak.

Şimdi stres vektörü kavramını tanıştıralım; eğer $\Delta A$ limite doğru giderse

$$
T_n = \lim_{\Delta A \to 0} \frac{\Delta F}{\Delta A}
$$

büyüklüğü stres vektörünü tanımlar.

Dikkat edersek $\Delta F$ büyüklüğü düzlemin duruşuna, yani $n$'ye bağlı olduğu
için özellikle $n$ ibaresini $T$ sembolüne ekledik; her değişik $n$ değişik bir
$T_n$ değerini verebileceği için.

Herhangi bir düzlem kullanabiliriz demiştik, fakat tekrarlanabilirlik, net ifade
açısından her eksene dik birer düzlem, toplam üç tane kullanmak daha iyi
olacak. Örnek $x$ eksenine dik olan bir düzlem altta,

\includegraphics[width=20em]{phy_020_strs_02_18.jpg}

Daha önce gördüğümüz $\Delta F$'in üstteki resimde düzleme göre bileşenlerine
ayıracağız, bunlar $\Delta F_x$, $\Delta F_y$, $\Delta F_z$. Baktığımız alan ise
kenarları $\Delta y$ ve $\Delta z$ olan bir dikdörtgen, alan $\Delta A_x$ ise
(notasyon olarak dik olduğumüz eksenin sembolünü verdik)
$\Delta A_x = \Delta y \Delta z$.

Daha önce olduğu gibi burada da limit tekniğini kullanabiliriz, $\Delta A_x \to
0$ olacak. Fakat yine notasyonel olarak referans eksen yönündeki strese $\sigma$
sembolü üzerinden normal stres , o eksene dik yani düzleme paralel olan
bileşenlere $\tau$ üzerinden kaykılma (shear) stresi adlarını vereceğiz.
Limitlerle beraber,

$$
\sigma_x = \lim_{\Delta A_x \to 0 } \frac{\Delta F_x}{\Delta A_x}
$$

$$
\tau_{xy} = \lim_{\Delta A_x \to 0 } \frac{\Delta F_y}{\Delta A_x}
$$

$$
\tau_{xz} = \lim_{\Delta A_x \to 0 } \frac{\Delta F_z}{\Delta A_x}
$$

Aynı düzlemle kesme tekniğini iki diğer eksen $y,z$ için de kullanırsak, ve
benzer hesapları yaparsak oradan da altı tane stres değeri elde ederiz, toplam
dokuz tane, hepsi bir arada bir matris içinde,

$$
\left[\begin{array}{rrr}
\sigma_x & \tau_{xy} & \tau_{xz} \\
\tau_{yx} & \sigma_y & \tau_{yz} \\
\tau_{zx} & \tau_{zy} & \sigma_z
\end{array}\right]
$$

[atlandı]

Euler-Bernoulli Kirişleri (Beams)

\includegraphics[width=20em]{phy_020_strs_02_09.jpg}

Mühendislikte, özellikle inşaat mühendisliğinde kirişler yaygın bilinen
bir konudur. Bir kiriş bağlamında

\begin{itemize}
   \item Kaykılma Kuvvet Fonksiyonu $V(X_1)$
   \item Bükülme Moment Fonksiyonu $M(X_1)$
   \item Saptırma (Deflection) Fonksiyonu $y(X_1)$
\end{itemize}

gibi kalemlerle ilgileniyor olabilirim. Bu kalemlerden ilk ikisi çok basittir.
Üçüncü hesap kirişin ``servis edebilir'' olup olmadığını söyleyebilir mesela, ki
insanlar bu kirişin üstünde yürüdüğünde oraya buraya savrulmasınlar (saptırma bu
hesabı bize verebilir). İşte üstteki türden hesapları yapabilmek için
Euler-Bernoulli kiriş faraziyesinden yola çıkmak yaygın bir yaklaşımdır.  Bu
önkabuller nelerdir?

\begin{itemize}
   \item Deformasyonlar ufak: Bu doğal bir varsayım, inşaat mühendisliğinde
     mesela eğer ortada bir deprem yok ise çok büyük şekil değişiklikleri
     beklemeye gerek yok. 
   \item Kiriş lineer elastik eşyönlü (isotropic) maddeden yapılmış: inşaat
     mühendisliğindeki çelik kirişlerin zaten böyle olduğu farz edilir.
   \item Poisson oranı etkileri yoksayılır.
   \item Düzlem bölümler düzlem kalır (plane sections remain plain). Boyu
     eninden çok daha fazla olan nesnelerde bu doğrudur, fakat daha ufak
     parçalarda varsayım tutmayabilir.
\end{itemize}

İç Kuvvetler - Kaykılma ve Bükülme Momenti

\includegraphics[width=30em]{phy_020_strs_02_10.jpg}

Kirişin ufak bir bölümüne odaklanıyoruz ve oradaki kuvvetleri listelemeye
uğraşıyoruz. Üstte birinci resimdeki parçaya bakıyoruz, eksenler yatay $X_1$
dikey $X_2$, ve parça yatay $\ud X_1$ genişliğinde, üzerinde dağıtık yük $q$
var. Unutmayalım bir parçayı kesip çıkartınca onun üzerindeki kuvvetler hayali
olarak ortaya ``saçılır'', bu saçılma çekiş vektörleri kadar çetrefil değil
muhakkak (çünkü önkabullerle pek çok şeyi burada basitleştirdik) ama üç tane
temel kuvvet olduğunu biliyoruz.

Kuvvetlerden ilki üstteki resimde soldan birinci, normal kuvvet $N$. Bu kuvvet
parçanın sol tarafında $N$, sağ tarafında $N$'deki ($X_1$'e göre) değişim çarpı
$X_1$'deki değişim. Aynı parçayla ve resimle devam ediyoruz, soldan ikinci
resimde kaykılma kuvveti $V$ var, aynı değişim matematiği orada da var, ve
nihayet soldan üçüncü resimde bükülme momenti, benzer matematik.

Kuvvet denge denklemi yazarsak dikey yöndeki toplam kuvvetlerin sıfır olması
gerekir, yatay aynı şekilde,

$$
\sum F_{X_2} = V - \left( V + \frac{\ud V}{\ud X_1} \ud X_1 \right) + q \ud X_1 = 0
$$

Not: Dağıtık yük $q$ birim uzunluğa düşen kuvvettir, $N/m$, bu sebeple $q$ bir
uzunluk olan $\ud X_1$ ile çarpılınca kuvvet elde edilir.

$V$'ler iptal olur, geriye kalanları tekrar düzenlersek

$$
\frac{\ud V}{\ud X_1} \ud X_1 = q \ud X_1 
$$

Eşitliğin her iki tarafında $\ud X_1$ var, onları da iptal edersek,

$$
q = \frac{\ud V}{\ud X_1}
\mlabel{3}
$$

Böylece kirişin üzerindeki dağıtık yük ile aynı kiriş üzerindeki kaykılma
kuvveti arasında bir ilişkiyi ortaya çıkarmış oldum. 

Momentler için de benzer bir denge formülü ortaya çıkartabilirim. Moment hesabı
için bir nokta seçilmeli, ufak parçanın sağ noktasını baz alıyorum (resimde
işaretli),

\includegraphics[width=15em]{phy_020_strs_02_11.jpg}

Referans nokta gerekli çünkü hatırlarsak moment bir nokta etrafındaki
döndürmeye bağlıdır, kuvvet dönüş çapına teğet olan kuvvettir. O zaman 

$$
\sum M_{X_3} = -M + \left( M + \frac{\ud M}{\ud X_1} \ud X_1 \right) -
V \ud X_1 - q \ud X_1 \left( \frac{\ud X_1}{2}  \right) = 0
$$

Formüldeki $\ud X_1 / 2$ nereden çıktı? Moment için bir kuvvet uygulama uzaklığı
lazım, uzaklık için de tek bir noktayı seçmek gerekli; bu sebeple $q$'nun etki
ettiği bölgedeki kuvveti tek bir noktaya yapılıyormuş gibi farzediyoruz, o
bölgenin tam ortasına, yani $- \ud X_1 / 2$ noktasına.  Kuvvet büyüklüğü için o
tüm alana etki eden kuvveti bulmak lazım, $q \ud X_1$.

Devam edelim, $M$ terimleri iptal olur, kalanları tekrar düzenleriz,

$$
V \ud X_1 + \frac{q}{2} \ud X_1^2 = \frac{\ud M}{\ud X_1} \ud X_1
$$

Eşitliğin her yerini $\ud X_1$'e bölelim,

$$
V + \frac{q}{2} \ud X_1 = \frac{\ud M}{\ud X_1} 
$$

$\ud X_1 \to 0$ iken limiti alırsak, eşitliğin solundaki ikinci terim yokolur,

$$
V = \frac{\ud M}{\ud X_1}
\mlabel{4}
$$

Böylece bir eşitlik daha elde ettim, kaykılma kuvveti $X_1$'e göre momentteki
değişim oranına eşit. Bir önceki eşitlik yük ve kaykılma, bu eşitlik kaykılma ve
moment arasında idi. Bu ilişkiler Statik (Statics) dersinden geliyor, onları
bulmak kolaydı.

Yer değişim (displacement) fonksiyonlarını bulmak biraz daha zor olacak. 

Bir kiriş hayal edelim, iki boyutlu, iki eksen üzerinde, $X_1,X_2$.  Kirişin
tamamı bir eksen üzerinde olacak, bu örnekte $X_1$.

\includegraphics[width=20em]{phy_020_strs_02_12.jpg}

Sonra kirişe bir yük bindirilecek, bükülme olacak, sapacak (deflect).  Bizim
bulmak istediğimiz yer değişim fonksiyonu, $y$ diyelim, bükülme öncesi eksen ile
bükülme sonrası kirişin eski ekseninin yeni geldiği yer (kesikli çizgi, nötr
eksen ismi de verilir) arasındaki mesafe olacak.

\includegraphics[width=20em]{phy_020_strs_02_13.jpg}

Bu amaçla üstte listelediğimiz dördüncü ``düzlem bölümler düzlem kalır''
önkabulünü kullanacağız. Kirişteki bir noktayı (kırmızı) takip ediyoruz, ve
bükülme sonrası nereye geldiğine bakıyoruz, önkabul bize kirişin sağ sınırının
(mavi) bükülme öncesi sonrası düz kaldığını ve nötr eksene dik kaldığını
söylüyor, ki ufak bükülmeler için bu abartılı bir önkabul değil.

Şimdi ilk kırmızı noktadan yukarı doğru bir kesikli çizgi çekiyoruz ve bu çizgi,
yeni mavi çizgi arasında bir üçgen oluşturuyoruz. Bu üçgenin kenarları $X_2$
bazlı olarak alttaki gibi betimlenebilir.

\includegraphics[width=10em]{phy_020_strs_02_14.jpg}

Şimdi eğer kiriş üzerindeki $X_1$ ekseninde herhangi bir noktanın bükülme
sonrası geldiği yeri üç boyutta göstermek istesek, üç boyutlu pozisyon
fonksiyonu $x$ şöyle olabilirdi,

$$
x = \left[\begin{array}{c}
X_1 - X_2 \sin \theta \\ y + X_2 \cos\theta \\ X_3
\end{array}\right]
$$

Kirişin yeni geldiği yerin kordinatı bunlar, formül içine $y$ koyduk, o bazlı
formülize ettik, ki sonra o $y$ sembolünü bulmak tüm denklemleri çözeceğiz.
$X_3$ olduğu gibi kaldı çünkü o derinlik ölçüsü, $z$ kordinatı, model iki
boyutlu olduğu için değişmiyor.

Üstteki denklemleri elde ettikten sonra cebirsel taklalar ile gerilme
tensörüne ulaşmak mümkün.

İlk takla ufak deformasyon önkabulünü kullanmak, ufak deformasyon ufak $\theta$
demektir, eğer $\theta$ ufak ise, bu

$$
\cos\theta \approx 1 \quad \sin\theta \approx \theta
$$

demektir [4]. O zaman üstteki formül

$$
x = \left[\begin{array}{c}
X_1 - X_2 \theta \\ y + X_2 \\ X_3
\end{array}\right]
$$

haline gelir. Tüm bunları hesapladık ki yer değişim fonksiyonu $u$'ya
gelebilelim, hatırlarsak $u = x - X$ idi, üstteki $x$'i kullanırsak,

$$
u = x - X =
\left[\begin{array}{c}
X_1 - X_2 \theta \\ y + X_2 \\ X_3
\end{array}\right] -
\left[\begin{array}{ccc}
X_1 \\ X_2 \\ X_3
\end{array}\right] =
\left[\begin{array}{ccc}
-X_2 \theta \\ y \\ 0
\end{array}\right] 
$$

Oldukça basitleşti. Bir takla daha atabiliriz, mesela $\theta$ $y$'nin eğimidir,
ve eğim hesapları yaklaşık olarak bir türevdir, yani $\theta = \frac{\ud y}{\ud X_1}$,
yeni formül,

$$
u = \left[\begin{array}{ccc}
-X_2 \frac{\ud y}{\ud X_1} \\ y \\ 0
\end{array}\right]
\mlabel{1}
$$

haline geldi. Artık $u$'yu bildiğimize göre yer değişim gradyanı $\nabla u$'yu
hesaplayabiliriz. Önceki dersten hatırlarsak bu gradyan

$$
\renewcommand*{\arraystretch}{2.5}
\nabla u = \frac{\partial u_i}{\partial X_j} =
\left[\begin{array}{ccc}
\dfrac{\partial u_1}{\partial X_1} & \dfrac{\partial u_1}{\partial X_2} & \dfrac{\partial u_1}{\partial X_3} \\
\dfrac{\partial u_2}{\partial X_1} & \dfrac{\partial u_2}{\partial X_2} & \dfrac{\partial u_2}{\partial X_3} \\
\dfrac{\partial u_3}{\partial X_1} & \dfrac{\partial u_3}{\partial X_2} & \dfrac{\partial u_3}{\partial X_3} 
\end{array}\right]
$$

idi. Eğer cebiri takip edersek kısmi türevleri alarak sonuca varabiliriz.

Yanlız dikkat, bu kısmi türevlerden bazıları bariz olabilir, fakat, mesela
$y$'nin $X_1,X_2,..$ $y$ formülünün hangi değişken temelli olduğunu bulmak
gerekiyor. Biraz düşününce ve üstteki $y$'nin olduğu şekle danışarak anlıyoruz
ki $y$ yatay gittiğimizde değişen bir şeydir, o zaman $y$ $X_1$'in fonksiyonu
olmalıdır, $X_1$'e göre bir türev vardır, diğer değişkenler $X_2,X_3$'e göre $y$
türevi sıfır olacaktır. O zaman,

$$
\renewcommand*{\arraystretch}{2.5}
\nabla u = \left[\begin{array}{ccc}
-X_2 \left( \dfrac{\ud^2 y}{\ud X_1^2} \right)  & - \dfrac{\ud y}{\ud X_1} & 0 \\ 
\dfrac{\ud y}{\ud X_1} & 0 & 0 \\ 
0 & 0 & 0
\end{array}\right]
$$

Bu sonuca nasıl geldik, en üst sol köşede istenen $\frac{\partial u_1}{\partial X_1}$
idi, bu demektir ki (1) matrisinin en üst ögesinin $X_1$'e göre türevi, bu
türevi alınca gereken sonuç elde edilmiş oluyor, diğerleri benzer şekilde.
Yer değişim gradyanını böylece bulmuş oluyoruz.

Yer değişim gradyanı niye lazımdı? Hatırlarsak gerinim tensörünü $\epsilon$
formülü

$$
\epsilon = \frac{1}{2} (\nabla u + \nabla u^T )
$$

formülde biraz önce bulduğumuz gradyan var. Formülü hesaplayınca 

$$
\renewcommand*{\arraystretch}{1.5}
\epsilon = \left[\begin{array}{ccc}
-X_2 \dfrac{\ud^2 y}{\ud X_1} & 0 & 0 \\
0 & 0 & 0 \\
0 & 0 & 0 
\end{array}\right]
$$

buluyoruz. Demek ki Euler-Bernoulli kirişlerinin sıfır olmayan tek gerilim öğesi
$\epsilon_{11}$, matrisin sol üst köşesindeki öğe. Ayrıca $X_1$ yukarı yönde, ya
da eğer kirişi dikey kessek o kesit üzerinde (cross-section) sabit olacağı için
$\frac{\ud^2 y}{\ud X_1}$ sabittir, o zaman $\epsilon_{11}$ dikey yönde lineer
olur. Çelik ya da beton tasarım dersi alanlar bilir bu lineer profil çoğu
formüllerimizin doğduğu yerdir.

Peki gerilim hesabını niye yaptık? Çünkü şimdi Belirleyici Kanunları
hatırlarsak, eğer gerilimi biliyorsak stresi de hesaplayabiliyoruz çünkü ikisi
arasında ilişki var. Eğer Poisson etkilerini yoksayarsak, tek eksenel (uniaxial)
durum var elimizde, yani Hooke Kanunu, ki bu da

$$
\sigma_{11} = E \epsilon_{11}
$$

Basit bir Young'in Genliği ile çarpım işlemi bu, demek ki

$$
\sigma_{11} = E \left( -X_2 \left( \dfrac{\ud^2 y}{\ud X_1} \right) \right)
\mlabel{2}
$$

Deformasyon ve yük bağlantısına erişmek için [5,6], bükülme moment fonksiyonu
$M(X_1)$ için bir formül daha oluşturabiliriz, alttaki şekle bakalım, orada
$M_3$ denen bizim $M$ olarak aradığımız, yani $X_3$ etrafında bir dönüşü
hesaplıyoruz, bunu her $X_1$ noktası, kesiti için yapıyoruz, $M_3(X_1) = M(X_1)
= M$.

\includegraphics[width=25em]{phy_020_strs_02_15.jpg}

Kirişe sağdan etki eden stres bileşeni $\sigma_{11}$, $X_1$ yönünde, bükülme
sırasında bu tür stres kirişe dönüş merkezine farklı $X_2$ uzaklıklarında etki
ediyor olacak, eksenin üstteki kısımda sıkışma, daha altında gerilme olacak. Bir
dönüş ekseni etrafında moment hesabı yapıyoruz sonuçta, moment uzaklık çarpı
kuvvettir, uzaklık $X_2$ boyunca

$$
M = \int \int -X_2 \sigma_{11} \ud X_2 \ud X_3
$$

Uzaklık $-X_2$ (eksi işareti kullanıldı ki isimleme geleneğine uymak için üste
doğru gidince sıkışma) çarpı kuvvet $\sigma_{11}$. Yukarı-aşağı $X_2$ ufak
parçalarla çarpıp entegre ediyoruz ve bu sonucu $X_3$ yönündeki ufak parçalar
üzerinden bir daha entegre ediyoruz, böylece üstteki formüle erişiyoruz.

(2)'deki sonucu üste $\sigma_{11}$'e sokarsak,

$$
M = \int \int E X_2^2 \frac{\ud^2 y}{\ud X_1^2} \ud X_2 \ud X_3
$$

Üstteki türden bir şekle sahip kirişlerin dönme direnci hesabının

$$
I = \int \int X_2^2 \ud X_2 \ud X_3
$$

olduğunu biliyoruz. Bu hesabı basitleştirme amacıyla iki üstteki formülde
kullanabiliriz,

$$
M = E I \frac{\ud^2 y}{\ud X_1^2} 
$$

Şimdi daha önce türetilen (3), (4) formüllerini hatırlayalım,

$$
\frac{\ud V}{\ud X_1} = q \quad \frac{\ud M}{\ud X_1} = V
$$

Üstteki üç formülü birleştirmek mümkün, üç üstteki $M$'nin türevini
alınca $V$'ye eşit olan forma geliyoruz, bir daha türev alınca ki $\frac{\ud V}{\ud X_1}$
eşitliğinden faydalanabilelim, eşitliğin bir tarafında $q$'ya, diğer tarafında
ise dördüncü derece bir diferansiyel denkleme erişiyoruz. Yani yük $q$ icin
Euler-Bernoulli kiriş yer değişim denklemi,

$$
E I \frac{\ud^4 y}{\ud X_1^4} = q
$$

Kaynaklar

[1] Kelly, Solid Mechanics Part III, Auckland University

[2] Petitt, {\em Intro to the Finite Element Method}, University of Alberta,
    \url{https://www.youtube.com/watch?v=2iUnfPRk6Ro&list=PLLSzlda_AXa3yQEJAb5JcmsVDy9i9K_fi}

[3] Crandall, {\em An Introduction to the Mechanics of Solids}

[4] Bayramlı, {\em Diferansiyel Denklemler, Ekler, Küçük Açı Yaklaşıklaması}

[5] Adeeb, {\em CivE 398, Solid Mechanics Lecture}, University of Alberta,
    \url{https://www.youtube.com/watch?v=RKBl3caP16U&list=PLWlJvChadvVz0kK6qf_W6YI3qWBydZtPL}

[6] Adeeb, {\em Introduction to Solid Mechanics, Online Book},
    \url{https://engcourses-uofa.ca/books/introduction-to-solid-mechanics/}

\end{document}





