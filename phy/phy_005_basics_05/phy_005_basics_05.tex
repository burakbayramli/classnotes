\documentclass[12pt,fleqn]{article}\usepackage{../../common}
\begin{document}
Hareketin Katı-Gövde Denklemleri - 1

Rotasyon Matrisi ve Türevi

Bir 3 x 3 dönüş matrisi ile herhangi bir vektörü döndürebileceğimizi biliyoruz.
Yersel taşıma daha da basit, 3 boyutlu bir vektör sadece, mevcut konuma
ekleyerek yeni konumu elde ediyoruz.

Bir katı gövdeyi parçacıkları üzerinden alırsak, ve bu gövdenin açısal dönüşsel
olarak hangi yöne baktığını bir dönüş matrisi $R$ ile temsil edersek, her
parçacık üzerinde bu işlemin uygulandığını düşünebiliriz. Ayrıca konumsal
taşınma ve bakılan yön başlangıçtaki bir ``gövde uzayı''na (body space) göre
yapılabilir, gövdenin kütle merkezini dünya kordinatlarının (0,0,0) orijin
noktasında ve yönü herhangi bir (başta belli) yöne doğru alalım, hareketler hep
bu konuma referansla, onu değiştirecek şekilde düşünülebilir.  Mesela gövde
üzerindeki, gövde uzayındaki, herhangi bir $p_0$ noktasını düşünelim, $t$ anında
bu noktanın dünya uzayındaki konumu

$$
p(t) = R(t) p_0 + x_{CM}(t)
$$

ki $x_{CM}(t)$ bir yersel taşınma, ve $R(t)$ açısal dönüş. Tabii taşınma her
zaman kütle merkezine uygulandığı için $x_{CM}$ aynı zamanda kütle merkezinin
her $t$ anında dünya uzayında olduğu yeri de gösteriyor.

\includegraphics[width=25em]{phy_005_basics_04_04.png}

Türeve gelirsek, bir vektör $r$'nin orijin etrafında döndüğünü
düşünelim. Herhangi bir anda bu dönüşün açısal hızı $\omega$ çapraz çarpımla
hesaplanabilir,

\includegraphics[width=10em]{phy_005_basics_04_03.png}

Hız tabii ki sonsuz küçük zamandaki yer değişimi olduğu için onu

$$
\frac{\ud r}{\ud t} = \omega \times r
$$

olarak ta görebiliriz. Şimdi bir katı gövdeyi düşünelim, onun baktığı yön
(orientation) bir matris $R$ içinde. Bu matrisin her kolonunda bir eksen var,
ilk kolon $x$, ikinci $y$, vs. Eğer gövdenin baktığı yönü $R$ ile temsil
ediyorsak tüm bu kolonlar gövde dönerken değişecektir. Eğer dönüş $\omega$ ise
her eksenin açısal hızı $\omega$ demek, o zaman bu eksenlerin, $b,c,d$ diyelim,
açısal hızı ayrı ayrı $\omega \times b$, $\omega \times c$, $\omega \times d$
olarak bulunabilir, ki bunların her biri aynı zamanda ayrı birer türevdir. Tüm
matrisin türevi

$$
\frac{\ud R}{\ud t} = \tilde \omega \cdot R
$$

ki $\tilde \omega$ ile $\omega$'yi eksi bakışımlı [4] bir matris hale getirdik,
böylece çapraz çarpımı normal çarpım haline çevirmiş oluyoruz [5, sf. G8],
[3, sf. 88].

Atalet Matrisi ve Dönüşler

Daha önce atalet matrisi $I(t)$'yi görmüştük,

$$
I(t) = \sum \left[\begin{array}{ccc}
m_i (y_i^2 + z_i^2) & -m_i x_i y_i & m_i x_i z_i \\
-m_i y_i x_i & m_i (x_i^2 + z_i^2) & -m_i y_i z_i \\
-m_i z_i x_i & -m_i z_i y_i & m_i (x_i^2 + y_i^2)
\end{array}\right]
$$

Burada $x,y,z$ değerleri gövde uzayında, her nokta $r_i'$ için $x_i,y_i,z_i$
değerleri $r_i - \chi(t)$ içeriğiyle hesaplanıyor. Ayrıca bir obje dönerse, onun
belli noktalarının eksenden olan uzaklıkları değişir ve farklı bir $I$ elde
ederiz... fakat üstteki hesabı obje hareket ederken sürekli yapmak oldukca
külfetlidir. Acaba $I$'nin bir baz kısmını hesaplasak, sonra dönüşe göre onu
güncellesek olmaz mı?

Bunun bir yolu var [5, sf. 14]. $r_i'^T r_i' = x_i^2 + y_i^2 + z_i^2$ olduğundan
hareketle, önceki $I$ denklemini şu şekilde yazabiliriz,

$$
I(t) = \sum
m_i r_i'^T r_i' \left[\begin{array}{ccc}
1 & 0 & 0 \\ 
0 & 1 & 0 \\ 
0 & 0 & 1 
\end{array}\right] -
\left[\begin{array}{ccc}
m_i x_i^2 & -m_i x_i y_i & m_i x_i z_i \\
-m_i y_i x_i & m_i y_i^2 & -m_i y_i z_i \\
-m_i z_i x_i & -m_i z_i y_i & m_i z_i^2
\end{array}\right]
$$

Simdi en sağdaki matrise dikkat edelim, onu bir dış çarpım (outer product)
olarak temsil edebiliriz, alttaki gibi,

$$
r_i' r_i'^T = \left[\begin{array}{c}
x_i \\ y_i \\ z_i
\end{array}\right]
\left[\begin{array}{ccc}
x_i & y_i & z_i
\end{array}\right] =
\left[\begin{array}{ccc}
x_i^2 &  x_i y_i &  x_i z_i \\
y_i x_i & y_i^2 & y_i z_i \\
z_i x_i & z_i y_i & z_i^2
\end{array}\right]
$$

Bunu kullanarak ve 3 x 3 boyutlu birim matrisini $\overline{1}$ ile göstererek
(normalde bu matris için $I$ notasyonu kullanılır ama o harf bu yazıda
kapılmış durumda),

$$
I(t) = \sum m_i ((r_i'^T r_i') \overline{1} - r_i' r_i'^T)
$$

Bu nasıl faydalı? Çünkü $r_i(t) = R(t) r_{0i} + x_{CM}(t)$ ki $r_{0i}$
başlangıçtaki kütlede $i$ parçacığın yeri, ve sabit, o zaman

$$
r_i(t) - x_{CM}(t) = R(t) r_{0i} = r_i'(t)
$$

Şimdi $r_i'(t) = R(t) r_{0i}$ eşitliğini iki üstteki formülde kullanırsak,

$$
I(t) = \sum
m_i ( (R(t) r_{0i})^T  (R(t) r_{0i})  \overline{1} -  (R(t) r_{0i})  (R(t) r_{0i})^T  )
$$

$$
= \sum
m_i ( r_{0i}^T R(t)^T R(t) r_{0i} \overline{1} -  R(t) r_{0i} r_{0i}^T R(t)^T )
$$

$R(t)$ dikgen, ortonormal matris oldugu icin $R(t)^TR(t) = \overline{1}$

$$
= \sum
m_i ( (r_{0i}^T r_{0i}) \overline{1} -  R(t) r_{0i} r_{0i}^T R(t)^T )
$$


Üstteki formülde ikinci terimde $R(t) .. R(t)^T$ ifadesi var, bunu
birinci terime de eklemek için, ve $r_{0i}^T r_{0i}$ bir tek sayı değer
olduğu için ve $R(t) R(t)^T$'nin birim matris olmasından hareketle,

$$
= \sum
m_i ( R(t) (r_{0i}^T r_{0i}) R(t)^T \overline{1} -  R(t) r_{0i} r_{0i}^T R(t)^T )
$$

Böylece $R(t)$ ve $R(t)^T$ dışarı çekilebiliyor,

$$
= R(t) \left( \sum 
m_i (( r_{0i}^T r_{0i}) \overline{1} -  r_{0i} r_{0i}^T
\right)  R(t)^T 
$$

Böylece parantez içindeki, $I_{body}$ denebilecek değerler parçacıkların
gövdenin ilk konumundaki yerlerine (ve değişmeyen kütle $m_i$ değerine) göre
hesaplanabileceği için, onu bir kez hesaplayabiliriz, ve sonra ona $R(t)$'leri
uygulayarak istediğimiz güncel $I(t)$ değerini elde ederiz [5, sf. 15].

$$
I(t) = R(t) I_{body} R(t)^T
$$

$I$'nin tersi $I_{body}^{-1}$ de gerekli (niye birazdan göreceğiz) fakat bu
hesap ta başta hesaplanıp depolanabilir, çünkü

$$
I^{-1} = ( R(t) I_{body} R(t)^T )^{-1} 
$$

$$
(R(t)^T)^{-1} I_{body}^{-1} R(t)^{-1} 
$$

$$
= R(t) I_{body}^{-1} R(t)^T
$$

$R(t)^T = R(t)^{-1}$ ve $R(t)^T = R(t)$ olduğunu hatırlayalım çünkü $R(t)$
orthonormal, dikgen bir matris. 

Hareketin Katı Gövde Denklemleri

Artık elimizde bir gövdenin her bakımdan konumunu, statüsünü temsil etmek için yeterli
matematik var. Bu konumu $\overline{X}(t)$ ile gösterebiliriz,

$$
\overline{X} = \left[\begin{array}{c}
x_{CM}(t) \\ R(t) \\ P(t) \\ L(t)
\end{array}\right]
$$

Momentum $P(t) = v(t) M$ olduğu için $v(t) = \frac{P(t)}{M}$.

$I(t)$'yi yukarıda gördük, $I(t) = R(t) I_{body} R(t)^T$.

$L(t) = I(t) \omega(t)$ olduğu için $\omega(t) = I(t)^{-1} L(t)$

Hepsini biraraya koyunca $\overline{X}$'nin türevi

$$
\frac{\ud}{\ud t} \overline{X}(t) =
\frac{\ud}{\ud t}
\left[\begin{array}{c}
x_{CM}(t) \\ R(t) \\ P(t) \\ L(t)
\end{array}\right]
=
\left[\begin{array}{c}
v(t) \\ \tilde \omega \cdot R(t) \\ F(t) \\ \tau(t)
\end{array}\right]
$$

%%%%%%%%%%%%%%%%%%%%%%%%%%%%%%%%%%%%%%%%%%%%%%%%%%%%%%%%%%%%%%%%%%%%%%%%%%

Katı-Gövde Simülasyonu

Dönüş Animasyonu

Bir örnek gövde üzerinde simülasyon yapmaya uğraşalım. Elimizde bir simit, ya da
geometride torus denen bir şekil var. Bu dosya STL denen bir format içinde,
detaylar için [6]. Kuvvet uygulama sonrası lineer ve açısal momentum içeren
simülasyon için pek çok değişkeni diferansiyel tanımları üzerinden entegre
etmemiz gerekiyor, daha basit bir örnek ile, özellikle sabit bir açısal hız
üzerinden salt döndürme ile başlamak uygun olabilir. [8]'te tarif edilen
döndürme matrisi türevini hatırlarsak,

$$
\frac{\ud R}{\ud t} = \tilde \omega \cdot R
$$

Döndürmeyi bir $\omega$ etrafında düşünüyorduk, $\omega$'nin büyüklüğü
açısal dönme hızına tekabül ediyordu, ve $\tilde \omega$ eksi-bakışımlı
matris idi.

\includegraphics[width=25em]{compscieng_bpp32sim_rigbod_01.png}

Tüm bunları entegre edici \verb!odeint! çağrısının kabul edeceği bir formda
nasıl kullanırız? Bu çağrı düzleştirilmiş bir liste içinde diferansiyel
sonuçların, ve ana değişkenlerin olmasını bekliyor. O zaman $R$'yi kolon bazlı
olmak üzere düzleştiririz, ve gerektiği o listeden matris formuna geçeriz, vs.

\begin{minted}[fontsize=\footnotesize]{python}
from scipy.integrate import odeint
from stl import mesh

def skew(a):
   return np.array([[0,-a[2],a[1]],[a[2],0,-a[0]],[-a[1],a[0],0]])

your_mesh = mesh.Mesh.from_file('torus.stl')
prop = your_mesh.get_mass_properties()
R0 = np.eye(3,3)
omega = np.array([1.0,1.0,1.0])
#omega = np.array([0.0,1.0,0.0])
skew_omega = skew(omega)
   
def dRdt(u,t):   
   R1x,R1y,R1z,R2x,R2y,R2z,R3x,R3y,R3z = u
   R = np.array([R1x,R1y,R1z,R2x,R2y,R2z,R3x,R3y,R3z])
   R = R.reshape((3,3)).T
   res = np.dot(skew_omega, R)
   return list(res.T.flatten())

LIM = 5
STEPS = 20
t=np.linspace(0.0, 3.0, STEPS)
R0 = np.eye(3,3)
u0 = R0.flatten()
u1=odeint(dRdt,list(u0),t)
\end{minted}

Üstte görülen mesela \verb!R1x! $R$ matrisinin 1'inci kolonunun $x$ değişkeni
anlamında.

Simülasyonda simit şeklinin baktığı yön $R$ içinde, ve grafik amaçlı olarak her
seferinde simit şeklini sıfırdan yükleyip son $R$'ye ilerletiyoruz, ve her
adımda bu grafiği basıyoruz.  Simülasyonu hesapladık, tüm sonuç \verb!u1!
içinde, görüntüden bazı seçilmiş kareler altta görülebilir,

\begin{minted}[fontsize=\footnotesize]{python}
import matplotlib.pyplot as plt
from mpl_toolkits import mplot3d

def plot_vector(fig, orig, v, color='blue'):
   ax = fig.gca(projection='3d')
   orig = np.array(orig); v=np.array(v)
   ax.quiver(orig[0], orig[1], orig[2], v[0], v[1], v[2],color=color)
   ax = fig.gca(projection='3d')  
   return fig

for i in range(STEPS):
    fig = plt.figure()
    axes = mplot3d.Axes3D(fig)
    your_mesh = mesh.Mesh.from_file('torus.stl')
    R = u1[i].reshape((3,3)).T
    your_mesh.rotate_using_matrix(R)
    scale = your_mesh.points.flatten()
    axes.add_collection3d(mplot3d.art3d.Poly3DCollection(your_mesh.vectors,alpha=0.3))
    plot_vector(fig, [0,0,0], omega, color='red')
    axes.auto_scale_xyz(scale, scale, scale)
    axes.set_xlim(-LIM,LIM);axes.set_ylim(-LIM,LIM);axes.set_zlim(-LIM,LIM)
    axes.view_init(azim=20,elev=0)
    plt.savefig('/tmp/rotate_%02d.png' % i)  
\end{minted}

\includegraphics[width=20em]{sim1/rotate_00.png}

\includegraphics[width=20em]{sim1/rotate_08.png}

\includegraphics[width=20em]{sim1/rotate_14.png}

\begin{minted}[fontsize=\footnotesize]{python}
! convert -delay 20 -loop 0 /tmp/rotate*.png /tmp/torus_rotate1.gif
\end{minted}

Animasyon sonucu [1]'de.

Torus şekli hakkında bazı istatistikler alttadır.

\begin{minted}[fontsize=\footnotesize]{python}
from stl import mesh

your_mesh = mesh.Mesh.from_file('torus.stl')

prop = your_mesh.get_mass_properties()
print ('hacim',np.round(prop[0],3))
print ('yercekim merkezi (COG)',np.round(prop[1],3))
print ('COG noktasinda atalet matrisi')
print (np.round(prop[2],3))
\end{minted}

\begin{verbatim}
hacim 4.918
yercekim merkezi (COG) [-0.  0. -0.]
COG noktasinda atalet matrisi
[[ 3.223 -0.     0.   ]
 [-0.     3.223  0.   ]
 [ 0.     0.     5.832]]
\end{verbatim}

COG sıfır noktasında olması, ayrıca atalet matrisinin köşegen olması mantıklı
çünkü simit şekli simetrik.

İttirme Animasyonu

Bu animasyon için katı gövdeye bir noktada bir kuvvet uygulayacağız. O noktayı
seçmek için STL formatında olan üçgenlerden birini kullanabiliriz, çünkü bu
üçgenlerin gövdenin yüzeyinde bir yerlerde olduğunu biliyoruz, Torus STL şekli
bu üçgenlerin herbirine dik olan normal vektörü STL formatında zaten, o
üçgenlerden birinin normal vektörünü ters çevirirsek, o noktaya o yönde bir
kuvvet uyguladığımızı hayal edebiliriz, ve simülasyonun geri kalanını bu
noktadan devam ettiririz.

\begin{minted}[fontsize=\footnotesize]{python}
import matplotlib.pyplot as plt
from mpl_toolkits import mplot3d

fig = plt.figure()
axes = mplot3d.Axes3D(fig)

SCALE = 4
LIM = 5

scale = your_mesh.points.flatten()
axes.add_collection3d(mplot3d.art3d.Poly3DCollection(your_mesh.vectors,alpha=0.3))
axes.auto_scale_xyz(scale, scale, scale)

def plot_vector(fig, orig, v, color='blue'):
   ax = fig.gca(projection='3d')
   orig = np.array(orig); v=np.array(v)
   ax.quiver(orig[0], orig[1], orig[2], v[0], v[1], v[2],color=color)
   ax = fig.gca(projection='3d')  
   return fig

axes.set_xlim(-LIM,LIM);axes.set_ylim(-LIM,LIM);axes.set_zlim(-LIM,LIM)

tidx = 2000
o = np.mean(your_mesh.vectors[tidx],axis=0)
axes.plot (o[0],o[1],o[2],'gd')
n = your_mesh.get_unit_normals()[tidx]
plot_vector(fig, o, n*SCALE)
plot_vector(fig, o, -n*SCALE, color='red')
axes.view_init(azim=84,elev=28)

plt.savefig('compscieng_bpp32sim_rigbod_02.png')
\end{minted}

\includegraphics[width=20em]{compscieng_bpp32sim_rigbod_02.png}

Kuvveti kırmızı vektörle gösterilen yönde uygulayabiliriz.

Şimdi ``sıfırıncı anda'' yani ilk başlangıçta uygulanan kuvvetleri, lineer,
açısal, hesaplamak lazım. Noktayı üstte seçtik, sonu o noktada başlangıcı nesne
ağırlık merkezinde olan bir vektör ile kuvvet vektörü arasında çapraz çarpım
yapıyoruz, bu bize torku veriyor.

\includegraphics[width=25em]{compscieng_bpp32sim_rigbod_03.png}

Benzer şekilde sonu nesne merkezinde başı o noktada olan bir vektör daha var,
lineer kuvvet bu doğrultuda uygulanacak, o vektör üzerine iki üstte görülen
kırmızı vektörü yansıtıyoruz, bu da lineer kuvvet oluyor. Bir üstteki resim
üzerinde gösterirsek,

\includegraphics[width=25em]{compscieng_bpp32sim_rigbod_04.png}

Daha önce söylediğimiz gibi her iki kuvvet de ilk anda lineer ve açısal
momentumu ekileyen faktörler, sonraki adımlarda etkileri yok.

Ayrıca entegrasyon için kendi pişirdiğimiz kodları kullanacağız, \verb!odeint!
işleri zorlaştırabilir çünkü zamana bağlı bazı farklı kodlamalar var, ayrıca
$I^{-1}$ her adımda sürekli değişiyor, yani bir konum güncellemesi var, bu
tür kodlamalar kapalı bir kutu isteyen \verb!odeint! ile daha zor olabilir.
Bunlar problem değil, [9]'te paket kullanmadan hesaplanan bir
kodlama şeklini görmüştük.

\begin{minted}[fontsize=\footnotesize]{python}
import numpy as np
from stl import mesh
import numpy.linalg as lin

your_mesh = mesh.Mesh.from_file('Torus.stl')   
prop = your_mesh.get_mass_properties()
cog = np.round(prop[1],3) # baslangic aninda obje COG
Ibody = np.round(prop[2],3)
Ibodyinv = lin.inv(Ibody)
dt = 0.1
x = np.zeros((1,3))
R = np.eye(3,3)
L = np.zeros((1,3))
v = np.zeros((1,3))
F = np.zeros((3,1))
M = 1
P = M*v

def skew(a):
   return np.array([[0,-a[2],a[1]],[a[2],0,-a[0]],[-a[1],a[0],0]])

tidx = 2000
apply_at = np.mean(your_mesh.vectors[tidx],axis=0) - cog
f_at = 1 * 5 * your_mesh.get_unit_normals()[tidx]
tau0 = np.cross(apply_at, f_at).reshape(1,3) * 10.0
flindir = cog-apply_at
flin0 = np.dot(f_at,flindir)*(flindir/np.abs(lin.norm(flindir)))

res = []
for i in range(30):
   xold,Rold,Pold,Lold = x.copy(),R.copy(),P.copy(),L.copy()
   
   Iinv = np.dot(np.dot(Rold, Ibodyinv), Rold.T)
   omega = np.dot(Iinv, Lold.T).T
   omega = omega.reshape(3)
   skew_omega = skew(omega)
   R = Rold + np.dot(skew_omega, Rold) * dt

   v = Pold / M
   x = x + v*dt
   P = Pold
   if i==0: # baslangic ani
      L = Lold + tau0*dt
      P = Pold + (flin0*dt)
   else:      
      L = Lold # sonraki adimlarda degisim yok
      P = Pold # momentum ayni kaliyor
   res.append([x,R,P,L])
\end{minted}

Hesaplar yapıldı, şimdi grafikleme,

\begin{minted}[fontsize=\footnotesize]{python}
import matplotlib.pyplot as plt
from mpl_toolkits import mplot3d

LIM = 5
SCALE = 4

def plot_vector(fig, orig, v, color='blue'):
   ax = fig.gca(projection='3d')
   orig = np.array(orig); v=np.array(v)
   ax.quiver(orig[0], orig[1], orig[2], v[0], v[1], v[2],color=color)
   ax = fig.gca(projection='3d')  
   return fig

for i, [x,R,P,L] in enumerate(res):
   fig = plt.figure()
   axes = mplot3d.Axes3D(fig)
   your_mesh = mesh.Mesh.from_file('torus.stl')
   # t-0 aninda uygulanan kuvvet yonunu goster
   o = np.mean(your_mesh.vectors[tidx],axis=0)
   n = your_mesh.get_unit_normals()[tidx]
   plot_vector(fig, o, -n*SCALE, color='red')
   
   your_mesh.rotate_using_matrix(R)
   your_mesh.translate(x.reshape(3))
   scale = your_mesh.points.flatten()
   axes.add_collection3d(mplot3d.art3d.Poly3DCollection(your_mesh.vectors,alpha=0.3))
   axes.auto_scale_xyz(scale, scale, scale)
   axes.set_xlim(-LIM,LIM);axes.set_ylim(-LIM,LIM);axes.set_zlim(-LIM,LIM)

   az = 80-(i*2) # kamera dondurmek icin
   axes.view_init(azim=az,elev=28)
   plt.savefig('/tmp/rotate_%02d.jpg' % i)
   plt.close('all')    
\end{minted}

\includegraphics[width=20em]{sim2/rotate_00.png}
\includegraphics[width=20em]{sim2/rotate_05.png}

\includegraphics[width=20em]{sim2/rotate_12.png}
\includegraphics[width=20em]{sim2/rotate_18.png}


\begin{minted}[fontsize=\footnotesize]{python}
! convert -delay 20 -loop 0 /tmp/rotate*.png /tmp/torus_rotate2.gif
\end{minted}

Animasyon sonucu [2]'de görülebilir. Hareket mantıklı gözüküyor, unutmayalım
grafikleme açısından kolay olduğu için öyle çizildi fakat aslında hesaplara göre
vektörün ucu kuvvet uygulanan noktada, ve uygulanan kuvvet sonrası şekil dönmeye
başlayarak ve yukarı doğru uçarak devam ediyor (kuvvet alttan yukarı doğru).
Simülasyon ortamı boşluk ortamı, uzay gibi yerçekimsiz bir yer, tek kuvvet ilk
başta şekle uygulanan kuvvet, ardından momentum muhafazası ile hareket devam
ediyor.


Kaynaklar

[1] Bayramlı, {\em Animasyon 1},
    \url{https://www.dropbox.com/scl/fi/l9wjyc2nar8bwucasfqpf/torus_rotate1.gif?rlkey=mhnye63g5auddh7m3e993ic43&st=ttluuezu&raw=1}

[2] Bayramlı, {\em Animasyon 2},
    \url{https://www.dropbox.com/scl/fi/ydmge867dfue6hzeya4ai/torus_rotate2.gif?rlkey=wob4ojwrkq4siwz18x8t7ggqf&st=zgz2un0j&raw=1}

[3] Rotenberg, {\em CSE169: Computer Animation, UCSD}

[4] Bayramlı, {\em Lineer Cebir, Ders 5}

[5] Witkin, {\em Physically Based Modeling}

[6] Bayramlı, {\em 3D Baskıya Hazır CAD Tasarımlarına Erişmek, Numpy-STL},
    \url{https://burakbayramli.github.io/dersblog/sk/2020/08/numpy-stl.html}

[7] Witkin, {\em Physically Based Modeling}

[8] Bayramlı, {\em Fizik, Temel Fizik 4, Katı Gövde}

[9] Bayramlı, {\em Fizik, Simulasyon}

\end{document}
