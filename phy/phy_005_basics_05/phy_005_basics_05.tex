\documentclass[12pt,fleqn]{article}\usepackage{../../common}
\begin{document}
Hareketin Katı-Gövde Denklemleri - 1

Runge-Kutta Hesabı

İleride lazım olacak bir hesapsal yöntemi görelim, katı gövde cisimlerinin
hareketi için diferansiyel denklemleri entegre etmemiz gerekiyor, bunun
için Runge-Kutta yaklaşımını bir örnek üzerine görebiliriz.

İlk Uygulanan Kuvvet

İki boyutlu ortama ağırlığı $m=1$ kg olan bir topu 45 derece açıyla sağ yukarı
doğru atıyoruz, uyguladığımız kuvvet 500 Newton. Bu topun gidiş yönü nasıl
hesaplanır?

Önemli bir konu uygulanan kuvveti başlangıçta sisteme nasıl dahil edeceğimiz.
Burada dürtü / vuruş / darbe (impulse) kavramına bakmak gerekiyor. Kuvvetle hız
değişimini ilintilendirmek için Newton'un ikinci kanunu ile başlayalım,

$$
F = \frac{\ud P}{\ud t} 
$$

Uygulanan kuvvet momentum değişimine eşittir, ki $P = mv$ olmak üzere. Fakat
sonsuz küçük zaman üzerinden iş yapamayız, iki tarafın entegralini alalım,

$$
\int_{t_1}^{t_2} F \ud t = \int_{t_1}^{t_2} \ud P \ud t
$$

$$
J = \int_{t_1}^{t_2} F \ud t = P(t_2) - P(t_1)
$$

Sonuçtaki $J$ dürtü büyüklüğü, kuvvetin zaman üzerinden entegrali, yani
bir zaman aralığında uygulanan kuvvetin toplamı. Bu büyüklük momentum
üzerinde bir değişime yol açıyor, $P(t_1)$ momentumundan $P(t_2)$
değerine geliyoruz.

Üstteki formül şunu söylüyor, momentum değişimi için bir kuvvetin uygulandığı
bir zaman aralığı gerekir. O zaman problem tanımı eksik, 500 Newton'luk kuvvet
uyguladık, peki ne kadar süreli? Devam etmeden önce diyelim ki bir saniye, o
zaman dürtü 500 Newton $\cdot$ saniye (Ns) olur. Dikkat $m$ değeri değişmediği
için tek değişim hız değişkeninde olacaktır, $J = m(v_2 - v_1) = m \Delta v$,
yani dürtü sonrası hızda değişim gözleyeceğiz.

Başlangıçta $v_1=0$, dürtü için 500 N $\cdot$ saniye

$$
J = m v_2 - m v_1 = m v_2 - m(0) \to J = m v_2
$$

$$
\to 500 Ns = m v_2
\to 500 \frac{kg \cdot m/s^2}{1 kg } \cdot s = v_2
\to v_2 = 500 m/s
$$

diyebiliriz.

Ilk Uygulanan Tork

Gerçi alttaki örnek için lazım değil ama ilk uygulanan kuvvet gibi ilk uygulanan
dürtü (impülse), tork kavramı da simülasyonlarda gerekli. Bu durumda açısal
dürtü belli bir zamanda uygulanan torktur, yani torkun zaman üzerinden
entegralidir denir, ve bu dürtü açısal momentumdaki değişime eşittir. İlk
tanımla başlarsak,

$$
J = \int \tau \ud t
$$

Eğer zaman aralığı ufak ve tork sabit ise

$$
J = \tau \Delta t
$$

Diğer yandan açısal momentum eşitliğini hatırlayalım,

$$
L = I \omega
$$

Açısal dürtü açısal momentum değişimine eşit ise

$$
\tau \Delta t = \Delta L = \Delta (I \omega)= I (\omega_2 - \omega_1)
$$

ki $\omega_1$ başlangıç açısal hızı, $\omega_2$ nihai hız. Hız başlangıçta sıfır
olsa, $\omega_1 = 0$, ve nihai hız ile başlangıçta uygulanan tork arasında
bağlantı kurmak istesek

$$
\tau \Delta t = I \omega_2 
$$

$$
\to \omega_2 = I^{-1} \tau \Delta t
$$

Örnek

Artık sistem denklemini entegre edebiliriz, sistem $\dot{x} = v_x$,
$\dot{y} = v_y$ $\dot{v} = a$. Sürekli etki eden tek ivme etkisi yerçekimi
$g$. Sistem bir vektör üzerinden temsil edilebiliyor,

$$
\left[\begin{array}{c}
\dot{x} \\ \dot{y} \\ \dot{v}
\end{array}\right] = F(\vec{x},t)
$$

\begin{minted}[fontsize=\footnotesize]{python}
import numpy as np
import matplotlib.pyplot as plt

def rk4_step(func, dt, t, y):
    k1 = dt * func(t, y)
    k2 = dt * func(t + 0.5 * dt, y + 0.5 * k1)
    k3 = dt * func(t + 0.5 * dt, y + 0.5 * k2)
    k4 = dt * func(t + dt, y + k3)
    return y + (k1 + 2*k2 + 2*k3 + k4) / 6

def projectile_motion_eom(t, X):
    x, y_pos, vx, vy = X

    g = 9.81  # yercekimi
    m = 1     # topun kutlesi

    dx_dt = vx
    dy_dt = vy
    dvx_dt = 0.0 
    dvy_dt = -g  

    return np.array([dx_dt, dy_dt, dvx_dt, dvy_dt])

m = 1.0
initial_force = 500.0  # baslangic kuvveti, Newton 
force_direction = np.array([1.0, 1.0])
force_direction = force_direction / np.linalg.norm(force_direction) 

dt = 0.01  # zaman dilimi (saniye)

# Initial position [x0, y0]
x0 = 0.0
y0 = 0.0

initial_acceleration = initial_force / m * force_direction
initial_velocity_magnitude = initial_force / m 
                                                                                                                                             
initial_velocity_vector = initial_velocity_magnitude * force_direction
vx0 = initial_velocity_vector[0]
vy0 = initial_velocity_vector[1]

print(f"Baslangic Hizi: [{vx0:.2f} m/s, {vy0:.2f} m/s]")

initial_state = np.array([x0, y0, vx0, vy0])

time_points = [0.0]
state_history = [initial_state]
current_state = initial_state
current_time = 0.0

while current_state[1] >= 0:
    current_state = rk4_step(projectile_motion_eom, dt, current_time, current_state)
    current_time += dt
    if current_state[1] < 0: break

    time_points.append(current_time)
    state_history.append(current_state)

state_history = np.array(state_history)
x_positions = state_history[:, 0]
y_positions = state_history[:, 1]
vx_values = state_history[:, 2]
vy_values = state_history[:, 3]

plt.figure(figsize=(10, 6))
plt.plot(x_positions, y_positions)
plt.xlabel('(m)')
plt.ylabel('(m)')
plt.grid(True)
plt.axhline(0, color='black', linestyle='--', linewidth=0.7) # yer
plt.axis('equal') 
plt.savefig('phy_005_basics_05_05.jpg')

print(f"\nBitis Zamani: {time_points[-1]:.2f} saniye")
print(f"Son yer: x = {x_positions[-1]:.2f} m, y = {y_positions[-1]:.2f} m")
print(f"Son hız: vx = {vx_values[-1]:.2f} m/s, vy = {vy_values[-1]:.2f} m/s")
\end{minted}

\begin{verbatim}
Baslangic Hizi: [353.55 m/s, 353.55 m/s]

Bitis Zamani: 72.08 saniye
Son yer: x = 25484.13 m, y = 0.07 m
Son hız: vx = 353.55 m/s, vy = -353.55 m/s
\end{verbatim}

\includegraphics[width=20em]{phy_005_basics_05_05.jpg}


Rotasyon Matrisi ve Türevi

Bir 3 x 3 dönüş matrisi ile herhangi bir vektörü döndürebileceğimizi biliyoruz.
Yersel taşıma daha da basit, 3 boyutlu bir vektör sadece, mevcut konuma
ekleyerek yeni konumu elde ediyoruz.

Bir katı gövdeyi parçacıkları üzerinden alırsak, ve bu gövdenin açısal dönüşsel
olarak hangi yöne baktığını bir dönüş matrisi $R$ ile temsil edersek, her
parçacık üzerinde bu işlemin uygulandığını düşünebiliriz. Ayrıca konumsal
taşınma ve bakılan yön başlangıçtaki bir ``gövde uzayı''na (body space) göre
yapılabilir, gövdenin kütle merkezini dünya kordinatlarının (0,0,0) orijin
noktasında ve yönü herhangi bir (başta belli) yöne doğru alalım, hareketler hep
bu konuma referansla, onu değiştirecek şekilde düşünülebilir.  Mesela gövde
üzerindeki, gövde uzayındaki, herhangi bir $p_0$ noktasını düşünelim, $t$ anında
bu noktanın dünya uzayındaki konumu

$$
p(t) = R(t) p_0 + x_{CM}(t)
$$

ki $x_{CM}(t)$ bir yersel taşınma, ve $R(t)$ açısal dönüş. Tabii taşınma her
zaman kütle merkezine uygulandığı için $x_{CM}$ aynı zamanda kütle merkezinin
her $t$ anında dünya uzayında olduğu yeri de gösteriyor.

\includegraphics[width=25em]{phy_005_basics_04_04.png}

Türeve gelirsek, bir vektör $r$'nin orijin etrafında döndüğünü
düşünelim. Herhangi bir anda bu dönüşün açısal hızı $\omega$ çapraz çarpımla
hesaplanabilir,

\includegraphics[width=10em]{phy_005_basics_04_03.png}

Hız tabii ki sonsuz küçük zamandaki yer değişimi olduğu için onu

$$
\frac{\ud r}{\ud t} = \omega \times r
$$

olarak ta görebiliriz. Şimdi bir katı gövdeyi düşünelim, onun baktığı yön
(orientation) bir matris $R$ içinde. Bu matrisin her kolonunda bir eksen var,
ilk kolon $x$, ikinci $y$, vs. Eğer gövdenin baktığı yönü $R$ ile temsil
ediyorsak tüm bu kolonlar gövde dönerken değişecektir. Eğer dönüş $\omega$ ise
her eksenin açısal hızı $\omega$ demek, o zaman bu eksenlerin, $b,c,d$ diyelim,
açısal hızı ayrı ayrı $\omega \times b$, $\omega \times c$, $\omega \times d$
olarak bulunabilir, ki bunların her biri aynı zamanda ayrı birer türevdir. Tüm
matrisin türevi

$$
\frac{\ud R}{\ud t} = \tilde \omega \cdot R
$$

ki $\tilde \omega$ ile $\omega$'yi eksi bakışımlı [4] bir matris hale getirdik,
böylece çapraz çarpımı noktasal çarpım haline çevirmiş oluyoruz [5, sf. 9], [3].

Devam edelim, diğer konuları daha önce bir gövdenin her bakımdan konumunu,
statüsünü temsil etmek için gerekli matematiği gördük. Bu konumu
$\overline{X}(t)$ ile gösterebiliriz,

$$
\overline{X} = \left[\begin{array}{c}
x_{CM}(t) \\ R(t) \\ P(t) \\ L(t)
\end{array}\right]
$$

Momentum $P(t) = v(t) M$ olduğu için $v(t) = \frac{P(t)}{M}$.

$I(t)$'yi yukarıda gördük, $I(t) = R(t) I_{body} R(t)^T$.

$L(t) = I(t) \omega(t)$ olduğu için $\omega(t) = I(t)^{-1} L(t)$

Hepsini biraraya koyunca $\overline{X}$'nin türevi

$$
\frac{\ud}{\ud t} \overline{X}(t) =
\frac{\ud}{\ud t}
\left[\begin{array}{c}
x_{CM}(t) \\ R(t) \\ P(t) \\ L(t)
\end{array}\right]
=
\left[\begin{array}{c}
v(t) \\ \tilde \omega \cdot R(t) \\ F(t) \\ \tau(t)
\end{array}\right]
$$

Katı-Gövde Simülasyonu

Dönüş

Bir örnek gövde üzerinde simülasyon yapmaya uğraşalım. Elimizde bir simit, ya da
geometride torus denen bir şekil var. Bu dosya STL denen bir format içinde,
detaylar için [6]. Kuvvet uygulama sonrası lineer ve açısal momentum içeren
simülasyon için pek çok değişkeni diferansiyel tanımları üzerinden entegre
etmemiz gerekiyor, daha basit bir örnek ile, özellikle sabit bir açısal hız
üzerinden salt döndürme ile başlamak uygun olabilir. [2]'te tarif edilen
döndürme matrisi türevini hatırlarsak,

$$
\frac{\ud R}{\ud t} = \tilde \omega \cdot R
$$

Döndürmeyi bir $\omega$ etrafında düşünüyorduk, $\omega$'nin büyüklüğü
açısal dönme hızına tekabül ediyordu, ve $\tilde \omega$ eksi-bakışımlı
matris idi.

\includegraphics[width=25em]{compscieng_bpp32sim_rigbod_01.png}

Tüm bunları entegre edici \verb!odeint! çağrısının kabul edeceği bir formda
nasıl kullanırız? Bu çağrı düzleştirilmiş bir liste içinde diferansiyel
sonuçların, ve ana değişkenlerin olmasını bekliyor. O zaman $R$'yi kolon bazlı
olmak üzere düzleştiririz, ve gerektiği o listeden matris formuna geçeriz, vs.

\begin{minted}[fontsize=\footnotesize]{python}
from scipy.integrate import odeint
from stl import mesh

def skew(a):
   return np.array([[0,-a[2],a[1]],[a[2],0,-a[0]],[-a[1],a[0],0]])

your_mesh = mesh.Mesh.from_file('torus.stl')
prop = your_mesh.get_mass_properties()
R0 = np.eye(3,3)
omega = np.array([1.0,1.0,1.0])
#omega = np.array([0.0,1.0,0.0])
skew_omega = skew(omega)
   
def dRdt(u,t):   
   R1x,R1y,R1z,R2x,R2y,R2z,R3x,R3y,R3z = u
   R = np.array([R1x,R1y,R1z,R2x,R2y,R2z,R3x,R3y,R3z])
   R = R.reshape((3,3)).T
   res = np.dot(skew_omega, R)
   return list(res.T.flatten())

LIM = 5
STEPS = 20
t=np.linspace(0.0, 3.0, STEPS)
R0 = np.eye(3,3)
u0 = R0.flatten()
u1=odeint(dRdt,list(u0),t)
\end{minted}

Üstte görülen mesela \verb!R1x! $R$ matrisinin 1'inci kolonunun $x$ değişkeni
anlamında.

Simülasyonda simit şeklinin baktığı yön $R$ içinde, ve grafik amaçlı olarak her
seferinde simit şeklini sıfırdan yükleyip son $R$'ye ilerletiyoruz, ve her
adımda bu grafiği basıyoruz. Simülasyonu hesapladık, tüm sonuç \verb!u1!
içinde, görüntüden bazı seçilmiş kareler altta görülebilir,

\begin{minted}[fontsize=\footnotesize]{python}
import matplotlib.pyplot as plt
from mpl_toolkits import mplot3d

def plot_vector(fig, orig, v, color='blue'):
   ax = fig.gca(projection='3d')
   orig = np.array(orig); v=np.array(v)
   ax.quiver(orig[0], orig[1], orig[2], v[0], v[1], v[2],color=color)
   ax = fig.gca(projection='3d')  
   return fig

for i in range(STEPS):
    fig = plt.figure()
    axes = mplot3d.Axes3D(fig)
    your_mesh = mesh.Mesh.from_file('torus.stl')
    R = u1[i].reshape((3,3)).T
    your_mesh.rotate_using_matrix(R)
    scale = your_mesh.points.flatten()
    axes.add_collection3d(mplot3d.art3d.Poly3DCollection(your_mesh.vectors,alpha=0.3))
    plot_vector(fig, [0,0,0], omega, color='red')
    axes.auto_scale_xyz(scale, scale, scale)
    axes.set_xlim(-LIM,LIM);axes.set_ylim(-LIM,LIM);axes.set_zlim(-LIM,LIM)
    axes.view_init(azim=20,elev=0)
    plt.savefig('/tmp/rotate_%02d.png' % i)  
\end{minted}

\includegraphics[width=20em]{sim1/rotate_00.png}

\includegraphics[width=20em]{sim1/rotate_08.png}

\includegraphics[width=20em]{sim1/rotate_14.png}

\begin{minted}[fontsize=\footnotesize]{python}
! convert -delay 20 -loop 0 /tmp/rotate*.png /tmp/torus_rotate1.gif
\end{minted}

Animasyon sonucu [1]'de.

Torus şekli hakkında bazı istatistikler alttadır.

\begin{minted}[fontsize=\footnotesize]{python}
from stl import mesh

your_mesh = mesh.Mesh.from_file('torus.stl')

prop = your_mesh.get_mass_properties()
print ('hacim',np.round(prop[0],3))
print ('yercekim merkezi (COG)',np.round(prop[1],3))
print ('COG noktasinda atalet matrisi')
print (np.round(prop[2],3))
\end{minted}

\begin{verbatim}
hacim 4.918
yercekim merkezi (COG) [-0.  0. -0.]
COG noktasinda atalet matrisi
[[ 3.223 -0.     0.   ]
 [-0.     3.223  0.   ]
 [ 0.     0.     5.832]]
\end{verbatim}

COG sıfır noktasında olması, ayrıca atalet matrisinin köşegen olması mantıklı
çünkü simit şekli simetrik.

Üstteki simülasyonda kuvvetler, lineer, açışal momentum gibi kavramlar
gösterilmedi, bir sonraki yazıda tüm kavramların olduğu bir simülasyon olacak,
yazıda dönüş matematiği de sayısal açıdan daha sağlam farklı bir yaklaşım
kullanıyor olacak.  

Kaynaklar

[1] Bayramlı, {\em Animasyon 1},
    \url{https://www.dropbox.com/scl/fi/l9wjyc2nar8bwucasfqpf/torus_rotate1.gif?rlkey=mhnye63g5auddh7m3e993ic43&st=ttluuezu&raw=1}

[2] Bayramlı, {\em Fizik, Temel Fizik 4, Katı Gövde}

[3] Rotenberg, {\em CSE169: Computer Animation, UCSD}

[4] Bayramlı, {\em Lineer Cebir, Ders 5}

[5] Witkin, {\em Physically Based Modeling}

[6] Bayramlı, {\em 3D Baskıya Hazır CAD Tasarımlarına Erişmek, Numpy-STL},
    \url{https://burakbayramli.github.io/dersblog/sk/2020/08/numpy-stl.html}

\end{document}
