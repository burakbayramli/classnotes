\documentclass[12pt,fleqn]{article}\usepackage{../../common}
\begin{document}
Temel Fizik 4 - Atalet Matrisi (Inertia Matrix, Tensor)

Bir objenin havaya fırlatıldığını düşünelim, fırlatma sırasında dönüş te var,
çetrefil bir hareket sözkonusu yani. Fakat şimdiye kadar gördüğümüz teknikler
ile hala bu hareketi analiz edebiliriz, hem lineer momentum, hem de açısal
momentum kütle merkezi odaklı olarak analiz edilebiliyor. Herhangi bir katı
gövde, cisim şeklini ve hareketi analiz için şimdi bazı genel formülleri
ortaya koyalım. 

Gövdenin açısal momentumu $L$ için,

$$
L = \sum m_i r_i \times v_i
$$

ki $L,r,v$ vektör. $v = \omega \times r$ eşitliğini üste sokarsak,

$$
= \sum m_i r_i \times (\omega \times r_i)
$$

Şimdi bu son ifadenin her vektörü öğelerini kullanarak açılımını yapalım böylece
başka bir forma erişmeyi umuyoruz. $\omega = [\begin{array}{ccc} \omega_x&\omega_y&\omega_z \end{array}]^T$
ve $r = [\begin{array}{ccc} x&y&z \end{array}]^T$ öğelerini kullanacağız, ve
üstteki formülün $A \times (B \times C)$ formunda olduğunu farkediyoruz, o zaman
genel bir $r \times (\omega \times r)$ üzerinde BAC-CAB açılımı yapmayı
deneyebiliriz, bu açılım hatırlarsak,

$$
A \times (B \times C) = B(A \cdot C) - C(A \cdot B)
$$

idi. Kendi denklemimiz üzerinde bu açılım

$$
r \times (\omega \times r) = \omega (r \cdot r) - r(r \cdot \omega)
$$

şeklinde olacaktır. 




















\end{document}
