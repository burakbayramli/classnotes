\documentclass[12pt,fleqn]{article}\usepackage{../../common}
\begin{document}


\begin{minted}[fontsize=\footnotesize]{python}
import matplotlib.pyplot as plt
from mpl_toolkits import mplot3d
import pandas as pd, numpy as np
from stl import mesh
your_mesh = mesh.Mesh.from_file('torus.stl')
#theta = np.deg2rad(20)
theta = np.deg2rad(0)
R = np.array([[1, 0, 0],[0,np.cos(theta),np.sin(theta)],[0,-np.sin(theta),np.cos(theta)]])
print (R)
your_mesh.rotate_using_matrix(R)
fig = plt.figure()
axes = mplot3d.Axes3D(fig)
scale = your_mesh.points.flatten()
axes.add_collection3d(mplot3d.art3d.Poly3DCollection(your_mesh.vectors,alpha=0.3))
axes.auto_scale_xyz(scale, scale, scale)
LIM = 5
axes.set_xlim(-LIM,LIM);axes.set_ylim(-LIM,LIM);axes.set_zlim(-LIM,LIM)
axes.view_init(azim=84,elev=28)

plt.savefig('out.png')
\end{minted}

\begin{verbatim}
[[ 1.  0.  0.]
 [ 0.  1.  0.]
 [ 0. -0.  1.]]
\end{verbatim}


\begin{minted}[fontsize=\footnotesize]{python}
prop = your_mesh.get_mass_properties()
print ('vol',prop[0])
print ('COG',prop[1])
print ('I',)
print (prop[2])
\end{minted}












\begin{minted}[fontsize=\footnotesize]{python}
print (help(mesh))
\end{minted}

\begin{verbatim}
Help on module stl.mesh in stl:

NAME
    stl.mesh

CLASSES
    stl.stl.BaseStl(stl.base.BaseMesh)
        Mesh
    
    class Mesh(stl.stl.BaseStl)
     |  Mesh object with easy access to the vectors through v0, v1 and v2.
     |  The normals, areas, min, max and units are calculated automatically.
     |  
     |  :param numpy.array data: The data for this mesh
     |  :param bool calculate_normals: Whether to calculate the normals
     |  :param bool remove_empty_areas: Whether to remove triangles with 0 area
     |          (due to rounding errors for example)
     |  
     |  :ivar str name: Name of the solid, only exists in ASCII files
     |  :ivar numpy.array data: Data as :func:`BaseMesh.dtype`
     |  :ivar numpy.array points: All points (Nx9)
     |  :ivar numpy.array normals: Normals for this mesh, calculated automatically
     |      by default (Nx3)
     |  :ivar numpy.array vectors: Vectors in the mesh (Nx3x3)
     |  :ivar numpy.array attr: Attributes per vector (used by binary STL)
     |  :ivar numpy.array x: Points on the X axis by vertex (Nx3)
     |  :ivar numpy.array y: Points on the Y axis by vertex (Nx3)
     |  :ivar numpy.array z: Points on the Z axis by vertex (Nx3)
     |  :ivar numpy.array v0: Points in vector 0 (Nx3)
     |  :ivar numpy.array v1: Points in vector 1 (Nx3)
     |  :ivar numpy.array v2: Points in vector 2 (Nx3)
     |  
     |  >>> data = numpy.zeros(10, dtype=BaseMesh.dtype)
     |  >>> mesh = BaseMesh(data, remove_empty_areas=False)
     |  >>> # Increment vector 0 item 0
     |  >>> mesh.v0[0] += 1
     |  >>> mesh.v1[0] += 2
     |  
     |  >>> # Check item 0 (contains v0, v1 and v2)
     |  >>> assert numpy.array_equal(
     |  ...     mesh[0],
     |  ...     numpy.array([1., 1., 1., 2., 2., 2., 0., 0., 0.]))
     |  >>> assert numpy.array_equal(
     |  ... mesh.vectors[0],
     |  ... numpy.array([[1., 1., 1.],
     |  ...     [2., 2., 2.],
     |  ...     [0., 0., 0.]]))
     |  >>> assert numpy.array_equal(
     |  ...     mesh.v0[0],
     |  ...     numpy.array([1., 1., 1.]))
     |  >>> assert numpy.array_equal(
     |  ...     mesh.points[0],
     |  ...     numpy.array([1., 1., 1., 2., 2., 2., 0., 0., 0.]))
     |  >>> assert numpy.array_equal(
     |  ...     mesh.data[0],
     |  ...     numpy.array((
     |  ...             [0., 0., 0.],
     |  ...             [[1., 1., 1.], [2., 2., 2.], [0., 0., 0.]],
     |  ...             [0]),
     |  ...         dtype=BaseMesh.dtype))
     |  >>> assert numpy.array_equal(mesh.x[0], numpy.array([1., 2., 0.]))
     |  
     |  >>> mesh[0] = 3
     |  >>> assert numpy.array_equal(
     |  ...     mesh[0],
     |  ...     numpy.array([3., 3., 3., 3., 3., 3., 3., 3., 3.]))
     |  
     |  >>> len(mesh) == len(list(mesh))
     |  True
     |  >>> (mesh.min_ < mesh.max_).all()
     |  True
     |  >>> mesh.update_normals()
     |  >>> mesh.units.sum()
     |  0.0
     |  >>> mesh.v0[:] = mesh.v1[:] = mesh.v2[:] = 0
     |  >>> mesh.points.sum()
     |  0.0
     |  
     |  >>> mesh.v0 = mesh.v1 = mesh.v2 = 0
     |  >>> mesh.x = mesh.y = mesh.z = 0
     |  
     |  >>> mesh.attr = 1
     |  >>> (mesh.attr == 1).all()
     |  True
     |  
     |  >>> mesh.normals = 2
     |  >>> (mesh.normals == 2).all()
     |  True
     |  
     |  >>> mesh.vectors = 3
     |  >>> (mesh.vectors == 3).all()
     |  True
     |  
     |  >>> mesh.points = 4
     |  >>> (mesh.points == 4).all()
     |  True
     |  
     |  Method resolution order:
     |      Mesh
     |      stl.stl.BaseStl
     |      stl.base.BaseMesh
     |      python_utils.logger.Logged
     |      collections.abc.Mapping
     |      collections.abc.Collection
     |      collections.abc.Sized
     |      collections.abc.Iterable
     |      collections.abc.Container
     |      builtins.object
     |  
     |  Data and other attributes defined here:
     |  
     |  __abstractmethods__ = frozenset()
     |  
     |  logger = <Logger python_utils.logger.ABCMeta (WARNING)>
     |  
     |  ----------------------------------------------------------------------
     |  Methods inherited from stl.stl.BaseStl:
     |  
     |  get_header(self, name)
     |  
     |  save(self, filename, fh=None, mode=<Mode.AUTOMATIC: 0>, update_normals=True)
     |      Save the STL to a (binary) file
     |      
     |      If mode is :py:data:`AUTOMATIC` an :py:data:`ASCII` file will be
     |      written if the output is a TTY and a :py:data:`BINARY` file otherwise.
     |      
     |      :param str filename: The file to load
     |      :param file fh: The file handle to open
     |      :param int mode: The mode to write, default is :py:data:`AUTOMATIC`.
     |      :param bool update_normals: Whether to update the normals
     |  
     |  ----------------------------------------------------------------------
     |  Class methods inherited from stl.stl.BaseStl:
     |  
     |  from_file(filename, calculate_normals=True, fh=None, mode=<Mode.AUTOMATIC: 0>, speedups=True, **kwargs) from abc.ABCMeta
     |      Load a mesh from a STL file
     |      
     |      :param str filename: The file to load
     |      :param bool calculate_normals: Whether to update the normals
     |      :param file fh: The file handle to open
     |      :param dict kwargs: The same as for :py:class:`stl.mesh.Mesh`
     |  
     |  from_files(filenames, calculate_normals=True, mode=<Mode.AUTOMATIC: 0>, speedups=True, **kwargs) from abc.ABCMeta
     |      Load multiple meshes from a STL file
     |      
     |      Note: mode is hardcoded to ascii since binary stl files do not support
     |      the multi format
     |      
     |      :param list(str) filenames: The files to load
     |      :param bool calculate_normals: Whether to update the normals
     |      :param file fh: The file handle to open
     |      :param dict kwargs: The same as for :py:class:`stl.mesh.Mesh`
     |  
     |  from_multi_file(filename, calculate_normals=True, fh=None, mode=<Mode.ASCII: 1>, speedups=True, **kwargs) from abc.ABCMeta
     |      Load multiple meshes from a STL file
     |      
     |      Note: mode is hardcoded to ascii since binary stl files do not support
     |      the multi format
     |      
     |      :param str filename: The file to load
     |      :param bool calculate_normals: Whether to update the normals
     |      :param file fh: The file handle to open
     |      :param dict kwargs: The same as for :py:class:`stl.mesh.Mesh`
     |  
     |  load(fh, mode=<Mode.AUTOMATIC: 0>, speedups=True) from abc.ABCMeta
     |      Load Mesh from STL file
     |      
     |      Automatically detects binary versus ascii STL files.
     |      
     |      :param file fh: The file handle to open
     |      :param int mode: Automatically detect the filetype or force binary
     |  
     |  ----------------------------------------------------------------------
     |  Methods inherited from stl.base.BaseMesh:
     |  
     |  __getitem__(self, k)
     |  
     |  __init__(self, data, calculate_normals=True, remove_empty_areas=False, remove_duplicate_polygons=<RemoveDuplicates.NONE: 0>, name='', speedups=True, **kwargs)
     |      Initialize self.  See help(type(self)) for accurate signature.
     |  
     |  __iter__(self)
     |  
     |  __len__(self)
     |  
     |  __setitem__(self, k, v)
     |  
     |  check(self)
     |      Check the mesh is valid or not
     |  
     |  debug(*args, **kwargs) from abc.ABCMeta
     |      Log a message with severity 'DEBUG' on the root logger. If the logger has
     |      no handlers, call basicConfig() to add a console handler with a pre-defined
     |      format.
     |  
     |  error(*args, **kwargs) from abc.ABCMeta
     |      Log a message with severity 'ERROR' on the root logger. If the logger has
     |      no handlers, call basicConfig() to add a console handler with a pre-defined
     |      format.
     |  
     |  exception(*args, exc_info=True, **kwargs) from abc.ABCMeta
     |      Log a message with severity 'ERROR' on the root logger, with exception
     |      information. If the logger has no handlers, basicConfig() is called to add
     |      a console handler with a pre-defined format.
     |  
     |  get_mass_properties(self)
     |      Evaluate and return a tuple with the following elements:
     |        - the volume
     |        - the position of the center of gravity (COG)
     |        - the inertia matrix expressed at the COG
     |      
     |      Documentation can be found here:
     |      http://www.geometrictools.com/Documentation/PolyhedralMassProperties.pdf
     |  
     |  get_unit_normals(self)
     |  
     |  info(*args, **kwargs) from abc.ABCMeta
     |      Log a message with severity 'INFO' on the root logger. If the logger has
     |      no handlers, call basicConfig() to add a console handler with a pre-defined
     |      format.
     |  
     |  is_closed(self)
     |      Check the mesh is closed or not
     |  
     |  log(msg, *args, **kwargs) from abc.ABCMeta
     |      Log 'msg % args' with the integer severity 'level' on the root logger. If
     |      the logger has no handlers, call basicConfig() to add a console handler
     |      with a pre-defined format.
     |  
     |  rotate(self, axis, theta=0, point=None)
     |      Rotate the matrix over the given axis by the given theta (angle)
     |      
     |      Uses the :py:func:`rotation_matrix` in the background.
     |      
     |      .. note:: Note that the `point` was accidentaly inverted with the
     |         old version of the code. To get the old and incorrect behaviour
     |         simply pass `-point` instead of `point` or `-numpy.array(point)` if
     |         you're passing along an array.
     |      
     |      :param numpy.array axis: Axis to rotate over (x, y, z)
     |      :param float theta: Rotation angle in radians, use `math.radians` to
     |                          convert degrees to radians if needed.
     |      :param numpy.array point: Rotation point so manual translation is not
     |                                required
     |  
     |  rotate_using_matrix(self, rotation_matrix, point=None)
     |  
     |  transform(self, matrix)
     |      Transform the mesh with a rotation and a translation stored in a
     |      single 4x4 matrix
     |      
     |      :param numpy.array matrix: Transform matrix with shape (4, 4), where
     |                                 matrix[0:3, 0:3] represents the rotation
     |                                 part of the transformation
     |                                 matrix[0:3, 3] represents the translation
     |                                 part of the transformation
     |  
     |  translate(self, translation)
     |      Translate the mesh in the three directions
     |      
     |      :param numpy.array translation: Translation vector (x, y, z)
     |  
     |  update_areas(self, normals=None)
     |  
     |  update_max(self)
     |  
     |  update_min(self)
     |  
     |  update_normals(self, update_areas=True)
     |      Update the normals and areas for all points
     |  
     |  update_units(self)
     |  
     |  warning(*args, **kwargs) from abc.ABCMeta
     |      Log a message with severity 'WARNING' on the root logger. If the logger has
     |      no handlers, call basicConfig() to add a console handler with a pre-defined
     |      format.
     |  
     |  ----------------------------------------------------------------------
     |  Class methods inherited from stl.base.BaseMesh:
     |  
     |  remove_duplicate_polygons(data, value=<RemoveDuplicates.SINGLE: 1>) from abc.ABCMeta
     |  
     |  remove_empty_areas(data) from abc.ABCMeta
     |  
     |  rotation_matrix(axis, theta) from abc.ABCMeta
     |      Generate a rotation matrix to Rotate the matrix over the given axis by
     |      the given theta (angle)
     |      
     |      Uses the `Euler-Rodrigues
     |      <https://en.wikipedia.org/wiki/Euler%E2%80%93Rodrigues_formula>`_
     |      formula for fast rotations.
     |      
     |      :param numpy.array axis: Axis to rotate over (x, y, z)
     |      :param float theta: Rotation angle in radians, use `math.radians` to
     |                   convert degrees to radians if needed.
     |  
     |  ----------------------------------------------------------------------
     |  Data descriptors inherited from stl.base.BaseMesh:
     |  
     |  areas
     |      Mesh areas
     |  
     |  attr
     |  
     |  max_
     |      Mesh maximum value
     |  
     |  min_
     |      Mesh minimum value
     |  
     |  normals
     |  
     |  points
     |  
     |  units
     |      Mesh unit vectors
     |  
     |  v0
     |  
     |  v1
     |  
     |  v2
     |  
     |  vectors
     |  
     |  x
     |  
     |  y
     |  
     |  z
     |  
     |  ----------------------------------------------------------------------
     |  Data and other attributes inherited from stl.base.BaseMesh:
     |  
     |  dtype = dtype([('normals', '<f4', (3,)), ('vectors', '<f4', (3, 3)), (...
     |  
     |  ----------------------------------------------------------------------
     |  Static methods inherited from python_utils.logger.Logged:
     |  
     |  __new__(cls, *args, **kwargs)
     |      Create and return a new object.  See help(type) for accurate signature.
     |  
     |  ----------------------------------------------------------------------
     |  Data descriptors inherited from python_utils.logger.Logged:
     |  
     |  __dict__
     |      dictionary for instance variables (if defined)
     |  
     |  __weakref__
     |      list of weak references to the object (if defined)
     |  
     |  ----------------------------------------------------------------------
     |  Methods inherited from collections.abc.Mapping:
     |  
     |  __contains__(self, key)
     |  
     |  __eq__(self, other)
     |      Return self==value.
     |  
     |  get(self, key, default=None)
     |      D.get(k[,d]) -> D[k] if k in D, else d.  d defaults to None.
     |  
     |  items(self)
     |      D.items() -> a set-like object providing a view on D's items
     |  
     |  keys(self)
     |      D.keys() -> a set-like object providing a view on D's keys
     |  
     |  values(self)
     |      D.values() -> an object providing a view on D's values
     |  
     |  ----------------------------------------------------------------------
     |  Data and other attributes inherited from collections.abc.Mapping:
     |  
     |  __hash__ = None
     |  
     |  __reversed__ = None
     |  
     |  ----------------------------------------------------------------------
     |  Class methods inherited from collections.abc.Collection:
     |  
     |  __subclasshook__(C) from abc.ABCMeta
     |      Abstract classes can override this to customize issubclass().
     |      
     |      This is invoked early on by abc.ABCMeta.__subclasscheck__().
     |      It should return True, False or NotImplemented.  If it returns
     |      NotImplemented, the normal algorithm is used.  Otherwise, it
     |      overrides the normal algorithm (and the outcome is cached).

FILE
    /home/burak/Documents/env3/lib/python3.6/site-packages/stl/mesh.py


None
\end{verbatim}




\end{document}
