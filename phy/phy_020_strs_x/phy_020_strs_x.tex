\documentclass[12pt,fleqn]{article}\usepackage{../../common}
\begin{document}
Materyel Mekaniği - Ekstralar

Gerinim Tensörü (Strain Tensor) 

Önce nesneleri nasıl temsil ettiğimizden bahsedelim [2]. Diyelim ki elimizde bir
patates var. Fakat bu patatesin matematiksel olarak bir anlamı yok. Eğer bu
nesneyi $R^3$ uzayında temsil etmek istiyorsak, onun üzerindeki belli seçilmiş
noktalar sayesinde bunu yapabiliriz.

\includegraphics[width=8em]{phy_020_strs_01_01.jpg}

Nesne üzerindeki mavi noktalar bu seçilmiş noktaları gösteriyor.

Seçilmiş noktaların kordinatı bir referansa göre alınmalı, $e_1,e_2,e_3$
şeklinde bir baz bu işi yapabilir. Artık bu baza, kordinat sistemine izafi
olarak patates üzerindeki her noktayı bir vektör olarak temsil edebiliyoruz.
Altta örnek olarak üç tane seçilmiş noktayı gösterdik,

\includegraphics[width=13em]{phy_020_strs_01_02.jpg}

Daha fazla nokta da seçebilirdik, tüm seçilmiş noktalardan gelen vektörlerin
kümesi cisim hakkında bize bir konum, durum bilgisi verecektir, bu konuma biçim
değiştirme öncesi noktaların konumu $\Omega_0$ diyelim, ya da referans konumu.
Nesne üzerindeki değişimler, özellikle bu ders sonlu öğeler (finite elements
method, FEM) dersi olduğu için deformasyon değişimleri referans vektörlerinin
nasıl değiştiği üzerinden incelenebilir. İlk konumdaki bir vektörü, $X$ diyelim,
değişimi $f$ fonksiyonu yapıyor olsun, sonuç vektörü $x$ olacak, yani $x =
f(X)$.

\includegraphics[width=17em]{phy_020_strs_01_03.jpg}

Üstteki resimde örnek bir değişim görüyoruz; yana kayma, dönme, uzama
var. Değişimi gerçekleştiren $f$ fonksiyonu. Bu ders için farz edilen $f$'nin
birebir ve örten (bijective) olduğu, liner cebirden hatırlarsak bu $f$'nin tersi
alınabilir olduğu anlamına geliyor, yani elimde deforme edilmiş konum var ise,
$f$'nin tersi ile başlangıç konumuna dönebilirim. Diğer bir faraziye fonksiyonun
sürekli (continuous), ve pürüzsüz (smooth), yani türevi alınabilir olduğu. Katı
cisim mekaniğinde türevi alınabilirlik önemli bir faraziyedir, gerçek hayatta
böyle mi, her zaman değil muhakkak, hatta bir bakıma bu sebepten dolayı FEM'e
ihtiyacımız var.

Ayrıca bize ileride lazım olabilecek bir üçüncü vektör $u$ da tanımlayabiliriz,
bu vektör varılan konumu referans konumuna direk ilintilendiriyor. Vektör $u$'ya
yer değişim fonksiyonu denir. Pozisyon fonksiyonu ile karıştırmayalım, o küçük
$x$, bu yer değişim fonksiyonu $u$.

\includegraphics[width=17em]{phy_020_strs_01_05.jpg}

Fakat bu tüm grafiğe baktığımızda $u$'nun aslında vektör çıkartma operasyonunu
gösterdiğini fark edebiliriz, yani $u = x - X$.

Üç tür katı gövde değişimine bakalım şimdi, not katı demek gövde esneyip,
uzamıyor demek.

Katı Gövde Yer Değişimi: $x = X + c$, ki $c$ sabit bir vektör. Pür yer değişimi
olduğu için basit bir toplanma işlemi sadece. Şimdi bildiğimiz sonuç konumu
formülünü yazarsak, $u = x - X$ bu formülde önceki $x$'i geçirelim, $u = X + c -
X$ yani $u = c$.

Katı Gövde Dönüşü: $x = Q X$, formüldeki $Q$ bir dönüş matrisidir. Tekrar yer
değişim formülünü yazalım, $u = x - X$ ve önceki $x$'i yerine koyalım, $u = QX -
X$, tekrar düzenlersek, $u = (Q-I)X$.

Katı Gövde Hareketi: $x = QX + c$, bu kalem aslında önceki iki kalemin
birleşimi, hem dönüş hem de yer değişimi var. Çoğunlukla fizik problemlerinde bu
kavramdan bahsedilir. Yine $u$'yu düşünürsek $u = (Q-I)X + c$ elde ederiz.

\includegraphics[width=10em]{phy_020_strs_01_04.jpg}

Bahsedilen değişim türleri üstte resmedildi, yanlız hala gerinim, esneme,
küçülme türü şekil değişimlerini görmedik. Diğerleri şunlar,

Her Yönde Uzama ve Küçülme (Uniform Extension and Contraction)

Her kordinat ekseninde uzama var ise, mesela $x_1 = k_1 X_1$, $x_2 = k_2 X_2$,
$x_3 = k_3 X_3$, ki $k_i > 0$ reel sayı olmak üzere. Bu değişimleri matris
ile şöyle gösterebiliriz,

$$
x = \left[\begin{array}{c}
x_1 \\ x_2 \\ x_3
\end{array}\right] =
\left[\begin{array}{ccc}
k_1 & 0   &   0 \\
0   & k_2 & 0 \\
0   & 0   & k_3
\end{array}\right]
\left[\begin{array}{c}
X_1 \\ X_2 \\ X_3 
\end{array}\right] =
\left[\begin{array}{c}
k_1 X_1 \\ k_2 X_2 \\ k_3 X_3 
\end{array}\right]
$$

Üstteki ifadeden pozisyon fonksiyonunu şöyle belirtebiliriz,

$$
u = x - X = \left[\begin{array}{ccc}
(k_1 - 1) X_1 \\ 
(k_2 - 1) X_2 \\ 
(k_3 - 1) X_3 
\end{array}\right]
$$

İki boyutta eğer hem $k_1$ hem $k_2$ 1'den büyükse her iki eksende esneme
görülürdü,

\includegraphics[width=10em]{phy_020_strs_01_06.jpg}

Eğer sadece $k_1 > 1$ ise o yönde büyüme olurdu, diğerinde değil. Eğer
$0 < k < 1$ ise küçülme, $k > 1$ ise büyüme, $k = 1$ değişim yok.

$k=0$ ya da $k<0$ fiziki dünyada mümkün değil, mesela $k<0$ durumunda nesnenin
tamamen kendi tersine dönmüş olması gerekiyor, bu da fiziksel olarak olamaz.

Basit Kaykılma (Simple Shear)

Üç boyutta $e_2$ etrafında bir kaykılma düşünüyor olsaydık, bunu

$$
x_1 = X_1 + k X_2, \qquad x_2 = X_2, \qquad x_3 = X_3
$$

ile temsil edebilirdik, $\theta$ açısında bir kaykılma için matris formunda

$$
x = \left[\begin{array}{ccc}
1 & \tan(\theta) & 0 \\
0 & 1 & 0 \\
0 & 0 & 1 
\end{array}\right]
\left[\begin{array}{c}
X_1 \\ X_2 \\ X_3
\end{array}\right] =
\left[\begin{array}{c}
X_1 + \tan(\theta) X_2 \\
X_2 \\
X_3
\end{array}\right] 
$$

$$
\implies u = x - X =
\left[\begin{array}{ccc}
\tan(\theta) X_2 \\
0 \\
0
\end{array}\right]
$$

\includegraphics[width=10em]{phy_020_strs_01_07.jpg}

Pür Kaykılma (Pure Shear)

Bu tür şekil değiştirme birden fazla eksen bazında olabilir,

$$
x = \left[\begin{array}{ccc}
1               & \tan(\theta/2) & 0 \\
\tan(\theta/2) & 1              & 0 \\
0               & 0              & 1 
\end{array}\right]
\left[\begin{array}{c}
X_1 \\ X_2 \\ X_3
\end{array}\right] =
\left[\begin{array}{c}
X_1 + \tan(\theta/2) X_2 \\
\tan(\theta/2) X_1 + X_2  \\
X_3
\end{array}\right] 
$$

$$
\implies u = x - X =
\left[\begin{array}{ccc}
\tan(\theta / 2) X_2 \\
\tan(\theta / 2) X_1 \\
0
\end{array}\right]
$$

\includegraphics[width=10em]{phy_020_strs_01_08.jpg}

Nihayet gerinim konusunda geldik. Bu dersin en önemli konusu bu. Başta
gösterdiğimiz patatese dönelim, referans konumundaki bir mavi noktaya $X$'e
bakıyorduk, patatesi yamulttuğumuz zaman sonuç konumdaki $x$ elde ediliyordu,
her iki noktayı birbiriyle ilintilendiren aralarındaki $u$ vektörü idi.

Şimdi gerinim tensörünü türetmek için patateslere ikinci bir nokta
ekleyeceğiz. Referans konumda $X$ noktası ile bu ikinci nokta arasındaki vektör
$\ud X$ olacak, sonuç patatesindeki ikinci noktaya olan vektör $\ud x$.  Tabii
bu noktaları rasgele eklemedik, yamultan fonksiyon $\ud X$'i yamultunca $\ud x$
elde ettik, sonuç patatesteki ikinci noktaya böyle eriştik.

\includegraphics[width=15em]{phy_020_strs_01_09.jpg}

Şimdi diyelim ki orijinden ikinci patatesteki $P$ noktasına gitmek
istiyoruz. Bir yol $X$, $\ud X$, $u(X+\ud X)$ olabilirdi, bu vektörlerin toplamı
bizi $P$'ye götürür. Fakat daha kısa bir yol daha var, $x$, $\ud x$. Aynı
noktaya eriştiğimize göre bu iki yolun vektör toplamları birbirine eşit olmalı.
O zaman şu ifade doğrudur,

$$
x + \ud x = X + \ud X + u(X+\ud X)
$$

Eğer $x$'i sol taraftan sağa geçirirsem, ve biraz düzenleme sonrası,

$$
\ud x = \ud X + u(X+\ud X) - (x - X)
$$

Tabii daha önceden hatırlıyoruz ki $u(X) = (x - X)$ o zaman 

$$
\ud x = \ud X + u(X+\ud X) - u(X)
\mlabel{1}
$$

Üstteki ifade tanıdık bir kavrama dönüşmeye başladı, eşitliğin sağ tarafı kısmen
bir türeve, gradyana benzemiyor mu? Gradyan tanımı,

$$
\nabla u = \frac{u(X+\ud X) - u(X)}{\ud X}
$$

$\nabla u$'ya yer değişim gradyanı deniyor.  Bunu (1) ifadesine uydurmak için 

$$
\nabla u \ud X = u(X+\ud X) - u(X)
$$


Şimdi (1)'de yerine geçirelim,

$$
\ud x = \ud X + \nabla u \ud X
$$

Bir basitleştirme daha,

$$
\ud x = (I + \nabla u) \ud X
\mlabel{2}
$$

$I + \nabla u$ ifadesine şu şekilde erişmek mümkün, hatırlarsak
$x = X + u(X)$ idi. Eğer değişim gradyanı $F$ olarak $F = \partial x / \partial X$,
ya da $F_{ij} = \frac{\partial x_i}{\partial X_j}$ tanımlarsak [1, sf. 144], 

$$
\frac{\partial (X + u(X))}{\partial X} = 1  + \frac{\partial u}{\partial X} = F
$$

$$
F = I + \nabla u
$$

Üsttekini (2)'ye sokarsak,

$$
\ud x = F \ud X
$$

elde ederiz.

$F$'ye bakmanın bir diğer yolu $x = \Phi(X,t)$ tanımından hareketle doğal olarak
elde edilen

$$
\ud x = \frac{\partial x}{\partial X} \ud X
$$

formülü. Burada $F$ dediğimiz $\ud X$ sonsuz ufak büyüklüğü $\ud x$'e ceviren
şey, açılımı tabii ki $F = \partial x / \partial X$, ya da $F_{ij} = \frac{\partial x_i}{\partial X_j}$.
$x$ için farklı bir formül kullanınca mesela $x = X + u(x)$ bu durumda $F$
hesabı bizi $I + \nabla u$'ya götürecektir.

Devam edelim, dersin geri kalanında gerinim konusunu işlerken hep $\ud x = F \ud
X$ formülünü baz alacağız. Niye?  Çünkü gerinim bir uzunluk değişimidir ve
üstteki formül bana baştaki ufak fark vektörlerinin nasıl uzunluksal olarak
değişime uğradığını gösteriyor. Orijinal pozisyondaki değişim $\ud X$ biliniyor,
buradan değişim sonrasındaki $\ud x$'e geçerek o mesafeyi hesaplayabilirim.


Gradyanlar

Deformasyon ve yer değişim gradyanından bahsettik. Yer değişim gradyanı $\nabla
u$ bir tensör alan (field) sonucunu verir, çünkü $u$ bir vektör değerli
fonksiyondur. Yer değişim (displacement) gradyanı $\nabla u$'nun tam tanımı,

$$
\renewcommand*{\arraystretch}{2.5}
\nabla u = \frac{\partial u_i}{\partial X_j} =
\left[\begin{array}{ccc}
\dfrac{\partial u_1}{\partial X_1} & \dfrac{\partial u_1}{\partial X_2} & \dfrac{\partial u_1}{\partial X_3} \\
\dfrac{\partial u_2}{\partial X_1} & \dfrac{\partial u_2}{\partial X_2} & \dfrac{\partial u_2}{\partial X_3} \\
\dfrac{\partial u_3}{\partial X_1} & \dfrac{\partial u_3}{\partial X_2} & \dfrac{\partial u_3}{\partial X_3} 
\end{array}\right]
$$

Yamulma / Deformasyon Gradyanı $F$

$F$'nin formülü $F = I + \nabla u$ ise üstteki matristen hareketle bu basit
bir hesap,

$$
F = I + \nabla u = 
\left[\begin{array}{ccc}
1 & 0 & 0 \\ 0 & 1 & 0 \\ 0 & 0 & 1
\end{array}\right] + 
\renewcommand*{\arraystretch}{2.5}
\left[\begin{array}{ccc}
\dfrac{\partial u_1}{\partial X_1} & \dfrac{\partial u_1}{\partial X_2} & \dfrac{\partial u_1}{\partial X_3} \\
\dfrac{\partial u_2}{\partial X_1} & \dfrac{\partial u_2}{\partial X_2} & \dfrac{\partial u_2}{\partial X_3} \\
\dfrac{\partial u_3}{\partial X_1} & \dfrac{\partial u_3}{\partial X_2} & \dfrac{\partial u_3}{\partial X_3} 
\end{array}\right]
$$

Üstteki toplam aslında alttaki kısmi türev matrisine eşittir,

$$
\renewcommand*{\arraystretch}{2.5}
F = \left[\begin{array}{ccc}
\dfrac{\partial x_1}{\partial X_1} & \dfrac{\partial x_1}{\partial X_2} & \dfrac{\partial x_1}{\partial X_3} \\
\dfrac{\partial x_2}{\partial X_1} & \dfrac{\partial x_2}{\partial X_2} & \dfrac{\partial x_2}{\partial X_3} \\
\dfrac{\partial x_3}{\partial X_1} & \dfrac{\partial x_3}{\partial X_2} & \dfrac{\partial x_3}{\partial X_3} 
\end{array}\right] =
\frac{\partial x_i}{\partial X_j}
$$

Yani elde ettiğimiz pozisyon fonksiyonunun gradyanıdır. Nihai olarak eşitliği
$\nabla u = F - I$ olarak ta gösterebilirdik, yani,

$$
\renewcommand*{\arraystretch}{2.5}
\nabla u =
\left[\begin{array}{ccc}
\dfrac{\partial u_1}{\partial X_1} & \dfrac{\partial u_1}{\partial X_2} & \dfrac{\partial u_1}{\partial X_3} \\
\dfrac{\partial u_2}{\partial X_1} & \dfrac{\partial u_2}{\partial X_2} & \dfrac{\partial u_2}{\partial X_3} \\
\dfrac{\partial u_3}{\partial X_1} & \dfrac{\partial u_3}{\partial X_2} & \dfrac{\partial u_3}{\partial X_3} 
\end{array}\right] = 
\left[\begin{array}{ccc}
\dfrac{\partial x_1}{\partial X_1} & \dfrac{\partial x_1}{\partial X_2} & \dfrac{\partial x_1}{\partial X_3} \\
\dfrac{\partial x_2}{\partial X_1} & \dfrac{\partial x_2}{\partial X_2} & \dfrac{\partial x_2}{\partial X_3} \\
\dfrac{\partial x_3}{\partial X_1} & \dfrac{\partial x_3}{\partial X_2} & \dfrac{\partial x_3}{\partial X_3} 
\end{array}\right] -
\renewcommand*{\arraystretch}{1.0}
\left[\begin{array}{ccc}
1 & 0 & 0 \\ 0 & 1 & 0 \\ 0 & 0 & 1
\end{array}\right]
$$





Şimdi Lagrange (Green) tensörüne gelelim.

Gerinim, yamulma sonucu uzunluk değişimidir. $\ud x = F \ud X$ formülü vektörler
arası değişimi gösteriyor, formülü vektör uzunluğu (norm) kullanacak şekilde
değişterebiliriz. Eşitliğin iki tarafını kendisi ile noktasal çarpıma tabi
tutarım, böylece her iki tarafta norm elde ederim,

$$
\ud x \cdot \ud x  = (F \ud X) \cdot (F \ud X)
$$

$$
||\ud x||^2  = \ud X \cdot (F^T F) \ud X
$$

Son eşitlik ortaya çıktı çünkü bir $Ax$ örneği üzerinden bakarsak,

$$
Ax \cdot Ax = (Ax)^T Ax = x^T A^T (Ax) = x^T (A^T A) x  = x \cdot (A^T A) x
$$

Ana konumuza dönersek, iki üstteki $F^T F$ grubuna $C$ diyelim, bu büyüklük
Sağ Cauchy-Green Yamulma Tensörü (Right Cauchy-Green Deformation Tensor)
olarak biliniyor.

Bir zihin egzersizi yapalım, eğer $C = F^T F = I$ yani $C$ birim matristir
dersem ne olur? O zaman

$$
||\ud x||^2  = \ud X \cdot I \ud X = \ud X \cdot \ud X
$$

$$
||\ud x||^2 = ||\ud X||^2
$$

Bu bize uzunlukları değişmediği bir ortamı gösteriyor, demek ki elimizde bir
katı-gövde hareketi (rigid-body motion) bulunuyor. Gövdede hiç esneme, gerinim,
uzama yok, obje şeklen olduğu gibi kalıyor.

Normal şartlarda $F = (I+\nabla u)$ ve $F^T F$ birim matris değil, bu durumda
tüm $C$'yi açarsak,

$$
C = F^T F = (I+\nabla u)^T (I+\nabla u) =
I + \nabla u + \nabla u^T + \nabla u^T \nabla u
$$

Katı-gövde için $C = I$ demiştik, üstteki son ifadede $C = I + ...$  gibi
bir açılım var, nokta nokta olan yerler tabii ki diğer semboller. Ama
tekrar katı-gövde durumunu düşünürsek, o durumda $C = I$ olacağı için
üstteki formülde $I$ sonrası gelen,

$$
.. + \nabla u + \nabla u^T + \nabla u^T \nabla u
$$

terimleri tamamen sıfır demektir.

Şimdi Cauchy-Green tensörünü tekrar tanımlayalım,

$$
C = I + 2 \epsilon_{Green} 
$$

öyle ki 

$$
\epsilon_{Green} = \frac{1}{2} (\nabla u + \nabla u^T + \nabla u^T \nabla u )
\mlabel{3}
$$

Not: $C$ içinde 2 ile çarpım var $\epsilon_{Green}$ içinde 1/2 bunlar birbirini
iptal eder ama 1/2 kullanımı karesel formlarda yaygın şekilde kullanılır,
basitleştirici bir amacı var muhakkak. 

Böylece Green Gerinim Tensörünü tanımlamış olduk.

Tensörün bizim için faydalı bazı özellikleri var.

1) Simetrik: Bu özellik hesapları rahatlaştırır, özellikle yaklaşık
hesaplara gelince simetrinin faydalarını göreceğiz. 

2) Sonlu (finite) yamulmalar için geçerli: Tensör hesabı için $\nabla u$
kullanılır, onun için de yer değişim fonksiyonu yeterlidir. Bunlar varsa geçerli
bir formül eldedir.

Sonsuz Küçük Gerinim Tensörü (Infinitesimal Strain Tensor)

Green gerinim tensörünü gördük, kuvvetli bir yaklaşım ama bizim daha çok
kullanacağımız şimdi anlatacağımız. Niye? Çünkü Green tensörü sonlu
deformasyonlar / yamulmalar için geçerli ama çoğu uygulamada bize lazım olan çok
ufak yamulmalar. Ufak derken, önceki dersteki (3) formülünden hareketle, oradaki
en son terimi hatırlarsak, çok ufak yamulmalar için $\nabla u^T \nabla u << \nabla u$ 
olur, yani ufak değişimlerde o karesel işlem $\nabla u$'dan daha ufak sonuç
verir. O zaman belli durumlarda son terim yaklaşık sıfır kabul edilebilir,
$\nabla u^T \nabla u \approx 0$. Green tensörü bu durumlarda yaklaşık
olarak alttaki gibi olur,

$$
\epsilon_{Green} \approx \frac{1}{2} (\nabla u + \nabla u^T )
$$

Tüm öğeleri gözükecek şekilde [3, 4.3.2.2], 

$$
\left[\begin{array}{ccc}
  \frac{\partial u_1}{\partial X_1} &
  \frac{1}{2}(\frac{\partial u_1}{\partial X_2} + \frac{\partial u_2}{\partial X_1} ) & 
  \frac{1}{2}(\frac{\partial u_1}{\partial X_3} + \frac{\partial u_3}{\partial  X_1} )
\\
  \frac{1}{2}(\frac{\partial u_1}{\partial X_2} + \frac{\partial u_2}{\partial X_1} ) &
  \frac{\partial u_2}{\partial X_2} &
  \frac{1}{2}(\frac{\partial u_2}{\partial X_3} + \frac{\partial u_3}{\partial X_2} )
\\
  \frac{1}{2}(\frac{\partial u_1}{\partial X_3} + \frac{\partial u_3}{\partial X_1} ) &
  \frac{1}{2}(\frac{\partial u_2}{\partial X_3} + \frac{\partial u_3}{\partial X_2} ) &
  \frac{\partial u_3}{\partial X_3} 
\end{array}\right]
$$

Bileşen formunda

$$
\epsilon_{ij} = \frac{1}{2}\left(
\frac{\partial u_i}{\partial X_j} + \frac{\partial u_j}{\partial X_i}
\right)
$$

Bu tensör de simetrik, fakat sadece ufak şekil değişimleri, yamulmalar için
geçerli. Fakat zaten, mesela inşaat mühendisliği durumunda, binalar, demir
kirişler (beam) ile iş yaptığımız zaman, bu tür şekil değişimi faraziyesi
yeterli. Çünkü eh, biraz düşünürsek eğer binamız büyük şekil değişimleri
yaşıyorsa önümüzde daha büyük bir problem var demektir.

Cok Boyutta Kuvvetler

Giriş dersinden hatırlarsak kuvvet uygulanan kirişlerdeki deformasyon maddenin
özellikleriyle ilişkilendirilebiliyordu, bunun için birim alandaki kuvvet ve
birim boya tekabül eden uzama gözönüne alınıyordu. Çok boyuta geçerken ilk
olarak birim alan bazlı içsel kuvvetlerin alttaki gibi genel bir nesneye nasıl
uygulanacağını görelim [4, sf. 184].

\includegraphics[width=15em]{phy_020_strs_02_16.jpg}

Bir $O$ noktasına etki eden iç kuvvetleri incelemek için o noktadan geçen bir
düzlem hayal edebiliriz, düzlemi temsil eden ona dik normal vektör $n$ olsun.
Eğer düzlemi ufak parçalara bölsek ve her bölgeye etki eden kuvvetleri ölçsek
oradaki etki eden kuvvetlerin birinden diğerine değişebileceğini görürdük.

\includegraphics[width=15em]{phy_020_strs_02_17.jpg}

Eğer benzer şekilde $O$ merkezli bir ufak kare $\Delta A$ ele alsak orada etki
eden bir $\Delta F$ olacaktır. $\Delta F$'nin düzleme dik olması gerekmez,
herhangi bir açıda duran bir vektör olabilir, cismin altındaki kuvvetler üstten
alta doğru bastıran kuvvetlerden daha büyükse $\Delta F$ onların tek bir
noktadaki bir tür birleşimi olduğu için yukarı doğru gösteriyor olurdu muhakkak.

Şimdi stres vektörü kavramını tanıştıralım; eğer $\Delta A$ limite doğru giderse

$$
T_n = \lim_{\Delta A \to 0} \frac{\Delta F}{\Delta A}
$$

büyüklüğü stres vektörünü tanımlar.

Dikkat edersek $\Delta F$ büyüklüğü düzlemin duruşuna, yani $n$'ye bağlı olduğu
için özellikle $n$ ibaresini $T$ sembolüne ekledik; her değişik $n$ değişik bir
$T_n$ değerini verebileceği için.

Herhangi bir düzlem kullanabiliriz demiştik, fakat tekrarlanabilirlik, net ifade
açısından her eksene dik birer düzlem, toplam üç tane kullanmak daha iyi
olacak. Örnek $x$ eksenine dik olan bir düzlem altta,

\includegraphics[width=20em]{phy_020_strs_02_18.jpg}

Daha önce gördüğümüz $\Delta F$'in üstteki resimde düzleme göre bileşenlerine
ayıracağız, bunlar $\Delta F_x$, $\Delta F_y$, $\Delta F_z$. Baktığımız alan ise
kenarları $\Delta y$ ve $\Delta z$ olan bir dikdörtgen, alan $\Delta A_x$ ise
(notasyon olarak dik olduğumüz eksenin sembolünü verdik)
$\Delta A_x = \Delta y \Delta z$.

Daha önce olduğu gibi burada da limit tekniğini kullanabiliriz, $\Delta A_x \to
0$ olacak. Fakat yine notasyonel olarak referans eksen yönündeki strese $\sigma$
sembolü üzerinden normal stres , o eksene dik yani düzleme paralel olan
bileşenlere $\tau$ üzerinden kaykılma (shear) stresi adlarını vereceğiz.
Limitlerle beraber,

$$
\sigma_x = \lim_{\Delta A_x \to 0 } \frac{\Delta F_x}{\Delta A_x}
$$

$$
\tau_{xy} = \lim_{\Delta A_x \to 0 } \frac{\Delta F_y}{\Delta A_x}
$$

$$
\tau_{xz} = \lim_{\Delta A_x \to 0 } \frac{\Delta F_z}{\Delta A_x}
$$

Aynı düzlemle kesme tekniğini iki diğer eksen $y,z$ için de kullanırsak, ve
benzer hesapları yaparsak oradan da altı tane stres değeri elde ederiz, toplam
dokuz tane, hepsi bir arada bir matris içinde,

$$
\left[\begin{array}{rrr}
\sigma_x & \tau_{xy} & \tau_{xz} \\
\tau_{yx} & \sigma_y & \tau_{yz} \\
\tau_{zx} & \tau_{zy} & \sigma_z
\end{array}\right]
$$





















Potansiyel Enerji ve Denge

İlk önce denge bağlamında potansiyel enerjinin ne demek olduğunu işleyelim.

Potansiyel enerji $\Pi$ sistemin stabilitesi ile alakalıdır. Mesela alttaki
resim stabilite konusunu işleyen her ders kitabında vardır, bir kapta duran topu
aldım, yukarı doğru çıkatıp (bordo renk) aşağı bıraktım, top kabın dibine gidip
orada kalacaktır (kırmızı renk). 

\includegraphics[width=10em]{phy_020_strs_04_01.jpg}

Yani top ilk denge konumuna dönecektir, ve o durumda potansiyel enerjisi minimum
olmuştur deriz ve bu denge stabil bir dengedir.

\includegraphics[width=10em]{phy_020_strs_04_02.jpg}

İkinci durumda topu orta noktada sola taşırız, top orada kalır, bu yeni
bir denge noktasıdır, $\Pi$ değişmemiştir, burada nötr bir denge vardır.

\includegraphics[width=10em]{phy_020_strs_04_03.jpg}

Üçüncü durumda ters kavisli bir yüzey var, top üst orta noktadan başlıyor
diyelim (orada durması zor olsa da), topu yine alıp sola taşıyorum, top aşağı
düşecektir. Üstteki durum potansiyel enerjisinin maksimum olduğu bir durumdur,
sistem stabil değildir. Rayleigh-Ritz yönteminin amacı (potansiyel enerjinin
minimum olduğu) stabil denge durumunu hedefleyerek bir yaklaşık çözüme
ulaşmaktır.

Bunu nasıl yaparız? Daha önce belirttiğimiz potansiyel enerjinin iki bileşeni
var, ilki sistemin toplam iç gerilim (deformasyon) enerjisi. Gerilim enerji
{\em yoğunluğu} $\overline{U}$ genel olarak bir materyelin stres-gerilim eğrisinin
altındaki alan olarak hesaplanabilir.

\includegraphics[width=10em]{phy_020_strs_04_04.jpg}

Mesela üstteki gibi stres $\sigma_{ij}$ ve gerilim $\epsilon_{ij}$ arasındaki
bir eğriyi düşünelim, bu eğrinin altında kalan alan, yani entegral hesabı 
gerilim enerji yoğunluğunu verir.

$$
\overline{U} = \int_{0}^{\epsilon_{ij}} \sigma_{ij} \ud \epsilon_{ij}
$$

Kolaylaştırıcı bir faktör, bizim bu derste kullanacağımız maddeler lineer
elastik, yani stres-gerilim eğrisi alttaki gibi,

\includegraphics[width=10em]{phy_020_strs_04_05.jpg}

Bu durumda ``eğri'' yani çizgi altındaki alan basit bir üçgen hesabı,

$$
\overline{U} = \frac{1}{2} \sigma_{ij} \epsilon_{ij}
$$

Fakat bu sadece tek bir boyutu halletti, mesela üstteki örnek $e_1$ yönündeki
bir esnemeyi temsil ediyor olabilirdi, fakat 3 boyutlu ortamda elimizde daha
fazla bileşen olduğunu biliyoruz, $\sigma_{11}$ haricinde $\sigma_{22}$ var,
$\sigma_{33}$ var, $\sigma_{12}$, $\sigma_{23}$, vs.. Tüm stres/gerilim
eşleri için [5, Ders 16],

$$
\overline{U} = \frac{1}{2} \sum_{i,j=1}^{3} \sigma_{ij} \epsilon_{ij}
$$

Toplamı açarsak,

$$
= \frac{1}{2} (\sigma_{11}\epsilon_{11} + \sigma_{22}\epsilon_{22}  +
\sigma_{33}\epsilon_{33} + \sigma_{12}\epsilon_{12} + \sigma_{23}\epsilon_{23} +
\sigma_{13}\epsilon_{13} + \sigma_{33}\epsilon_{33} + \sigma_{23}\epsilon_{23} +
\sigma_{32}\epsilon_{32} )
$$

Son ifadeyi daha basitleştirmek mümkün, $\sigma$ ve $\epsilon$'un simetrik
olduğunu unutmayalım,

$$
\overline{U} = \frac{1}{2} (\sigma_{11}\epsilon_{11} + \sigma_{22}\epsilon_{22}  +
\sigma_{33}\epsilon_{33} + 2 \sigma_{12}\epsilon_{12} +
2 \sigma_{13}\epsilon_{13} + 2 \sigma_{23}\epsilon_{23} )
$$

Üsttekileri mühendislik gerilimi $\gamma_{ij} = 2 \epsilon_{ij}$ ile temsil
etmek mümkün,

$$
\overline{U} = \frac{1}{2} (\sigma_{11}\epsilon_{11} + \sigma_{22}\epsilon_{22}  +
\sigma_{33}\epsilon_{33} + \sigma_{12}\gamma_{12} +
\sigma_{13}\gamma_{13} + \sigma_{23}\gamma_{23} )
$$
 
Eksenel Yükleme (Axial Loading)

Euler-Bernoulli kiriş formülasyonu sadece bükülmenin sebep olduğu deformasyonu
hesaba kattı, bunu yaparken nötr eksen üzerindeki eksenel deformasyonu yok saydı
[3, 8.4]. Euler-Bernoulli modelini eksenel yatay deformasyonu hesaba katacak
şekilde genişletmek mümkündür, fakat, belki de bu iyi haber, ufak deformasyon
önkabulü sayesinde eksenel yük ve bükülme deformasyonları birbirinden
bağlantısız (uncoupled) hale gelir, eksenel yük sadece eksenel deformasyonu,
yatay yük sadece yatay deformasyonu etkiler.

\includegraphics[width=25em]{phy_020_strs_04_06.jpg}

Diğer faraziyeler Euler-Bernoulli modeline benzer, düzlem bölümler düzlem kalır,
Poisson oranı etkileri yok sayılır, ve yatay yer değişimi $u_1$ pürüzsüz bir
fonksiyondur. Farklı bir resmi [6, Ders 15] eklersek,

\includegraphics[width=20em]{phy_020_strs_04_08.jpg}

Bu faraziyelerle modeli oluşturalım; üstteki resme bakarsak $u_1 = u_1(X_1)$, ve
pozisyon vektör fonksiyonu olarak,

$$
x = \left[\begin{array}{c}
X_1 + u_1 \\ X_2 \\ X_3
\end{array}\right]
$$

Bu durumda yer değişim fonksiyonu

$$
u = x - X = \left[\begin{array}{c}
u_1 \\ 0 \\ 0
\end{array}\right]
$$

Hatırlarsak yaklaşık olarak gerilim tensörü

$$
\epsilon = \frac{1}{2} (\nabla u + \nabla u^T )
$$

Gradyanlar ile hesabı yaparsak sadece $\epsilon_{11}$'in sıfır olmadığını
görüyoruz,

$$
\epsilon = \left[\begin{array}{ccc}
\frac{\ud u_1}{\ud X_1} & 0 & 0 \\
0 & 0 & 0 \\
0 & 0 & 0 
\end{array}\right]
\mlabel{1}
$$

Bize gerekli diferansiyel denklemi kuvvet dengelerine bakarak ortaya
çıkartabiliriz. Şimdi kirişin $\ud X_1$ genişliğindeki ufak bir parçasına
odaklanalım,

\includegraphics[width=20em]{phy_020_strs_04_07.jpg}

Bu parçanın sol ve sağındaki kuvvetlere bakarsak üstteki resim ortaya çıkar.

Oklar sola ya da sağa doğru gösterildi çünkü mesela $\sigma_{11}$ ve
$\sigma_{11} + \frac{\partial \sigma_{11}}{\partial X_1} \ud X_1$'in
birbirini dengeleyeceklerini / birbirlerine karşı ortaya çıktıklarını
biliyoruz, sağa doğru olan $p$ zaten dışarıdan uygulanan eksenel kuvvet.
$\frac{\partial \sigma_{11}}{\partial X_1}$ kullanımı $\sigma_{11}$'in
$X_1$'e oranla değişim hesabı için kullanıldı, bu oranı $X_1$'deki
olan değişimle çarpınca (örnekte $\ud X_1$) tabii ki ufak parçanın sağındaki
$\sigma_{11}$ eki ortaya çıkıyor, bunu $\sigma_{11}$'e topluyoruz.

Parçanın sol ve sağındaki alan büyüklüğü benzer şekilde, solda $A$ varsa
$A$'nin $X_1$'e oranlı değişimi çarpı $X_1$ değişimi bize parçanın sağındaki
alan büyüklük ekini veriyor. 

$X_1$ ekseni bazındaki denge denklemi o zaman alttaki gibi olur, stres hesabı
kuvvet bölü birim alan olduğu için kuvveti elde etmek için stres çarpı alan
gerekeceğini hatırlayalım, ayrıca $p$ kuvveti birim $X_1$ bazlı alınıyor,
o zaman $p \ud X_1$ kullanmak gerekir,

$$ \sum F_{X_1} = - \sigma_{11} A +
\left( \sigma_{11} + \frac{\partial \sigma_{11}}{\partial X_1} \ud X_1  \right)
\left( A + \frac{\partial A}{\partial X_1} \ud X_1  \right) + p \ud X_1 = 0
$$

Formülü açarsak

$$
-\sigma_{11} A  + \sigma_{11} A  +
\sigma_{11} \frac{\partial A}{\partial X_1} \ud X_1 +
A \frac{\partial \sigma_{11}}{\partial X_1} \ud X_1 +
\frac{\partial A}{\partial X_1} \frac{\partial \sigma_{11}}{\partial X_1} \ud X_1^2 +
p \ud X_1 = 0
$$

$\sigma_{11} A$ terimleri iptal olur,

$$
\sigma_{11} \frac{\partial A}{\partial X_1} \ud X_1 +
A \frac{\partial \sigma_{11}}{\partial X_1} \ud X_1 +
\frac{\partial A}{\partial X_1} \frac{\partial \sigma_{11}}{\partial X_1} \ud X_1^2 +
p \ud X_1 = 0
$$

$\ud X_1$ dışarı çekilir, ve sağda sıfır olduğu için iptal edilebilir,

$$
\sigma_{11} \frac{\partial A}{\partial X_1}  +
A \frac{\partial \sigma_{11}}{\partial X_1}  +
\frac{\partial A}{\partial X_1} \frac{\partial \sigma_{11}}{\partial X_1} \ud X_1 +
p  = 0
$$

Hala basitleştirme mümkün, dikkat edersek $\ud X_1$ terimini kullandık ve
ona ``çok küçük bir parça'' dedik. Bu parçayı sonsuz küçültürsek, yani
limiti alırsak, ki $\ud X \to 0$, o zaman üstteki formülde üçüncü terim
yokolur,

$$
\sigma_{11} \frac{\partial A}{\partial X_1}  +
A \frac{\partial \sigma_{11}}{\partial X_1}  
p  = 0
$$

Daha kısa bir formüle ulaştık. Fakat bize lazım olan yer değişimi, üstteki
formülde bu yok. Oraya ulaşmaya çalışalım. Dikkat edersek üstteki formülde
ilk iki terim sanki Calculus'ta çarpım kuralının açılmış haline benziyor,
o zaman o kuralı ters yönde işletirsek, yani gruplama amaçlı geriye gidersek,

$$
\frac{\partial }{\partial X_1} (\sigma_{11} A ) + p = 0
$$

Şimdi $\sigma_{11} = E \epsilon_{11}$ formülünü hatırlayalım, bu nereden geldi?
Elimizde bir tekeksenel yük var, o zaman en baz stres-gerilme ilişkisi geçerli
olur, yerine koyarsak,

$$
\frac{\partial }{\partial X_1} (E \epsilon_{11} A ) + p = 0
$$

Peki $\epsilon_{11}$ nedir? Bu büyüklüğü (1)'de gördük, $\epsilon$'un tek sıfır
olmayan öğesi $\epsilon_{11}$ ve orada $\frac{\ud u_1}{\ud X_1}$ değeri
var. Bunu üstteki formüle koyalım,

$$
\frac{\partial }{\partial X_1} \left( E A \frac{\ud u_1}{\ud X_1} \right) + p = 0
$$

Böylece içinde yer değişimi içeren bir formül elde etmiş oldum, $u_1$ yer
değişimidir.

Bu denklemi artık çözüm için kullanabiliriz. Tasarımcı olarak biz Young'in
Genliği $E$'yi biliriz, kirişin herhangi bir noktasındaki satıhsal alan $A$'yi
biliriz, kirişe uygulanan yük $p$'yi biliriz, tüm bunları kullanarak yer değişim
$u_1$'i üstteki formülle bulabiliriz. Tek bilinmeyen $u_1$ çünkü.

Şimdi bu noktada bazı püf noktalar ortaya çıkıyor; çünkü eğer $E,A$ büyüklükleri
$X_1$'in fonksiyonu iseler çözüm daha karmaşık hale gelebilir çünkü üstteki
formülde dış türev $X_1$'e göre. Fakat şimdiye kadar bu derste $X_1$'e bağlı bir
$E$ görmedik, yani Young'in Genliği kirişin her noktasında aynı, özetle sabit.
Sabit ise $E$ diferansiyelin dışına alınabilir. $A$ aynı şekilde. Eğer $A$
değişken ise o zaman Calculus çarpım kuralı uygularız.

O zaman iki senaryo şöyle olabilir, $E$ sabit ama $A$ değil,

$$
E \frac{\partial A}{\partial X_1} \frac{\partial u_1}{\partial X_1} +
EA \frac{\partial^2 u_1}{\partial X_1} + p = 0
$$

Hem $E$ hem $A$ sabit,

$$
E A \frac{\partial^2 u_1}{\partial X_1} + p = 0
$$

Üç Boyutta Eşyönlü (Isotropic) Stres-Gerinim İlişkisi 

Şimdi bir kütleye uygulanan stres sonucu ortaya çıkan gerinimi üç boyut için
formülize edeceğiz [7, sf. 871]. Maddenin lineer elastik ve eşyönlü olduğu farz
edilecek, yani uygulanan bir stres farklı yönlerde etkilere sebep olursa bu etki
her yönde eşit şekilde ortaya çıkacak. Aradığımız formül Hooke Kanunu'nun üç
boyutlu hali, buna bazı kaynaklar (listelenen şartlar için) Genelleştirilmiş
Hooke Kanunu ismi de verebiliyor.

Bir gövdeyi her eksen üzerinden $\sigma_x$, $\sigma_y$, $\sigma_z$ streslerine
tabi tutacağız ve sonuçları inceleyeceğiz. Mesela temel Hooke Kanunu $\sigma = E
\epsilon$'den yola çıkarak $\epsilon_x^x = \sigma_x / E$ diyebiliriz,
$\epsilon_x^x$ büyüklüğündeki üstsimge gerinimin $x$ stresi sebebiyle olduğunu
söylüyor, diğerleri de olacak.

\includegraphics[width=35em]{phy_020_strs_00_10.jpg}

Fakat stres-gerinim ilişkisi sadece tek eksenle kısıtlı değil. Bir eksende stres
uyguladığımızda bunun diğer eksenler üzerinde de etkileri olacaktır.  Çünkü
madde bir yöne uzayıp şekil değiştirir fakat diğer eksenlerde ufalma olacağı
için o eksenlerde eksi yönde gerinim olur.

Mesela üstte ortadaki resmi düşünürsek, $\sigma_y$ stresi uygulandığında $x$
yönünde bir negatif gerinim olur, çünkü madde o eksen bağlamında içe doğru
daralır, şekil değiştirir, bunu formül

$$
\epsilon_x^y =  \frac{- v \sigma_y}{E}
$$

ile gösterebiliriz ki $v$ Poisson oranı. Dikkat $y$ yönündeki stresin etkisi
sadece $E$ sabiti ile değil $v/E$ sabiti ile $y$ eksenine yansıyor. Bu normal
olmalı çünkü bir eksene direk uygulanan stres ve onun aynı eksende yol açtığı
gerinim diğer eksenlerdeki yan etki gibi görülebilecek gerinimler ile aynı
olamaz.

Benzer şekilde $z$ stresinin yol açtığı gerinim

$$
\epsilon_x^z =  \frac{- v \sigma_z}{E}
$$

Tüm bu gerinimleri toplarsak $x$ eksenindeki toplam gerinim elde edilir,

$$
\epsilon_x = \epsilon_x^x + \epsilon_x^y + \epsilon_x^z
$$

$$
= \frac{\sigma_x}{E} - \frac{v \sigma_y}{E} - \frac{v \sigma_z}{E} 
$$

Benzer şekilde $y$ ve $z$ yönündeki gerinimler de elde edilebilir,

$$
\epsilon_y = \frac{\sigma_y}{E} - \frac{v \sigma_x}{E} - \frac{v \sigma_z}{E} 
$$

$$
\epsilon_z = \frac{\sigma_z}{E} - \frac{v \sigma_x}{E} - \frac{v \sigma_y}{E} 
$$

Son üç denklemi birleştirip stresler solda olacak şekilde düzenlersek,

$$
\sigma_x = \frac{E}{(1+v)(1-2v)} [\epsilon_x (1-v) + v \epsilon_y + v \epsilon_z ]
$$

$$
\sigma_y = \frac{E}{(1+v)(1-2v)} [ v \epsilon_x + \epsilon_y (1-v) + v \epsilon_z  ]
$$

$$
\sigma_z = \frac{E}{(1+v)(1-2v)} [v \epsilon_x + v \epsilon_y + \epsilon_z (1-v)  ]
$$

Ayrıca normal stresler için kullanılan Hooke Kanunu $\sigma = E \epsilon$ benzer
bir şekilde kaykılma (shear) stresi ve kaykılma gerinimi için de geçerlidir, ama
sabit $E$ yerine $G$ kullanılır,

$$
\tau = G \gamma
$$

$G$ sabitine Kaykılma Genliği (Shear Modulus) ismi veriliyor. 0 zaman üç boyutta
ortaya çıkabilecek üç kaykılma stresi

$$
\tau_{xy} = G \gamma_{xy} \qquad 
\tau_{yz} = G \gamma_{yz} \qquad 
\tau_{zx} = G \gamma_{zx}
$$

$G$ ile $E$ arasında bir ilişki var, bu formül

$$
G = \frac{E}{2(1+v)}
$$

Bu formülün türetilmesi için [7, sf. 70]'e bakılabilir.

Şimdi üstteki tüm formülleri matris formunda bir araya koyabiliriz,

$$
\left[\begin{array}{c}
\sigma_x \\ \sigma_y \\ \sigma_z \\ \tau_{xy} \\ \tau_{yz} \\ \tau_{zx}
\end{array}\right] =
\frac{E}{(1+v)(1-2v)}
\left[\begin{array}{cccccc}
1-v &  v  &  v  &            0      &               0  &  0  \\
 v  & 1-v &  v  &            0      &               0  &  0  \\
 v  &  v  & 1-v &            0      &               0  &  0  \\
 0  &  0  &  0  & \dfrac{1-2v}{2}   &               0  &  0  \\
 0  &  0  &  0  &            0      &  \dfrac{1-2v}{2} &  0  \\
 0  &  0  &  0  &            0      &               0  &  \dfrac{1-2v}{2} 
\end{array}\right]
\left[\begin{array}{c}
\epsilon_x \\ \epsilon_y \\ \epsilon_z \\ \gamma_{xy} \\ \gamma_{yz} \\ \gamma_{zx}
\end{array}\right]
$$

Matrisin sol alt kısmının simetri sebebiyle sağ üst kısım ile aynı olduğuna
dikkat.

Eğer gerinim değişkenlerini eşitliğin solunda stresleri sağda tutmak istersek,
üstteki matrisin tersini bulmamız lazım [8, sf. 161], sembolik ters alma
işlemini \verb!sympy! ile yapabiliriz,

\begin{minted}[fontsize=\footnotesize]{python}
import sympy as sym
E, v = sym.symbols('E v')
matrix = E/((1+v)*(1-2*v))*sym.Matrix([[1-v,v,v,0,0,0],
                     [v,1-v,v,0,0,0],
		     [v,v,1-v,0,0,0],
		     [0,0,0,(1-2*v)/2,0,0],
                     [0,0,0,0,(1-2*v)/2,0],
		     [0,0,0,0,0,(1-2*v)/2]])
sym.latex(matrix.inv())
\end{minted}


$$
\left[\begin{matrix}\dfrac{1}{E} & - \dfrac{v}{E} & - \dfrac{v}{E} & 0 & 0 & 0\\- \dfrac{v}{E} & \dfrac{1}{E} & - \dfrac{v}{E} & 0 & 0 & 0\\- \dfrac{v}{E} & - \dfrac{v}{E} & \dfrac{1}{E} & 0 & 0 & 0\\0 & 0 & 0 & \dfrac{2 v + 2}{E} & 0 & 0\\0 & 0 & 0 & 0 & \dfrac{2 v + 2}{E} & 0\\0 & 0 & 0 & 0 & 0 & \dfrac{2 v + 2}{E}\end{matrix}\right]
$$

Bir basitleştirme daha yapılabilir, bunu kendimiz görebiliyoruz, $1/E$ dışarı
çekelim, hepsini bir araya koyalım,

$$
\left[\begin{array}{c}
\epsilon_x \\ \epsilon_y \\ \epsilon_z \\ \gamma_{xy} \\ \gamma_{yz} \\ \gamma_{zx}
\end{array}\right] =
\frac{1}{E}
\left[\begin{array}{cccccc}
1 & -v & -v & 0 & 0 & 0 \\
-v & 1 & -v & 0 & 0 & 0 \\
-v & -v & 1 & 0 & 0 & 0 \\
0 & 0 & 0 & 2(1+v) & 0 & 0 \\
0 & 0 & 0 & 0 & 2(1+v) & 0 \\
0 & 0 & 0 & 0 & 0 & 2(1+v)
\end{array}\right]
\left[\begin{array}{c}
\sigma_x \\ \sigma_y \\ \sigma_z \\ \tau_{xy} \\ \tau_{yz} \\ \tau_{zx}
\end{array}\right]
$$

Düzlem Stresi (Plane Stres)

Eğer bir gövde sadece iki boyutta strese tabi tutuluyorsa bu gövdenin ``düzlem
stresi'' durumunda olduğu söylenir [7, sf. 70]. Bu tür stres / gerinimde
$\sigma_z = \tau_{xz} = \tau_{yz} = 0$'dir yani üçüncü boyut $z$ eksenine dönük
hiçbir aksiyon yoktur. Bu durumda Genel Hooke Kanunu alttaki üç denkleme
indirgenebilir,

$$
\epsilon_x = \frac{1}{E} (\sigma_x - v \sigma_y )
$$

$$
\epsilon_y = \frac{1}{E} (\sigma_y - v \sigma_x )
$$

$$
\gamma_{xy} = \frac{1}{G} \tau_{xy}
$$

Üç boyutlu durumda olduğu gibi üstteki formülleri matris formunda
gösterebiliriz,

$$
\left[\begin{array}{c}
\epsilon_{x} \\ \epsilon_{y} \\ \gamma_{xy}
\end{array}\right] =
\frac{1}{E}
\left[\begin{array}{ccc}
1 & -v & 0 \\
-v & 1 & 0 \\
0 & 0 & 2(1+v)
\end{array}\right]
\left[\begin{array}{c}
\sigma_x \\ \sigma_y \\ \tau_{xy}
\end{array}\right]
$$

Yine bir yer değiştirme işlemi yapılabilir, üstteki matrisin tersini alırsak
stresler sola geçer, 

$$
\left[\begin{array}{c}
\sigma_x \\ \sigma_y \\ \tau_{xy}
\end{array}\right] = 
\frac{E}{(1-v)^2}
\left[\begin{array}{ccc}
1 & v & 0 \\ v & 1 & 0 \\ 0 & 0 & \dfrac{(1-v)}{2}
\end{array}\right]
\left[\begin{array}{c}
\epsilon_{x} \\ \epsilon_{y} \\ \gamma_{xy}
\end{array}\right] 
$$

Bazı formülasyonlarda üstteki matrisin çarpan sabitin böleninde $(1+v)$
görülebiliyor, bu durumda $(1-v)^2=(1+v)(1-v)$ olduğunu hatırlayalım, ve ona
göre matrisin tüm öğeleri $(1-v)$ ile bölünmüş olacaktır, farketmez, her iki
form da aynı sonucu verir.

Elastiklik Denge Denklemleri

Gerinme (yer değişim) ve stres arasındaki ilişki gösterildi, şimdi tüm
kuvvetlerin arasındaki denge denklemlerine bakalım. Eğer dışarıdan her eksen
üzerinde, X, Y, Z kuvvetleri uygulansa, ya da uygulanmasa bile, mevcut direk
stresler $\sigma$ ve yüzeylere paralel giden kaykılma stresleri $\tau$
arasındaki denge ilişkisi ne olurdu?

Önce idealize edilmiş her kenarı ufak $\ud x$, $\ud y$, $\ud z$ boyutlarında
olan bir küp düşünelim [9, sf. 11], küpün kenarlarına $X,Y,Z$ kuvvetleri
uygulanıyor.

\includegraphics[width=10em]{equilibrium_stress2.jpg}

Bu kuvvetlerin ortaya çıkardığı direk ve kaykılma stresleri alttaki gibi
gösterilebilir.

\includegraphics[width=30em]{equilibrium_stress1.jpg}

Küpün kordinat eksenlerine paralel olan yüzü nötr diğerleri ``artı yüzleri''
olarak simgelendirildi, mesela üst sağdaki resimde okuyucuya yakın olan yüzden
dışarı doğru çıkan stres $\sigma_x^+$ sembolü, aynı yüzdeki kaykılma stresleri
benzer şekilde artı işaretini alıyor, $x$ eksenine doğru / dik oldukları için
ilk altsembolleri $x$, paralel ilerledikleri eksen ikinci alt sembol, mesela sağ
yönünü gösteren kaykılma stresi $\tau_{xy}^+$. Artı yüzün tam karşısındaki direk
stres $\sigma_x$, onda artı işareti yok.

Eğer bir denge formülü bulmak istiyorsak her yüz için bu eşitlikleri ayrı ayrı
kurabiliriz. Mesela $x$ ekseni ile başlayalım, iki stresten bahsettik zaten,
$\sigma_x^+$ ve $\sigma_x$. Fakat bu yönde yani $x$ ekseni boyunca etki eden
stresler sadece onlar değil, bu yönde olan kaykılma stresleri de var. İşaret
edilen yön ikinci altsembol demiştik, orada $x$ diyen tüm kaykılma streslerini
kolayca bulabiliriz, $\tau_{yx}$, $\tau_{zx}$, $\tau_{yx}^+$, $\tau_{zx}^+$.

Denge formülünden önce artı yüzleri hakkında bir ek bilgi daha verelim, bu
yüzlerin stresini karşısındaki stresi baz alarak formülize edebiliriz,
yani $\sigma_x^+$ formülü $\sigma_x$ bazlı gösterilebilir, bunu kısmı türev
ile yaparız, 

$$
\sigma_x^+ = \sigma_x + \frac{\partial \sigma_x}{\partial x} \ud x
$$

Türev bağlamında üstteki akla yatkın olmalı, $\partial \sigma_x / \partial x$
bir değişim oranınıdır, onu ufak değişim miktarı $\ud x$ ile çarpıp
$\sigma_x$'e eklersek $\sigma_x^+$ elde edebiliriz. Benzer mantığı tüm artı
yüzlerdeki stresler için kullanabiliriz.

Bir ek konu daha, stres daha önce belirtiğimiz gibi kuvvet / alan hesabıdır,
fakat denge formülü kuvvetler üzerinden yapılacak o zaman bildiğimiz, bulduğumuz
her stres büyüklüğünü onun etki ettiği alan ile çarpmamız gerekir, ki böylece
kuvvet elde edelim ve bu kuvvetleri toplayarak sıfır eşitleyip denge formülünü
bulalım. Yine $x$ örneği, $\sigma_x$'in etki ettiği alan $\ud y \ud z$ alanıdır.

Devam edelim, $x$ ile başlayalım, o yöndeki denge için $x$ yönündeki tüm
kuvvetleri toplamak gerekir, artı yüzdeki kuvveti pozitif yapacağız (bunu
hatırlaması kolay), tersini gösteren kuvvet ise negatif olacak, $X$ gövdeye
uygulanan birim hacimdeki dış kuvvettir, o zaman tüm $x$ toplamı

$$
  \sigma_x^+ \ud y \ud z - \sigma_x \ud y \ud z +
  \tau_{yx}^+ \ud x \ud z - \tau_{yx} \ud x \ud z +
  \tau_{zx}^+ \ud x \ud y - 
$$
$$
  \tau_{zx} \ud x \ud y +   X \ud x \ud y \ud z = 0
$$

Artı yüzlerdeki stresleri diğer yüz bağlamında temsil edebiliriz demiştik, bunu
yapalım,

$$
(\sigma_x  + \dfrac{\partial \sigma_x}{\partial x} \ud x )\ud y \ud z - \sigma_x \ud y \ud z +
(\tau_{yx}  + \dfrac{\partial \tau_{yx}}{\partial y} \ud y ) \ud x \ud z -
\tau_{yx} \ud x \ud z +
$$
$$
(\tau_{zx}  + \dfrac{\partial \tau_{zx}}{\partial z} \ud z ) \ud x \ud y - \tau_{zx} \ud x \ud y +
X \ud x \ud y \ud z = 0
$$

Basitleştirip her şeyi $\ud x \ud y \ud z$ ile bölersek,

$$
\frac{\partial \sigma_x}{\partial x} + 
\frac{\partial \tau_{yx}}{\partial y} + 
\frac{\partial \tau_{zx}}{\partial z} + X = 0
$$

Benzer hesabı $y$, $z$ icin yaparsak,

$$
\frac{\partial \tau_{xy}}{\partial x} + 
\frac{\partial \sigma_y}{\partial y} + 
\frac{\partial \tau_{zy}}{\partial z} + Y = 0
$$

$$
\frac{\partial \tau_{xz}}{\partial x} + 
\frac{\partial \tau_{yz}}{\partial y} + 
\frac{\partial \sigma_z}{\partial z} + Z = 0
$$

İki Boyutta Elastiklik Denklemleri

İki boyut için üstteki üç denklemin $z$ yönünde hiçbir stres olmayan halini
düşünebiliriz, üçüncü denklem tamamen yokolur, kalanlardan

$$
\frac{\partial \sigma_x}{\partial x} + 
\frac{\partial \tau_{yx}}{\partial y} + 
\cancel{\frac{\partial \tau_{zx}}{\partial z}} + X = 0
$$

$$
\frac{\partial \tau_{xy}}{\partial x} + 
\frac{\partial \sigma_y}{\partial y} + 
\cancel{\frac{\partial \tau_{zy}}{\partial z}} + Y = 0
$$

Yani [10, sf. 331]

$$
\frac{\partial \sigma_x}{\partial x} + 
\frac{\partial \tau_{yx}}{\partial y} + X = 0
$$

$$
\frac{\partial \tau_{xy}}{\partial x} + 
\frac{\partial \sigma_y}{\partial y} + Y = 0
$$


Kaynaklar

[1] Kim, {\em Introduction to Non-linear Finite Element Analysis}

[2] Petitt, {\em Intro to the Finite Element Method}, University of Alberta,
    \url{https://www.youtube.com/watch?v=2iUnfPRk6Ro&list=PLLSzlda_AXa3yQEJAb5JcmsVDy9i9K_fi}
    
[3] Adeeb, {\em Introduction to Solid Mechanics, Online Book},
    \url{https://engcourses-uofa.ca/books/introduction-to-solid-mechanics/}

[4] Crandall, An Introduction to the Mechanics of Solids

[5] Petitt, {\em Intro to the Continuum Mechanics}, University of Alberta,
    \url{https://www.youtube.com/playlist?list=PLLSzlda_AXa3N5jaDART7kimBlYz1dFnX}

[6] Petitt, {\em Intro to the Continuum Mechanics}, University of Alberta,
    \url{https://www.youtube.com/playlist?list=PLLSzlda_AXa3N5jaDART7kimBlYz1dFnX}

[6] Logan, {\em A First Course in the FEM, 5th Ed}

[7] Craig, {\em Mechanics of Materials, Third Edition}

[8] Khennane, {\em Introduction to Finite Element Analysis using Matlab and Abaqus}

[9] Bhavikatti, {\em Finite Element Analysis}    
    
[10] Pepper, {\em The Finite Element Method}

\end{document}
