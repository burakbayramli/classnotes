\documentclass[12pt,fleqn]{article}\usepackage{../../common}
\begin{document}
Materyel Mekaniği - 11

Üç boyutta herhangi bir yönde olabilecek bir kiriş öğesinin direngenlik
matrisini bu bölümde geliştirelim [2, sf. 281]. Bunu yapabilmek için daha önce
gördüğümüz eksenel, iki boyutlu kiriş, burulma direngenlik matrislerini
birleştireceğiz.

İki boyuttaki kirişin mekaniği bükülme alttaki gibiydi [1],

\includegraphics[width=25em]{phy_020_strs_11_02.jpg}

\includegraphics[width=25em]{phy_020_strs_11_03.jpg}

Şimdi işaretler alttaki gibi olacak,

\includegraphics[width=25em]{phy_020_strs_11_01.jpg}

İki boyutlu kiriş matrisini [1] iki kez kullanacagiz, ilki $x-z$ düzlemi icin,
ikincisi $x-y$ düzlemi için,

$x-z$ düzlemi

$$
\frac{EI_y}{L^4}
\left[\begin{array}{cccc}
12L & -6L  & -12L & -6L^2 \\
    & 4L^3 & 6L^2 & 2L^3  \\
    &      & 12L  & 6L^2  \\
    &      &      & 4L^3
\end{array}\right]
\mlabel{1}
$$

$x-y$ düzlemi

$$
\frac{EI_y}{L^4}
\left[\begin{array}{cccc}
12L & 6L  & -12L & 6L^2 \\
    & 4L^3 & -6L^2 & 2L^3  \\
    &      & 12L  & -6L^2  \\
    &      &      & 4L^3
\end{array}\right]
\mlabel{2}
$$

Dikkat edersek $x-y$ düzleminin matrisi [1] matrisi ile aynı. $x-z$ düzlemi
matrisinin bazı işaretleri farklı, bunun sebebi düzlemdeki bükülmenin sağ el
kuralına göre farklı yönleri gösterebilmesi. Mesela [1]'deki 

$$
m_1 = -m = -EI \frac{\ud^2 v(0)}{\ud x^2} =
\frac{EI}{L^3} ( 6L v_1 + 4L^2 \phi_1 - 6L v_2 + 2 L^2 \phi_2 )
$$

formülünü hatırlarsak, o formülde bir uçta $m$ diğer üçta $-m$ vardı, fakat
üstteki figürde $x-z$ düzlemindeki iki uçtaki $m$ değerleri aynı işarettedir,
sağ el kuralını düşünürsek $x-z$'deki bükülme 1 noktasında kağıttan bize doğru
gösteriyor, 2 noktasında aynı şekilde. O zaman üstteki formüldeki işaretler
değişir,

$$
\Rightarrow \frac{EI}{L^3} ( -6L v_1 - 4L^2 \phi_1 + 6L v_2 - 2 L^2 \phi_2 )
$$

Bir değişim daha açılarda, üstteki formülde $\phi_1,\phi_2$ olan acılar,
üç boyuttaki üstteki şekilde bunlar $\phi_{1y}$ ve $\phi_{2y}$. Bu açılar
iki boyutlu durumun aksine artı yönde tam ters işaretli, tersi yönde bir
yer değişim $v_1,v_2$'ye sebep oluyorlarlar, bu yüzden o işaretler de
tersine dönüyor, notasyonu da düzeltince,

$$
\Rightarrow \frac{EI}{L^3} ( -6L v_1 + 4L^2 \phi_{1y} + 6L v_2 + 2 L^2 \phi_{2y} )
$$

Böylece matrisin ikinci satırında $-6,+4,+6,+2$ katsayılarını elde ediyoruz.
Bu işaretlerin (1) matrisinin ikinci satırıyla aynı olduğunu görebiliriz,
diğer satırlar benzer şekilde değiştiriliyorlar.

Üstdüşüm (Superposition)

Artık bahsedilen matrisleri birleştirebiliriz. Bu üstdüşümü \verb!sympy!  ile
otomatik olarak yapacağız, daha önce sayısal değerler için kullandığımız
\verb!expand_dataframe! kodu yine kullanılabilecek, çünkü kod bir \verb!pandas!
Dataframe'i baz alıyor, bu Dataframe içinde herhangi bir obje depolamak mümkün,
oraya sayılar yerine \verb!sympy! sembolik matematik objeleri koyabiliriz. Bir
bonus ta elde ediyoruz, toplama işlemi \verb!sympy! tipleri için önceden
tanımlıdır, yani eğer üstdüşüm sırasında çakışma olursa, sembolik objeler
birbiriyle toplanacaktır!

\begin{minted}[fontsize=\footnotesize]{python}
from sympy import symbols, pprint, latex
from sympy.matrices import Matrix
import pandas as pd, pickle
pd.set_option('display.max_columns', None)

all_vars = ['u1','v1','w1','phi1x','phi1y','phi1z',\
            'u2','v2','w2','phi2x','phi2y','phi2z']
A,G,J,E,L,Iy,Iz = symbols("A,G,J,E,L,Iy,Iz")
\end{minted}

Önce üstte bahsedilen iki düzlemi alalım,

\begin{minted}[fontsize=\footnotesize]{python}
# x-z
vars1 = ['w1','phi1y','w2','phi2y']
M1 = pd.DataFrame([[12*L, -6*L**2,-12*L,-6*L**2],
                  [-6*L**2,4*L**3,6*L**2,2*L**3],
                  [-12*L,6*L**2,12*L,6*L**2],
                  [-6*L**2,2*L**3,6*L**2,4*L**3]],index=vars1)
M1.columns = vars1
M1 = M1 * (E*Iy/L**4 )
# x-y
vars2 = ['v1','phi1z','v2','phi2z']
M2 = pd.DataFrame([[12*L, 6*L**2,-12*L,6*L**2],
                  [6*L**2,4*L**3,-6*L**2,2*L**3],
                  [-12*L,-6*L**2,12*L,-6*L**2],
                  [6*L**2,2*L**3,-6*L**2,4*L**3]],index=vars2)
M2.columns = vars2
M2 = M2 * (E*Iz/L**4 )
\end{minted}

Şimdi [1]'deki eksenel yükleri tanımlayan makaskiriş direngenlik matrisini
alalım,

\begin{minted}[fontsize=\footnotesize]{python}
# Eksenel Yuk
vars3 = ['u1','u2']
M3 = pd.DataFrame([[1,-1],[-1,1]],index=vars3)
M3.columns = vars3
M3 = M3 * (A*E/L)
\end{minted}

[4]'te tanımladığımız burulma mekaniğinin matrisini belirtelim,

\begin{minted}[fontsize=\footnotesize]{python}
# Burulma (Torsion)
vars4 = ['phi1x','phi2x']
M4 = pd.DataFrame([[1,-1],[-1,1]],index=vars3)
M4.columns = vars4
M4 = M4 * (G*J/L)
\end{minted}

Hepsini üstdüşüm ile birleştirelim,

\begin{minted}[fontsize=\footnotesize]{python}
import sys; sys.path.append('../phy_020_strs_08')
import dfutil

M1f = dfutil.expand_dataframe(M1,all_vars)
M2f = dfutil.expand_dataframe(M2,all_vars)
M3f = dfutil.expand_dataframe(M3,all_vars)
M4f = dfutil.expand_dataframe(M4,all_vars)
Mall = Matrix(M1f + M2f + M3f + M4f)
pickle.dump(Mall,open("Mall.pkl","wb"))
\end{minted}

Sonuç matrisini bir dosyaya yazalım, 

\begin{minted}[fontsize=\footnotesize]{python}
print (latex(Mall)[:100],'...')
\end{minted}

\begin{verbatim}
\left[\begin{array}{cccccccccccc}\frac{A E}{L} & 0 & 0 & 0 & 0 & 0 & - \frac{A E}{L} & 0 ...
\end{verbatim}

$$
\left[\begin{array}{cccccccccccc}\frac{A E}{L} & 0 & 0 & 0 & 0 & 0 & - \frac{A E}{L} & 0 & 0 & 0 & 0 & 0\\0 & \frac{12 E Iz}{L^{3}} & 0 & 0 & 0 & \frac{6 E Iz}{L^{2}} & 0 & - \frac{12 E Iz}{L^{3}} & 0 & 0 & 0 & \frac{6 E Iz}{L^{2}}\\0 & 0 & \frac{12 E Iy}{L^{3}} & 0 & - \frac{6 E Iy}{L^{2}} & 0 & 0 & 0 & - \frac{12 E Iy}{L^{3}} & 0 & - \frac{6 E Iy}{L^{2}} & 0\\0 & 0 & 0 & \frac{G J}{L} & 0 & 0 & 0 & 0 & 0 & - \frac{G J}{L} & 0 & 0\\0 & 0 & - \frac{6 E Iy}{L^{2}} & 0 & \frac{4 E Iy}{L} & 0 & 0 & 0 & \frac{6 E Iy}{L^{2}} & 0 & \frac{2 E Iy}{L} & 0\\0 & \frac{6 E Iz}{L^{2}} & 0 & 0 & 0 & \frac{4 E Iz}{L} & 0 & - \frac{6 E Iz}{L^{2}} & 0 & 0 & 0 & \frac{2 E Iz}{L}\\- \frac{A E}{L} & 0 & 0 & 0 & 0 & 0 & \frac{A E}{L} & 0 & 0 & 0 & 0 & 0\\0 & - \frac{12 E Iz}{L^{3}} & 0 & 0 & 0 & - \frac{6 E Iz}{L^{2}} & 0 & \frac{12 E Iz}{L^{3}} & 0 & 0 & 0 & - \frac{6 E Iz}{L^{2}}\\0 & 0 & - \frac{12 E Iy}{L^{3}} & 0 & \frac{6 E Iy}{L^{2}} & 0 & 0 & 0 & \frac{12 E Iy}{L^{3}} & 0 & \frac{6 E Iy}{L^{2}} & 0\\0 & 0 & 0 & - \frac{G J}{L} & 0 & 0 & 0 & 0 & 0 & \frac{G J}{L} & 0 & 0\\0 & 0 & - \frac{6 E Iy}{L^{2}} & 0 & \frac{2 E Iy}{L} & 0 & 0 & 0 & \frac{6 E Iy}{L^{2}} & 0 & \frac{4 E Iy}{L} & 0\\0 & \frac{6 E Iz}{L^{2}} & 0 & 0 & 0 & \frac{2 E Iz}{L} & 0 & - \frac{6 E Iz}{L^{2}} & 0 & 0 & 0 & \frac{4 E Iz}{L}\end{array}\right]
$$

Bu matrisin [2, sf. 282]'deki matris ile aynı olduğunu göreceğiz.

Döndürme işlemine geldik, döndürme için bir $T$ matrisi gerekiyor ki daha önce
olduğu gibi bir $T^T k' T$ işlemini yapabilelim. Bu $T$ matrisi içinde dört tane
3x3 boyutunda $\lambda$ matris bloğu olacak, $T$ köşegeni üzerinde
tekrarlanacaklar, ki böylece her değişken bloğu çarpılabilsin /
döndürülebilsin. Bu bloklar $(u_1,v_1,w_1)$, $(\phi_{1x},\phi_{1y},\phi_{1z})$,
$(u_2,v_2,w_2)$, $(\phi_{2x},\phi_{2y},\phi_{2z})$ [2, sf. 282].

$$
T = \left[\begin{array}{cccc}
[\lambda] &  & & \\
 & [\lambda] & & \\
 & & [\lambda] & \\
 & & & [\lambda]
\end{array}\right]
$$

Her $\lambda$ matrisi ``yön kosinüsleri'' denen değerleri içeriyor olacak, bu
kavramın detayları için bkz [6, sf. 73], [5, sf. 224], özet olarak tarif etmek
gerekirse bir vektörün ya da yeni kordinat sisteminin yeni ekseninin, referans
bir diğer ekseni ile olan açısının kosinüsü olarak görebiliriz.

\includegraphics[width=20em]{phy_020_strs_11_04.jpg}

Eğer eksen değişimi, transformasyonunu hesaplıyorsak $y$ ekseninin yeni $x'$
ekseni ile oluşturduğu açı $\cos_{yx'}$ olarak gösterilebilir, ya da kısaca
$C_{yx'}$, tüm bu kombinasyonlar için elde edilen açılar bir matris içinde
$\lambda$'da yer alır,

$$
\lambda = \left[\begin{array}{ccc}
C_{xx'} & C_{yx'} & C_{zx'} \\
C_{xy'} & C_{yy'} & C_{zy'} \\
C_{xz'} & C_{yz'} & C_{zz'} 
\end{array}\right]
$$

Sonlu Öğeler (Finite Element) bu hesap şu şekilde uygulanır, FEM problemlerinde
genellikle yapılan bir kiriş parçasını transform edilmiş $x'$ ekseni kabul
edilir, ve geri kalan $\lambda$ değerleri buna göre doldurulur. Bu durumda
kirişin uç ve baş noktasını kullanarak $x'$ ile oluşan açıları şöyle
bulabiliriz,

$$
\cos_{xx'} = \frac{x_2 - x_1}{L} = l
$$

$$
\cos_{yx'} = \frac{y_2 - y_1}{L} = m
$$

$$
\cos_{zx'} = \frac{z_2 - z_1}{L} = n
$$

ki kirişin son noktası $x_2,y_2,z_2$, baş noktası $x_1,y_1,z_1$, uzunluk $L$.

Kiriş bazlı yeni eksen oluştururken bir kez $x'$ elde ettikten sonra diğer
eksenleri ona göre bulabiliriz, diğer herhangi bir vektörle oluşturulan düzleme
dik olan yeni bir vektörü $y'$ için kullanabilirdik mesela, genellikle burada
yapılan $x'$ ile $z$ ekseninin çapraz çarpımını almaktır, ki bilindiği gibi iki
vektöre dikgen üçüncü bir vektör çapraz çarpım ile hesaplanır, böylece bir $y'$
elde edilir, sonra $x'$ ile $y'$ ekseninin bir çapraz çarpımı daha alınarak $z'$
bulunabilir.  Bu hesapla elde edilen $x',y',z'$ bir kordinat ekseninin sahip
olması gerekli tüm koşulları yerine getiriyor, eksenler birbiriyle dikgen.

Biraz önceki semboller üzerinde görelim,

$$
z \times x' = y' = \frac{1}{D}
\left[\begin{array}{ccc}
i & j & k \\ 0 & 0 & 1 \\ l & m & n
\end{array}\right]
$$

$$
y' = - \frac{m}{D} i + \frac{l}{D} j
$$

$$
D = (l^2 + m^2)^{1/2}
$$





















Problem 5.8

\begin{minted}[fontsize=\footnotesize]{python}
from sympy.matrices import Matrix
from sympy import symbols
import pickle

A,G,J,E,L,Iy,Iz = symbols("A,G,J,E,L,Iy,Iz")
Mall = pickle.load(open('Mall.pkl','rb'))

D = {E: 30000, G: 10000, J:50, Iy:100, Iz:100, A:10, L:100}
M = Mall.subs(D)
print (latex(M)[:100], '..')
\end{minted}

\begin{verbatim}
\left[\begin{array}{cccccccccccc}3000 & 0 & 0 & 0 & 0 & 0 & -3000 & 0 & 0 & 0 & 0  ..
\end{verbatim}

$$
{\scriptsize
\left[\begin{array}{cccccccccccc}3000 & 0 & 0 & 0 & 0 & 0 & -3000 & 0 & 0 & 0 & 0 & 0\\0 & 36 & 0 & 0 & 0 & 1800 & 0 & -36 & 0 & 0 & 0 & 1800\\0 & 0 & 36 & 0 & -1800 & 0 & 0 & 0 & -36 & 0 & -1800 & 0\\0 & 0 & 0 & 5000 & 0 & 0 & 0 & 0 & 0 & -5000 & 0 & 0\\0 & 0 & -1800 & 0 & 120000 & 0 & 0 & 0 & 1800 & 0 & 60000 & 0\\0 & 1800 & 0 & 0 & 0 & 120000 & 0 & -1800 & 0 & 0 & 0 & 60000\\-3000 & 0 & 0 & 0 & 0 & 0 & 3000 & 0 & 0 & 0 & 0 & 0\\0 & -36 & 0 & 0 & 0 & -1800 & 0 & 36 & 0 & 0 & 0 & -1800\\0 & 0 & -36 & 0 & 1800 & 0 & 0 & 0 & 36 & 0 & 1800 & 0\\0 & 0 & 0 & -5000 & 0 & 0 & 0 & 0 & 0 & 5000 & 0 & 0\\0 & 0 & -1800 & 0 & 60000 & 0 & 0 & 0 & 1800 & 0 & 120000 & 0\\0 & 1800 & 0 & 0 & 0 & 60000 & 0 & -1800 & 0 & 0 & 0 & 120000\end{array}\right]
}
$$












Kaynaklar

[1] Bayramlı, {\em Fizik, Materyel Mekaniği 7}

[2] Logan, {\em A First Course in the Finite Element Method, 6th Ed}

[3] Witt {\em Concepts and Apps of FEM}

[4] Bayramlı, {\em Fizik, Materyel Mekaniği 9}

[5] Dunn, {\em 3D Math Primer for Graphics and Game Development}

[6] Stasa, {\em Applied Finite Element Analysis for Engineers}

\end{document}

