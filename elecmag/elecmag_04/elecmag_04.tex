\documentclass[12pt,fleqn]{article}\usepackage{../../common}
\begin{document}
Ders 4

�nceki derste y�k�n muhafaza edildi�inden bahsettik, evrenin toplam y�k�
de�i�tirilemez, Bu demektir ki evrendeki t�m pozitif y�klerden t�m negatif
y�kleri ��kart�rsam, o say� her neyse, ne oldu�unu bilmiyoruz, o say�
de�i�emez. Y�k yarat�p yoketmek m�mk�n, e�er bunu �iftler �zerinden yaparsak.

Bugun yal�tkanlar ve iletkenler hakkinda konusacagiz. Yalitkanlarda elektronlar
atomlarina yakin dururlar, iletkenlerde yukler sanki bir sivinin aktigi gibi
akarlar, hatta formel olarak onlarin bir sivi oldugunu da iddia edebiliriz,
bunun anlamini daha detayli olarak gorecegiz. Ayrica her iletkende, denge
durumunda ki ``denge'' sozuyle ne demek istedigimi de aciklayacagim, toplam
elektrik alan sifir olmalidir. 





\url{https://www.youtube.com/watch?v=GJiUoP5ldgA}

\end{document}










