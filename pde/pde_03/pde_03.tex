\documentclass[12pt,fleqn]{article}\usepackage{../../common}
\begin{document}
Ders 3

Dalga Denklemi Başlangıç Değer Problemi (Initial Value Problem) 

Başlangıç değer problemi 

$$ u_{tt} = c^2u_{xx} $$

$$ -\infty < x < \infty $$

ve başlangıç şartları 

$$ u(x,0) = \phi(x) $$

$$ u_t(x,0) = \psi(x) $$

ki $\phi$ ve $\psi$, $x$'in herhangi bir fonksiyonu. 

Her $\phi$ ve $\psi$ için bu problemde tek bir çözüm vardır, yani tek bir
$u$ elde edilir. Mesela $\phi(x) = \sin(x)$ ve $\psi(x) = 0$ ise, o zaman
$u(x,t) = \sin x \cos ct$. 

Kontrol edelim. Çözüm şöyleydi:

$$ u(x,t) = f(x+ct) + g(x-ct)  
\mlabel{5}
$$

$t=0$ dersek 

$$ u(x,0)  = f(x) + g(x) $$

$u(x,0) = \phi(x)$ demiştik

$$ \phi(x) = f(x) + g(x) 
\mlabel{1} 
$$

Zincirleme Kanunu kullanarak $t$'ye göre türev alalım ve $t=0$ diyelim

$$ u_t(x,0) = cf' - cg' $$

$u_t(x,0) = \psi(x)$ demiştik

$$ \psi(x) = cf'(x) - cg'(x) 
\mlabel{2} $$

Formül (1) ve (2)'yi bir denklem sistemi olarak görebiliriz, ve aynı anda
çözmeye uğraşabiliriz. (1)'in türevini alalım, ve (2)'yi $c$'ye bölelim

$$ \phi' = f' + g' $$

$$ \frac{1}{c}\psi = f' - g' $$

Üstteki iki denklemi toplarsak

$$ f' = \frac{1}{2} \bigg( \phi' + \frac{\psi}{c}  \bigg) $$

Çıkarırsak

$$ g' = \frac{1}{2}  \bigg( \phi' - \frac{\psi}{c}  \bigg) $$


$f'$'in entegralini alalım. Ek: $x$ yerine $s$ kullanacağız şimdi,

$$ \int f' = \int \frac{1}{2} \bigg( \phi' + \frac{\psi}{c}  \bigg) $$

$$ f = \frac{1}{2}\phi(s) + \frac{1}{2}\int_0^s \frac{\psi(x)}{c} + A$$

$$ f = \frac{1}{2}\phi(s) + \frac{1}{2c} \int_0^s \psi(x) + A 
\mlabel{3}
$$

$g'$'nin entegrali

$$ g = \frac{1}{2}\phi(s) - \frac{1}{2c} \int_0^s \psi(x) + B
\mlabel{4}
$$


ki $A$ ve $B$ sabitler. $A=0,B=0$ çünkü (3) ve (4) toplanınca içinde $A,B$
olmayan (1) elde edilir, o zaman $A=0,B=0$ olmalıdır. Böylece genel formül
(5) için gereken $f,g$ fonksiyonlarını elde etmiş oluyoruz. Genel formülde
biraz önce bulduğumuz $f,g$ yerine geçirirsek, ve $f$ için $s = x+ct$, $g$
için $s = x-ct$ kullanırsak,

$$ u(x,t) = 
\frac{1}{2}\phi(x+ct) + 
\frac{1}{2c} \int_0 ^{x+ct} \psi + 
\frac{1}{2}\phi(x-ct) -
\frac{1}{2c} \int_0 ^{x+ct} \psi 
 $$

Basitleştirirsek 

$$ = \frac{1}{2}[\phi(x+ct) + \phi(x-ct)] + 
\frac{1}{2c} \int_{x-ct} ^{x+ct} \psi (s) \ud s
 $$

Bu çözüm matematikçi d'Alembert'in 1746'da bulduğu başlangıç değer
probleminin çözümdür. 

Örnek

$$ \phi(x) = 0, \psi(x) = \sin(x)\cos(ct) $$

Üstteki formüllerden 2.'yi 1.'den çıkartalım, 

$$ \sin(x+ct) - \sin(x-ct) = 2\cos(x)\sin(ct) $$

$u(x,t)$ içinde yerine koyalım

$$ u(x,t) = \frac{1}{c}\cos(x)\sin(ct) $$

Çözüm bu. Kontrol edersek, 

$$ u_{tt} = -c \cos(x)\sin(ct) $$

$$ u_{xx} = -(1/c)\cos(x)\sin(ct) $$

böylece 

$$ u_{tt} = c^2 u_{xx} $$

Başlangıç şartını kontrol etmek daha da kolay, ilk şart

$$ u(x,0) = \frac{1}{c}\cos(x)\sin(c \cdot 0)  = 0$$

İkinci şart

$$ u_t(x,t) = \frac{1}{c}\cos(x) c \cos(ct) $$

$$ u_t(x,0) = \frac{1}{c}\cos(x) c $$

$$  = \cos(x) $$

Yayılım (Diffusion) Denklemi

Bu bölümde çözmeye uğraşacağımız problem

$$  u_t = k u_{xx}  $$

$$ u(x,0) = \phi(x) $$

öyle ki 

$$ -\infty < x < \infty, 0 < t < \infty $$

Bu denklemi özel (particular) bir $\phi$ için çözeceğiz, ve sonra genel
çözümü bu özel çözümden elde edeceğiz. Bunu yaparken yayılım denklemin 5
tane değişmezlik (invariance) özelliğini kullanacağız.

a) Bir sabit $y$ için $u(x,y)$ çözümünü $u(x-y,t)$ olarak tercüme edilir /
taşınırsa (translate), ortaya çıkan denklem yeni bir çözümdür.

b) Çözümün her türevi, mesela $u_x$, $u_t$, $u_{xx}$ gibi, yeni bir
çözümdür. 

c) Çözümlerin lineer kombinasyonları yeni bir çözümdür. Bu lineerliğin
doğal bir sonucu zaten.

d) İki çözümün entegrali bir çözümdür, mesela $S(x,t)$ bir çözüm ise, o
zaman $S(x-y,t)$ de bir çözümdür -(a)'da söylendiği gibi-, o zaman

$$ v(x,t) = \int_{-\infty}^{\infty}S(x-y,t)g(y)\ud y   $$

bir çözümdür. Aslında bu beyan, (c)'nin limite gidiyor iken olan versiyonu
/ şekli.

e) Eğer $u(x,t)$ yayılım denkleminin bir çözümü ise, o zaman genişletilmiş
(dilated) fonksiyon $u(\sqrt{a} x, at)$ da, herhangi bir $a>0$ için, bir
çözümdür. İspat (Zincirleme Kanunu ile)

$$ v(x,t) =  u(\sqrt{a} x, at) $$

$$ v_t = [\partial(at) / \partial t]u_t = au_t $$

$$ v_x = [\partial(\sqrt{a}x) / \partial x]u_x = \sqrt{a}u_x $$

$$ v_{xx} = \sqrt{a} \cdot \sqrt{a} u_{xx} = a u_{xx} $$

Hedefimiz özel bir çözüm bulmak ve özellik (d)'yi kullanarak tüm diğer
çözümleri inşa etmek. Aradığımız özel fonksiyon ise $Q(x,t)$ adını
vereceğimiz  bir fonksiyon ve bizim tanımladığımız özel başlangıç
şartları şöyle:

$$
Q(x,0) = 1, \ x>0 \textit{ için } \mlabel{7}
$$

$$ Q(x,0)=0, \ x<0 \textit{ için }  $$

Bu başlangıç şartlarını seçmemizin sebebi genişletme (dilation)
operasyonunun bu şartlara etki etmemesi. 

$Q$'yu üç adımda bulacağız. 

1'inci Adım

Çözümün şu özel formda olacağını söylüyoruz. 

$$ Q(x,t) = g(p), \ p = \frac{x}{\sqrt{4kt}} 
\mlabel{6} $$


$g$ tek değişkenli bir fonksiyon, fonksiyonun ne olduğunu sonra
tanımlayacağız, $\sqrt{4kt}$ ise daha ilerideki başka bir formülü
basitleştirmek için kullanılıyor. 

$Q$'nin niye bu özel formda olmasını istiyoruz? Çünkü çözümün genişletmeye
``dayanıklı'' olması gerekiyorsa, böyle bir çözüm sadece $Q$ $x,t$'ye
$x/\sqrt{t}$ kombinasyonu üzerinden bağlıysa mümkündür. Kontrol edelim,
genişleme $x/\sqrt{t}$'yi alıp $\sqrt{a}x/\sqrt{at} = x/\sqrt{t}$ yapar,
yani başlangıca döndürür. 

İkinci Adım

Öne sürdüğümüz çözümden ana PDE'ye erişmek için $Q_t,Q_{x},Q_{xx}$'i
bulmamız lazım. (6) üzerinde Zincirleme Kanunu kullanarak bunları elde
etmeye uğraşalım. 

$$ \frac{\partial Q}{\partial t} = 
\frac{\partial Q}{\partial p}\frac{\partial p}{\partial t}
$$

$Q$ tek bir değişken $p$'ye bağlı olduğuna göre $\partial$ yerine $d$
kullanabiliriz. 

$$ Q_t = 
\frac{dQ}{dp}\frac{\partial p}{\partial t}  = 
g'(p)\frac{\partial p}{\partial t} 
$$

$\partial p/\partial t$'yi şu şekilde hesaplarız

$$ 
\frac{\partial }{\partial t}
\bigg( 
\frac{x}{\sqrt{4k} t ^{-1/2}}
\bigg) = 
-\frac{1}{2} \frac{x}{\sqrt{4k}} t ^{-3/2} = 
-\frac{1}{2t} \frac{x}{\sqrt{4kt}}
 $$

Yani 

$$ Q_t = 
- g'(p)\frac{1}{2t} \frac{x}{\sqrt{4kt}}
$$

Aynı mantıkla $Q_x$

$$ Q_x = g'(p)\frac{1}{\sqrt{4kt}} $$

Bir daha $x$'e göre kısmi türev alalım, yine Zincirleme Kanunu 

$$ Q_{xx} = \frac{\partial dQ_x}{\partial dp}
\frac{\partial p}{\partial x} = 
\frac{1}{\sqrt{4kt}}g''(p)\frac{1}{\sqrt{4kt}} 
 $$

$$ = \frac{1}{4kt}g''(p) $$

$$ Q_t - kQ_{xx} = 
- g'(p)\frac{1}{2t} \frac{x}{\sqrt{4kt}} -
k\bigg[ 
 \frac{1}{4kt}g''(p) 
\bigg] = 0
 $$

(6)'ya göre $p = x/\sqrt{4kt}$, o zaman 

$$ =
- g'(p)\frac{1}{2t} p - \frac{1}{4t}g''(p)  = 0
 $$

$$ = \frac{1}{t} \bigg[
-p\frac{1}{2}g'(p) - \frac{1}{4}g''(p)
\bigg] = 0
 $$

Tüm çarpım sıfıra eşitse, köşeli parantez içi sıfır demektir, çünkü $0 < t$
şartı var, $t$ sıfır olamaz, ayrıca parantez içini 4 ile çarparsak, sıfıra
eşitlik bozulmaz, o zaman 

$$ g'' + 2pg = 0 $$

Bu bir ODE ! Çözüm için entegre edici faktör (integrating factor) $\exp \int 2p
\ud p = exp(p^2)$ kullanılabilir. $g'(p) = c_1 \exp(-p^2)$ elde ederiz.

$$ Q(x,t) = g(p) = c_1 \int e^{-p^2} \ud p + c_2 $$

3'üncü Adım 

Bu adımda $Q$ için nihai bir formül üretmeye uğraşacağız. Üstteki formül
sadece $t>0$ şartını dikkate aldı, fakat bizim (7)'de belirttiğimiz
başlangıç şartlarımız da var. Bu şartları limit olarak ifade edebiliriz,
çünkü mesela $x>0$ sonrası tüm değerler için bir şey tanımladık, ve $Q$'nun bu
durum için olacağı nihai değer belli, şarta göre 1. Bu tanım limit tanımını
çağrıştırıyor, 

Eğer $x>0$ ise

$$ 1 = \lim_{t \to 0} Q = 
 c_1 \int_0^{\infty} e^{-p^2} \ud p + c_2 = 
c_1 \frac{\pi}{2} + c_2
 $$

Eğer $x<0$ ise

$$
0 = \lim_{t \to 0} Q = 
 c_1 \int_0^{-\infty} e^{-p^2} \ud p + c_2 = 
-c_1 \frac{\pi}{2} + c_2
 $$

Not: $\int_0^{-\infty} e^{-p^2} \ud p$ entegralinin niye $\pi/2$ sonucunu
verdiğini ``$e^{-x^2}$ Nasıl Entegre Edilir?'' yazısında bulabilirsiniz.

Böylece katsayılar $c_1 = 1/\pi$ ve $c_2 = 1/2$ sonuçlarına erişiyoruz. O
zaman $Q$ şu fonksiyon olmalı

$$
Q(x,t) = 
\frac{1}{2} + \frac{ 1}{\sqrt{\pi}} \int _{0}^{x/\sqrt{4kt}} e^{-p^2} \ud p
$$

$t>0$ olmak üzere. Bu sonucun ana PDE denklemi, (6) ve (7) denklemlerini
tatmin ettiğine dikkat. 

4'üncü Adım

$Q$'yu bulduktan sonra şimdi $S = \partial Q/\partial x$ tanımını
yapalım. (b) özelliği üzerinden $S$'nin aynı zamanda ana denklemin bir
çözümü olduğunu biliyoruz. Herhangi bir fonksiyon $\phi$ için, ayrıca şunu
da tanımlıyoruz. 

$$
u(x,t) = \int _{-\infty}^{\infty} S(x-y,t) \phi(y) \ud y  \mlabel{8}
$$

$t>0$ için. (d) özelliği üzerinden $u$, PDE için yeni bir çözümdür. Biz ek
olarak şunu iddia ediyoruz ki $u$ özgün (unique) çözümdür. PDE'nin ve
sınır şartının çözüm için doğru olup olmadığını kontrol etmek için 

$$
u(x,t) = \int _{-\infty}^{\infty} 
\frac{\partial Q}{\partial x}(x-y,t)\phi(y) \ud y
$$

$$ = -\int _{-\infty}^{\infty} 
\frac{\partial }{\partial x}[Q(x-y,t)]\phi(y)\ud y
  $$

Parçalarla entegral (integration by parts) uyguladıktan sonra 

$$ 
=  + \int _{-\infty}^{\infty}  Q(x-y,t)\phi'(y)\ud y - 
Q(x-y,t)\phi(y) \bigg| _{y=-\infty}^{y=+\infty}
$$

Limitlerin yokolacağını farzediyoruz. Özelde, geçici olarak $\phi(y)$'nin
büyük $|y|$ değerleri için sıfıra eşit olacağını farzedelim. O zaman 

$$ u(x,0) = 
\int _{-\infty}^{\infty} Q(x-y, 0) \phi'(y)\ud y 
 $$

$$ 
= \int _{-\infty}^{\infty} \phi'(y)\ud y = \phi \bigg| _{-\infty}^{x} = \phi(x)
$$

Üsttekinin sonucun sebebi $Q$ için farzettiğimiz başlangıç şartları, ve
$\phi(-\infty) = 0$ faraziyesi. Bu PDE'nin sınır şartı. (8)'ın çözüm
formülü olduğunu anlıyoruz, ki 

$$ S = \frac{\partial Q}{\partial x} = \frac{ 1}{2\sqrt{\pi kt}} 
e^{ -x^2 / 4kt}
$$

$t>0$ için. Yani 

$$
u(x,t) =  \frac{ 1}{2\sqrt{\pi kt}} 
\int _{-\infty}^{\infty} e^{-(x-y)^2 / 4kt} \phi(y) \ud y
$$

$S(x,t)$ literatürde {\em kaynak fonksiyonu} (source function), {\em Green'in
fonksiyonu}, {\em temel çözüm} (fundamental solution), {\em gaussian}, ya
da {\em aktarıcı / yayıcı} (propagatör) ya da {\em yayılım} (diffusion) denklemi
olarak bilinir. Ya da basitçe yayılım çekirdeği (diffusion kernel). Formül
sadece $t>0$ için bir çözüm verir, $t=0$ için anlamsızdır. 


\end{document}
