\documentclass[12pt,fleqn]{article}\usepackage{../../common}
\begin{document}
Doktora Derecesi

Gerçekten istediğim şey doktora mı? Doktora süreci neyi kapsıyor?

Doktora, tek bir konu üzerinde yoğunlaşıp, 'derinliğine' inerek araştırma
yapacağınız uzun bir dönemdir. Uzun derken 6 sene gibi bir uzunluktan
bahsediyorum. Derinlik derken doktora sonuna yaklaştığınızda, araştırma
yaptığınız dalda dünyanın en önde, ya da öne yakın bir uzmanı
olacaksınız. Bu alanda size danışmanlık yapan hocanızdan bile daha çok
biliyor olacaksınız. İçinde olduğunuz okulda konunuzun en ileri noktası siz
olacaksınız. Tek bir konu derken, doktora yıllarınızın son iki yılında
oldukça odaklı ve yoğunlaştığınız tek bir konu üzerinde araştırma yapmaktan
bahsediyorum. Doktora zamanı, enine öğrenme (genelleşme) zamanı değil, konu
'odaklı' ve 'derinliğine' öğrenme zamanıdır [1].

Artık Önemli Olan Ders Almak Değil

Lisans, Yüksek lisans ve Mastır dereceleri hep 'enine' olan
derecelerdir. Mezuniyet için gerekli olan derslerin sayısı bu iki derece
için oldukça fazladır. Mesela bilgisayar programcılığı hakkında lisans için
Matematik, Bilgisayar ya da Mühendislik bölümlerinden 3 ya da 4 ders almak
gerekir. Mastır için de durum aynıdır.

Karşılaştıracak olursak, doktora programı için geçen 6 sene içerisinde
genelde 10'dan az sayıda ders almanız yetiyor. Carnegi Mellon'da 5 mecburi
ana ders, 3 tane de mecburi seçmeli dersi vardır. Yani doktora sırasında
vurgulanan alınan ders sayısı değil 'araştırma yapmaktır'. Bir doktora
öğrencisi, bir dersi genelde araştırmasına yardımcı olacağını umduğu için
alır. Bilgisayar konusunda araştırma yapan doktora öğrencisin aldığı
dersler, bilgisayar bölümünden bile olmayabilir! İstatistik, Uygulama
Araştırma, Ruhbilim, Dilbilim ya da öğrencinin araştırmasına hangi ders
yararlı olacaksa bu ders alınır.

Araştırma Süreci, Danışman ile Doktora Öğrencisinin Etkileşimi

Daha önce belirttiğimiz gibi, doktora derecesinin amacı araştırma
yapmaktır. Genelde araştırmanız, kendinize bir danışman seçtiğinizde
başlar. Çoğu okulda danışman seçimi ilk senede oluyor. CMU üniversitesinde,
biz doktora öğrencilerinin hemen araştırma yapmaya başlamasını istiyoruz, o
sebeple programa girdiğinizden bir/iki ay sonra danışman seçmeniz
gerekiyor.

Tekrar belirtelim. Araştırma, derse girmek gibi değildir. Birçok öğrenci
derse gitmek ve derse çalışmak olgusundan araştırma yapma olgusuna geçişi
yapamıyor, sonuç olarak doktoraya başlayan öğrencilerin ancak yarısı
doktora derecesini alarak mezun oluyor. (CMU'da 3/4 ögrenci doktor olarak
mezun oluyor). Unutmayın, bu bahsettiğimiz öğrenciler lisans diplomasını 4
üzerinden 4.0 ile almış öğrenciler.

Araştırma ile ders girmenin farkı:

\begin{itemize}
\item Derste ev ödevi olarak verilen problemlerin çözümü bellidir, ve çözüm
  için kullanılacak teknik bilinir, ödev verilmeden önce derste
  öğretilmiştir. Araştırma yaparken bir konu üzerinde yıllarca, çözümün
  bile olup olmadığını bilmeden çalışabilirsiniz. O problemi çözmek için
  yeni yöntemler bulacak, 'keşfedecek' olan sizsiniz.
\item Derste, üzerinde çalışacağınız problemler size verilir. Araştırmada,
  üzerinde çalışacağınız problemleri kendinizin seçme şansınız
  var. Ayrica, 'güzel' problemleri bulmak sizin göreviniz. Güzel derken
  'temel' demek istiyorum. Yani, Oracle veri taban programına bir yama
  yapıp daha hızlı işlemesini sağlamak eminim Oracle'ı çok sevindirir, ama
  bu yaptığınız temel araştırma sayılmaz. Ama veri analiz hızını arttıracak
  yeni bir algoritma bulmak temel araştırma sayılır. Böyle çığır açan
  araştırmalar yaparken, bir ayrı göreviniz de, aynı şeyi başkasının yapıp
  yapmadığını kontrol etmek. Bunu başarmak için seçtiğiniz alanda daha önce
  yazılmış yüzlerce makale okumanız gerekebilir.
\item Ders alırken eğer bir ödev problemini çözemiyorsanız, öteki sınıf
  arkadaşlarınıza sorabilirsiniz. Eğer arkadaşlarınız bilmiyorsa, hocanıza
  sorarsınız, o mutlaka biliyordur. Araştırma yaparken, 'yanlızsınız', en
  iyi şartlarda bile danışman hocanız ve bir başka öğrenci ile aynı anda
  çalışıyor olacaksınız. Sonuçta dünyadaki herkese sorunuzu sorabilirsiniz,
  fakat genelde cevabı onlarda bilmiyor olacak; bilselerdi yaptığınız zaten
  araştırma olmazdı! Birçok öğrenci yanlız çalışmakta çok güçlük çeker.
\item Derslerde sürekli not verilirsiniz, ve sürekli önünüzde olan konular,
  ne öğretileceği önceden söylenir. Araştırmada not yoktur. Genelde
  danışmanınızdan biraz yön verilir, ama onun haricinde hedefinize doğru
  ilerlemek, kendinizi motive etmek ve insiyatif göstermek sizin göreviniz.
\item Ders alırken, bir hocayı çoğu zaman tek başına
  yakalayamazsınız. Fakat araştırma yaparken, eğer bir hoca sizin
  danışmanınız ise, hocanız ile her hafta en az 1 saat başbaşa
  oturacaksınız. Eğer hocanız çok meşgul bir hoca ise (unutmayalım
  hocalarda araştırma, ders, teşvik para başvuruları, komitelerde oturmak,
  konferans vermek için seyahat, vs ile uğraşmaları gerekir), o zaman
  haftada 1/2 saat alabilirsiniz. Eğer hocanız yeni bir hoca ise, o zaman
  haftada 2 saat alabilirsiniz. Önceden plan yaparak bu saatler için
  hazırlanmak sizin göreviniz, ki böylece bu saatten en faydalı şekilde
  yararlanmanız mümkün olsun.
\item Derste hocanız ile aranızda bir mesafe vardır. Araştırma yaparken
  danışman hocanız ile yanyana çalışacaksınız. Tabii ki hocanız size yön
  verecek, fikir sağlayacak, okumanız için makaleler verecek, program
  ödevleri verecek, ve bu görevler için son zaman tesbit edecek, vs. Fakat
  hocanız ile yan yana çalışırken, 'eşit meslektaş' olarak
  çalışacaksınız. İkiniz de birbirinizden öğreneceksiniz. Beraber keşif
  yapacaksınız. Çoğu öğrenci, danışman hocalarının beraber araştırma
  yaparken derstekinden ne kadar değişik olduğuna şaşırıyor. Mesela derste
  çok sıkıcı, kuru ve rahatsız bir şekilde ders anlatan bir hoca, araştırma
  yaparken canlanıyor, ve muazzam bir heyecan ile işe sarılıyor. Derste,
  profosörleri 'zaten çözdüğü problemleri' tekrar çözerken
  seyredeceksiniz. Bütün problemler dersin sonunda çözülmüş
  olurlar. Araştırma yaparken, hocanızı sesli düşünürken dinleyeceksiniz,
  ve problemleri nasıl çözdüğünü ve yaklaştığını görmeniz mümkün
  olacak. Öğrenciler genelde bunu çok yararlı buluyorlar. Bazen, mesela
  kendinizin hocanızdan daha hızlı şekilde düşündüğünüzü farkedebilirsiniz,
  ama hocanızın sizden daha çok fikri olabilir. Siz hesaplama ve
  programlama da iyi olabilirsiniz, ama hocanız matematiksel ispatlarda,
  makale yazmak ve konuşmalarda sizden daha iyi olabilir. Bu birçok
  öğrenciyi şaşırtıyor, çünkü tipik olarak danışman hocalarının her konuda
  kendilerinden daha iyi olmasını bekliyorlar! Dudak bükmeyin.. zaten bu ne
  kadar gerçek dışı bir beklenti değil mi? Hayatta olduğu gibi, yakınmadan
  eğer hocanızın sizde olmayan hangi özelliklerinin olduğunu saptayıp, bu
  özellikleri kapmak için uğraşırsanız başarılı olursunuz.
\end{itemize}

Unutmayın ki, başka kimse size araştırmanın nasıl birşey olduğunu
anlatamaz. Araştırmanın nasıl olduğunu anlamanın en rahat yolu, araştırma
yapmaktır! Ne kadar erken olursa, o kadar iyi.

Araştırmanın Verdiği Mutluluk ya da Moral Bozuklukları

Araştırma çok ödüllendirici, ama bazen de çok engel çıkartan ve bu yüzden
hayal kırıklığı yaratan bir eylem olabilir. Çoğu üst-lisans öğrencisi
üst-lisans eğitimini, ta en aşağılardan, en yukarılara kadar inip çıkabilen
bir dönmedolap gibi tarif ediyorlar.

Elde olmayan sebeplerden çıkabilecek öfke ve sıkıntı şu sebeplerden
olabilir. Mesela, aynı konu hakkında çalışan başka biri, sorunu sizden önce
çözmüştür. Ya da, öfke/sıkıntı yanlızlıktan gelebilir. Fakat en büyük
ihtimalle, elde olmayan sebeplerden çıkacak öfke/sıkıntının sebebi şu
olacaktır.

Sandığınız kadar akıllı olmadığınızı farketmek.

Ekteki çok tipik bir örnek.

"Öğrenci X ülkesi Y'de çok ünlü olan Z üniversitesinden mezun olarak
gelmiş. Geldiği okulda binlerce kişi arasından 5. olarak mezun olmuş. Not
ortalaması olarak son sene, sınıf birincisi olmuş. Öğrenci, doktora
programına 'ben en iyi olacağım' beklentisi ile başlıyor, ve araştırma ile
çok yoğun olarak dört elle sarılıyor. Birinci ya da ikinci senesinin
sonunda bakıyor ki, hiç makale yayınlayamamış. Evdeki arkadaşları, ailesi
'neyi var bu çocuğun' diye merak etmeye başlıyorlar. Bu görünmez engele
karşı öğrenci kızgınlık ve utanç hissediyor. Danışman hocasını suçluyor,
bölümünü suçluyor, okulunu suçluyor. Fakat en sonunda, olgunluk gösterip,
'büyüyüp', belki de en iyi olmadığını kabul ediyor, fakat gene de eğer iyi
çalışırsa başarıya ulaşabileceğini anlıyor. Daha çok 'dinlemeye' başlıyor,
çok çalışıyor ve sonunda başarıya ulaşıyor."

Araştırma bütün ters gidebilecek yanlarına rağmen, çok haz verici bir
süreçtir de aynı zamanda. Bazısı için araştırmanın zevki, kimsenin
bilmediği yeni bir şeyi keşfetmektir. Yeni bir algoritma bulmuş
olabilirsiniz, yeni işletim sistem tasarım fikri bulmuş olabilirsiniz, ya
da bir sabit disk erişim hızı arttırmış olabilirsiniz. Ötekiler için
araştırmanın hazzı, gerçekten ama gerçekten anlamış olmanın verdiğı
hazdır. Sınıfta ders veren hocanızın tam ders ya da kitap ilginç gelmeye
başlarken, durup, "bu konunun gerisi, ders kapsamımımız dışında" dediğine
şahit oldunuz mu? Araştırma yaparken, bir konuyu istediğiniz kadar
derinlikte peşinden koşarsınız, ve hakkında her şeyi anlayabilirsiniz. Çoğu
doktora öğrencisi için de araştırmanın hazzı, damgasını vurmuş olmak, bir
konuda etkisini hissettirmek bir şeyleri değiştirmiş olmaktır, mesela
sistemlerin yapılış tarzını değiştirdiniz, ya da sistemlerin daha akıllıca
tasarlanması için yardım etmiş oldunuz. Tabii 'bir işi, yapılması gerektiği
gibi yapmanın' verdiği haz da vardır. Bir şirkette, amaç ürünü çalışır hale
getirip piyasaya sürmektir. Araştırma yaparken, projenizi uzun uzadıya
planlayıp ve her açıdan kararlaştırıp, her tasarım seçiminizi gurula
savunabilir hale gelmeniz mümkündür. Araştırma, çabucak toparlama yamayla
kapatma zamanı değildir. Çoğu insan da, bir konu hakkında otorite olmayı,
ve araştırmasının başkaları tarafından referans gösterilmesini sever.

Doktora Sırasında Sermaye Kaynağı

Anne babanızın para verdiği lisans sürecinin, ya da mastır sırasında
asistan olarak çalışıp ve halen para ödediğiniz sistemin tersine, doktora
sırasında para artık sizin için problem olmayacak. Çoğu okulda doktora
öğrencileri doktora sırasında hiç para ödemezler. Hatta üstüne okuldan
yaşam gelirleri için maaş bile bağlanır, bu genelde ayda \$1700
civarıdır. En iyi şartlarda, tek yaptığınız araştırma olacaktır. Bunun ismi
'araştırma yardımcı görevlisi' (Research Assistanship) olmaktır.

Doktora, muazzam bir fırsattır. Istediğiniz konuda istediğiniz danışmanı
seçip, bol yardım göreceğiniz, problemler hakkında derin derin
düşünebileceğiniz, makale yayınlayabileceğiniz, ünlü olabileceğiniz, aynı
zamanda 6 sene sıfır okul ücreti ödeyip, üstüne maaş alacağınız bir ortamda
olacaksınız. Bu fırsatın bedelini danışman hocanız ödüyor olacak; bunu,
şirketlere ya da devletten teşvik sermayesi alarak yapacak. Bir danışman
hoca için her doktora öğrencisinin maliyeti yılda 50,000 doları bulabilir
(ders ücretleri, maaş, okulun kestiği vergi, alet/edevat masrafı, vs).

Önemli not 1: Çoğu okulda, yardımcı araştırmacı olarak çalışmak, sadece
danışman hocanızın tesvik parası var ise mümkün oluyor. Bazı hocalar tesvik
için başvuru yapmadığı, ya da teşvik parasının az olduğu alanlarda
oldukları için, yaşam ücreti için yardımcı öğretmen olarak çalışmaya mecbur
olabilirsiniz. Ben üst-lisans öğrencisi iken, bazı arkadaşlarım tam 13
dönem yardımcı öğretmenlik yapmak zorunda kaldılar, kendi kendilerine okulu
devam ettirebilmek için! Bu tabii seçeneklerden sadece birisi, diğer
seçenek, danışman hocalanızı elinde teşvik olanlardan seçmek. CMU
üniversitemizde sistem gayet güzel, her doktora öğrencisi maaş ve ders
ücreti, danışmanı kim olursa olsun okul tarafından ödeniyor.

Önemli not 2: Birçok şirket ve hükümet birimleri, üst-lisans para desteği
(graduate fellowship) verir. Eğer şanslı çıkıp bunlardan birini
alabilirseniz, bu destekler bütün doktora sürenizi karşılamış olur, ve
böylece danışman hocanızın tesvik parası olup olmadığı önemli olmaz.

Doktoradan Sonra Hayat

Hayatınızın 6 senesini planladığınız şu zamanda, şöyle bir durup,
'bitirdikten sonra' ne yapacağınızı düşünmeniz yararlı olur. Çoğu öğrenci
doktora bittikten sonra, akademiya (ya üniversiteye ya sadece ders verilen
bir ortama) geri dönüyor ve profosör oluyor, ya da araştırma labaratuvarına
giriyorlar. Bazı öğrenciler doktorayı aldıktan sonra bir daha hiç araştırma
yapmıyorlar, böyle arkadaşlar için, bizce, doktora derecesi ve onun için
harcanan süre, koca bir zaman kaybıdır.

Eğer bir araştırma üniversitesinde profosör olacaksanız, hayatınız şöyle
geçecek.

\begin{itemize}
   \item Ne istersen o konuda araştırma yap
   \item Doktora öğrencilere yardım et
   \item Ders ver
   \item Teşvik sermayesi için başvuru yap
   \item Başka araştırmacılar ile çalışmak ve konferanslar vermek için
     seyahat yap
   \item Bölümünüz için yardımcı bazı hizmetler yapmak (bu konuşmayı size
     vermek gibi)
\end{itemize}

Farkettiyseniz, 'hayatınız' dedim, 'işiniz' demedim. Çünkü yeni bir
araştırmacı için, işiniz, hayatınız olacak. Benim için harika bir hayat bu,
çünkü bütün bu eylemlerin hepsini yapmayı zaten ben çok seviyorum. Ve bu
eylemlerin hepsinde de sıkı çalışıyorum, fakat aynı zamanda farketmeden de
geçemiyorum ki, bu herkese göre bir iş değil.

Eğer sadece öğretim yapan bir üniversitede iseniz, işiniz şunlar olacak.

\begin{itemize}
   \item Bir sürü ders ver
   \item Bölümün için hizmetler yap
   \item Arada sırada alt-lisans öğrencilere araştırmalar hakkında yardım et
   \item Arada biraz kendi araştırmanı yap
\end{itemize}

Eğer araştırma labaratuvarına katılırsanız, işiniz şunlar olacak.

\begin{itemize}
   \item Araştırma yap (yarısı kendi istediğin konular üzerinde, yarısı
     şirketinizin istediği konular hakkında)
   \item Şirketteki öteki insanlar ile çalış
   \item Ötekiler ile çalışmak ve konuşma yapmak için biraz seyahat.
\end{itemize}

Doktora Derecesi Almalı mıyım?

Bu kararı alırken, akılda tutulacak konulardan bazıları:

\begin{itemize}
   \item Doktora herkes için uygun değildir!
   \item Doktora derecesi, ortalama 6 sene gerektirir.
   \item Eğer araştırmayı ve öğretmeyi denediniz, ve bunlardan en az birini
     sevmiyor iseniz, doktorayı hiç düşünmeyin! Not: Doktora programı
     çoğunlukla araştırma içerir, öğretmek değil, fakat eğer içinizde
     öğretme aşkı var ise, bu motivasyon doktorayı bitirmenize yardım
     edebilir, sonuçta öğretmen olabilmek için. Bunun birçok örneğini
     gördüm.
   \item Doktora, belli bir karakter yapısı gerektiriyor. Bir problemi
     çözmeye fanatik bir saplantı haline getiren biri olmanız
     lazım. Kesinlikle pes etmeyen bir kapasiteniz olmalı, ve ağır çalışmayı
     göze alabilen ve yapabilen biri olmalısınız. Probleminizi çözmek için ne
     gerekiyorsa yapmayı göze almak da lazım, mesela 5 tane matematik dersi
     almak, veri tabanı gibi tamamen yeni bir alan öğrenmek, bütün işletim
     sistem çekirdeğini baştan yazabilmek gibi
   \item Niye doktora istediğinizi bilmelisiniz. Bu amacınız hakkında
     vizyon ve fikir sahibi olmalısınız, ve kendinizi bu konular hakkında
     anlatabilmelisiniz.
   \item Normal olarak, 4 sene alt-lisansı bitirdikten sonra bazı
     öğrenciler hala tam karar vermemiş oluyor. Bu normal, ben de bu
     öğrencilerden biriydim. Böyle öğrenciler için en iyisi, bir araştırma
     ya da sanayii labaratuvarında bir kaç sene araştırma ortamında
     çalışıp, sonra kararı vermek. Eğer emin değilseniz, birkaç sene
     çalışmayı şiddetle tavsiye ederim. Üst lisansa, ne istediğinizi
     anlamadan katiyen başvurmayın.
\end{itemize}

Benim hikayem şöyle oldu: Matematik ve Bilgisayar hakkında alt-lisansı
bitirmiştim. Bundan sonra, GTE şirketinin sanal zeka labaratuvarında
çalışmaya başladım. İlk önce, maaş ve tek başıma kendimi destekleyebilmek
bana çok güzel geldi. Araştırma yaptığım alanı da seviyordum; benzer-oluş
tanıma ve kategorileştirme. Otomatik karşılıklı ilişki matrislerinin
özvektörlerini kullanarak, bakış açısı bazlı değişimler ile
uğraşıyordum. Fakat bir süre sonra farkettim ki, bu konuda daha fazla bilgi
sahibi olmak istiyorum. Niye bazı algoritmaların iyi sonuç verdiğini, niye
ötekilerin kötü sonuç verdiğini anlamak istiyordum. Kendi algoritmalarımı
yaratmak istiyordum. Kendi sorularımı cevaplayacak yeterli matematik bilgim
olmamasından endişe ediyorum, vs.. Yani sonuç olarak, konuya daha derin
dalmak istiyordum. Şirkette beraber çalıştığım çoğu kişi, böyle şeyleri
istediğim için benim bir 'garip' olduğumu düşünüyordu. 2 sene sonra istifa
ettim, ve doktoraya başladım. Okuldaki ilk ay etrafıma bakıp gördüm ki,
herkes aynen benim gibi bir garip! Bunu farkedince, doğru seçimi yaptığımı
anladım.

Doktora Öğrencilerine Öğütler

(Carnegie Mellon üniversitesi profosörü Manuel Blum'un lisansüstü
bilgisayar bilim öğrencilerine konuşmasından alınmıştır)

Üst lisansın dört eylemi: Okumak, Aritmetik, Araştırmak, Yazmak

Sunuş Sırası

\begin{itemize}
   \item Okumak, Çalışmak, Düşünmek
   \item Doktoranın Başında
   \item Doktoranın Ortasında
\end{itemize}

Okumak

Kitaplar tomar değildir.

Tomarların, torah gibi baştan sona okunması gerekir.

Kitaplar, rasgele erişimlidir -- tomarlara göre büyük ilerleme yani.

Kitapların bu özelliğinden istifade edin! Bir kitabı baştan sona okumakla
kendinizi yükümlü hissetmeyin. Kitabın herhangi bir yerinden açıp okumaya
başlamakta hiç bir sakınca yoktur.

Özellikle matematik ve fizik gibi ağır olan konuların kitaplarında,
anlayabildiğiniz ne var ise oradan başlayın. Okuyabildiğiniz kadarını
okuyun. Sayfa boşluklarına yazın (bunun ne kadar faydalı olabileceğini
biliyorsunuz). Böylelikle, aynı kitaba geri döndügünüzde, artık daha çok
şey okuyabileceksiniz. Böyle yaparak, her seferinde azar azar mesafe
katederek, muazzâm zor konuları bile öğrenmeniz mümkündür.

Okuduğunuzu bir yandan deftere yazmayı düşünün. Eğer çok zor bir konuyu
okuyacaksanız, okuduklarınızı yazman yararlı olabilir.

MİT'de Bertram Konstant adında bir matematik profosörünü hatırlıyorum. Ne
zaman odasında olsa, kapısı açık olurdu.

Yazardı.

Yazardı. Sürekli yazardı.

Araştırmasını mı yazıyordu? Belki.

Aklına gelen fikirleri mi yazıyordu? Belki.

Bence, okuyordu, ve okuduklarını aynen yazıyordu.

Şahsen benim için de okuduklarımı yazmak, zor bir konuyu öğrenmenin en
kârlı ve zevkli yollarından biridir.

Çalışmak

Hepiniz bilgisayar bilimcisiniz.

Hepiniz Finite Automata'nın ne yapabileceğini biliyorsunuz. Hepiniz Turing
makınasının ne yapabileceğini biliyorsunuz. Mesela Finite Automata toplama
yapabilir, ama çarpma yapamaz.

Turing makinaları bütün hesaplanabilir fonksiyonları hesaplayabilir.

Turing makinaları Finite Automata'dan kat kat daha üstündür.

Fakat TM ile FA arasındaki yegane fark şudur: TM'sının elinde kağıt ve
kalemi vardır, FA'nın ise yoktur.

Bir düşünün.

Bu, yazmanın gücünü gösteriyor.

Demek ki eğer yazmıyorsanız, Finite Automata seviyesine düşüyoruz demektir.

Ama yazarak, Turing makinasının gücüne erişebiliyoruz.

Düşünmek

Claude Shannon bana bir seferinde şunu anlatmıştı: Küçük yaştayken bir
resimli bulmaca yapıyormuş ve bir yerde takılıp kalmış. O sırada abisi
yanından geçerken şöyle demiş: "Sana şimdi bir ipucu verirdim, çözerdin
ama...".

Abisi sadece bu kadar demiş.

Fakat bu dedikleri Claude'ın bilmeceyi çözmesi için yeterli olmuş.

Bu ipucu'nun en güzel tarafı nedir biliyormusunuz?

Kendinize bu ipucunu istediğiniz zaman verebilirsiniz.

Tavsiyem şudur: Çok çetin bir problemde takılıp kaldığınızda minik bir
kuşun, ya da, kendinizin yaşlı hâlinin kendinize şöyle fısıldadığını
düşünün:

"Sana şimdi bir ipucu verirdim, çözerdin ama..."

Bir keresinde Umesh Vazirani adındaki bir MİT öğrencisine, her dönem nasıl
6 lisans seviyesinde ders alabildiğini sordum.

Bana problemleri zor yoldan çözmeye vaktinin olmadığını, o yüzden hep bir
kestirme bulduğunu söyledi.

Umesh anlamıştı ki, problemlerin çoğunlukla hem kısa hem de zekice bir
çözüm yolu vardır.

Bazen de öyle olur ki, bir problemin üstünde uzun uzun düşünürsünüz, ve
çözüm bula\textbf{ma}zsınız. Ve bir bakarsınız aynı meseleyi bir başkası
çözmüş. Dikkat edin, bu, yeni bir şey öğrenmek için büyük fırsattır.

Kaçırmayın.

Kendinize sorun: "Nasıl düşünmeliydim de bu çözümü {\em ben} bulabilmeliydim".

Bunu yapmanın bana çok yararlı olduğunu gördüm.

Bazen de bir problem üzerinde uzun uzun düşünürsünüz, ve çözümü BULURSUNUZ!

Ondan sonra bir bakarsınız ki, bir başkası böyle bir çözümü sizden önce
yayınlamış.

Bu durum sizin için ağır olabilir, ama bu da öğrenmek için iyi bir fırsattır.

Yayınlanan makaleyi okuyun.

Şaşkınlik ile göreceksiniz ki bu makale, sizin makalenize göre bazı
açılardan çok değişik. Bu 'öteki' makale aşaği yukarı:

\begin{itemize}
   \item \%50 ihtimalle sizin makalenizden tamamen değişik
   \item \%25 ihtimalle aynı, ama sizinki kadar iyi değil
   \item \%25 ihtimalle sizinkinden daha iyi
\end{itemize}

Bu demektir ki, \%50'den fazla bir ihtimalle halâ yayın yapma şansınız var.

Ya öteki makalenin daha iyi olduğu \%25 ihtimal sözkonusu ise?

İşte size öğrenmek için bir fırsat!

Kendinize sorun: "Nasıl düşünmeliydim de bu çözümü {\em ben} bulabilmeliydim".

Genç bir mühendisken, şu "modern cebir" denen güçlü yöntemi öğrenmem
gerektiğini işte böyle anlamıştım.

Elektrik mühendisliği lisans derecesinden, Matematik üst lisansa geçmemin
sebebi bu idi. Tabii bu daha bilgisayar bilim denen şeyden çok önceydi.

Gene düşünmek üstüne..

'Paradoks, yani mantığa aykırı gözükebilen düşüncenin' ve 'çelişkinin'
önemi üstüne..

Bir söylemin 'doğru' olduğunu matematiksel ispat etmişseniz, ve gene aynı
söylemin bir de yanlış olduğunu ispat edebiliyorsanız, bir buluşa çok
yaklaşmış olabilirsiniz.

Bir yerde bir şey yerine oturmamış demektir.

Çelişkinin gücünü hiç küçümsemeyin.

İnsanoğlu'nun en önemli bilgi kaynaklarından biridir.

Örneklerden biri, yalancının paradoksu olan: "Bu söylem yanlış"
paradoksudur. Bu düşüncenin kümeler kuramındaki uygulamalarını düşünün,
dilbilimde getirdiği yenilikleri...

Sayılabilme ve sayılamama alanlarında paradoklar var.

Yazılımbilimde 'donma problemi' paradoksu var.

Fizikte bir çok paradokslu konu var.

Kuantum kuramında Einstein-Rosen-Podulsky paradoksu var.

İzafi olarak hızlanan ikizler.

Maddenin dalgasal ve tanecik olabilme özelliği.

Burada benim şu anda üzerinde çalışmakta olduğum araştırmamdan
bahsedeyim. Paradoks kullanıyorum. Özellikle biliçin paradoksu ile
ilgileniyorum. Şu iki apayrı görüşü karşılaştırın.

1. Bu görüşe göre insanlar bir MEKANİZMAlar, oldukça fazla ama sonuçta
sınırlı hafızaları var, robotumsu varlıklar. Aynen bilgisayarın
programlandığı gibi programlanabiliyorlar. Ya da,

2. İnsanlar düşünce dolu, gözlemci yaratıklar ve tanrıvari bir hür
iradeleri var. İnsanlar bilinçli varlıklar. Son derece çetrefilli ve
yetenekli bir mekanizmanın kontrolü ellerinde, yaptıklarını, bilinçaltından
çıkıveren/üste gelen düşüncelerin arasından seçiyorlar, ve uyguluyorlar.

Bana göre bu iki görüşte doğru. Ama bu nasıl olabilir?

Johnsun'un hayatı adlı kitabında James Boswell, Samuel Johnson'ın bir deyişini
aktarır. "Hür iradeye bütün teoriler karşı gelir, ama bütün tecrübeler
destekler". Johnson, Newton'un öldüğü yılda 18 yaşındaydı.

Johnson biliyordu ki, F=ma'nin gösterdiği, insanların mekanizma olduğu idi.

"Hür iradeye bütün teoriler karşı gelir, ama bütün tecrübeler destekler"

Benim dipnotum burada bitiyor.

Bir problemi nasıl çözeceğinize dair bir liste yapın. Benim en gözde
yöntemim ufak başlamak. Kıyaslamak gerekirse, David Gries'ınki kendini
muhtemel bir çözümün içine koymak. Örnek olarak David'in ünlü kahve kutusu
problemi. Bir kutu siyah ve beyaz kahve çekirdeği olduğunu düşünelim, ve
şunları yapalım. İki çekirdek çıkartalım, eğer ikisi de aynı renk ise
onların yerine bir beyaz çekirdek geri koyalım. Eğer çekilen iki çekirdek
ayrı renkler ise, onların yerine siyah bir çekirdek geri koyalım. Böyle
gidersek en son çekirdeğin rengi ne olur?

Beyin bir kastır. Kullandıkça güçlenir. Çok güçlü olsa bile, kullanılmazda
zayıf düşer. Kasparov Deep Blue'ya karşı satranç maçını kaybetmeden aylar
önce annesi Kasparov'a kızmıştı, satranç talimi yapmıyor diye. Annesi
endişesinde haklı çıkmıştı.

Doktoranın Başında

IVIC adlı bir şirkette girdiğim harika bir işi
hatırlıyorum. (IVIC=Instituto Venezolano de Investigaciones
Cientificos). Svaetichin adlı bir nürofizikçi bana çözmem için bir güzel
bir sorun verdi. Problemi ne yazıki ki çözemedim. Problem, ışığı altın
balığının gözündeki "tek bir hücrenin" üzerine odaklamanın yöntemini bulmak
idi. Svaetichin, siyah teneke üzerinde ufacık bir delik açarak ışığı
buradan süzmeyi denemişti, bu yaklaşım orta boy deliklerde işlemiş olsa
bile, çok ufak deliklerde ışığın sapmasına yol açıyor, değişik ışık
kalıpları ortaya çıkıyordu.

Svaetichin problemi çözemediğine göre, ben de çözemem diye karar verdim. Ya
da, bu problemin fiziki olarak çözümsüz olduğunun düşündüm. Şimdiki aklım
olsaydı, fizik kitaplarımın hepsini okumaya tekrar başlar, özellikle optik
kitaplarını hatim etmeye uğraşırdım, bir yandan etraftaki öteki
araştırmacılar ile konuşur, Svaetichin'a danışırdım, vs. Svaetichin, eğer
okuyuyor, düşünüyor, çalışıyor olsaydım, bana yardım ederdi.

Tez danışmanınızın size "kendisinin çözebileceği" bir problemi vermesini
beklemeyin. Tabii bunu yapabilir de.

\begin{itemize}
   \item Size sonucunu zaten bildiği bir problem verebilir
   \item Size çözümlenebileceğini düşündüğü, ama daha kendisinin çözmediği
     bir problem verebilir
   \item Size muzammam zor bir problem verebilir
\end{itemize}

Eğer size verilen problem yeterince zor ise, size tavsiyem olağan-dışı
cevaplara bakmanız. Bu noktaya geri döneceğiz.

Tez danışmanı hocanız, size kendisinin çok rahat ve bilgili olduğu bir
alanda problem verebilir. Böylece sorun çıktığında ona soru sorabilir, ve
yön alabilirsiniz.

Ya da, size kendisinin az ya da hiç bir şey bilmediği bir alanda problem
verebilir, böyle şartlarda sizin öğrendikleriniz ile onu bilgilendirmeniz,
ve eğitmeniz gerekecek.

Normal olarak ikinci şık için, sizin her şeyi kendi başınıza öğrenmeniz
gerekecek. Tabii kaynak olarak öteki arkadaşlarınız, makaleler, kitaplar,
ve derslerden yararlanacaksınız.

Bu iki tür tez danışmanı da sizin için iyi olabilir. Şahsen, hangi türün
ötekinden iyi olduğunu bilmiyorum. Tek bildiğim, hangi tür danışmanınızın
olduğunu "baştan bilmenizdir".

Hangi konuyu araştırırsanız araştırın, konuyu severek araştırıyor
olmalısınız. Öyle sevmelisiniz ki, başkaları çoktan o konuyu bıraktıktan
sonra halâ onu araştırabiliyor, düşünüyor olabilmelisiniz.

Doktoranın Ortasında

ANATOLE FRANCE şöyle demiş: "Bir üniversite öğrencisi (özellikle doktora
öğrencileri), herşey hakkında birşeyler, birşey hakkında da herşeyi
bilmelidir".

Doktora öğrencileri hakkındaki espriyi bilirsiniz. Doktora öğrencisi
gitgide daha az şey hakkında daha fazla şey öğrenir, sonunda hiçbirşey
hakkında herşeyi biliyordur.

Doktora sırasında konuyu öyle daraltacaksınız ki, bu konu hakkında herşeyi
bilebilesiniz.

İlk başta bu, bir topluiğnenin ucu ile uğraşıyorsunuz gibi
gelebilir. Dünyanın ufacık bir kesitidir sanki sizinki, kristal tanesidir,
güzeldir, ama daha büyük bir resim içinde mikroskopik kalır.

Bu ufacık dünyanızda usanmadan çalışın. Göreceksiniz ki, bu ufacık kesiti
anlamaya başladığınızda, konunuz, kesitiniz dünyayı kaplıyor.

Zamanla, kendi kum taneciğiniz üzerinde dünyayı göreceksiniz.

\begin{verbatim}
Kum tanesi üzerinde dünyayı görmek
Ya da yabani ot üzerinde cenneti,
Sonsuzluğu elinde tutmaktır
Ya da ebediyeti bir saatte.

WILLIAM BLAKE (1757-1827)
\end{verbatim}

Çok değişik türden araştırma çeşitleri vardır.

Mesela, doğru bildiğinizi ispatlamak için araştırma yapabilirsiniz.

Doğru olanı araştırabilirsiniz. Böyle araştırmaların en iyi olanları, başta
doğru bildiğiniz birşeyin yanlışlığını ispatlamayı başarırlar.

Mesela, Fred Hoyle "Büyük Patlama" terimini, tersini ispatlamaya uğraşırken
bulmuştu.

Gene şahsımdan örnek vereyim. N tane tamsayının ortalamasını bulmaya
uğraşan herhangi bir deterministik algoritmanın, N tam sayıyı sıraya dizmek
için gerektiği kadar karşılaştırma işlemi yapması gerektiğini düşünüyordum,
yani N log N. Hayretle gördüm ki, N tamsayının orta değeri O(n)
karşılaştırma ile bulunabiliyor!

Bir söylem S'in doğruluğunu ispata uğraşırken, hiç değilse biraz zamanı
bu söylemin yanlışlığını ispat için ayırın. Söylem hakikaten doğru bile
olsa, yanlışlığını ispata çalışmak yeni bir açıdan bakmanızı sâğlayacak, ve
size yeni fikirler verecektir [2].

Cahit Arf'in Tavsiyeleri

Ünlü matematikçilerden Dr. Cahit Arf öğrenci olduğu zamanlarda, hatta bazen
sonrasında bile, bir teoriyi öğrenmek ve incelemek istediği zaman kitapta
onun ispatının olduğu bölümü kapatarak o teoriyi önce kendisinin
ispatlamaya çalıştığını söyler. Bu çaba başarısız olabilir, ama sonra
cevaba, ispata baktığı zaman ondan daha çok şey öğrenebilecektir, çünkü
problemi kendimiz çözmeye çalıştığı zaman zihninde bir sürü soru
oluşmuştur, ve çözüme bakıldığı zaman, ve bu soruların cevabı alınınca,
daha derin bir şekilde öğrenmek mümkün olacaktır.

Kaynaklar

[1] Mor Harchol-Balter, 
    {\em Applying to Ph.D. Programs in Computer Science, Carnegie Mellon}, 
    \url{https://www.cs.cmu.edu/~harchol/gradschooltalk.pdf}

[2] Manuel Blum, 
    {\em Advice to a Beginning Graduate Student}, 
    \url{https://www.cs.cmu.edu/~mblum/research/pdf/grad.html}

\end{document}
