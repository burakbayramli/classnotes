\documentclass[12pt,fleqn]{article}\usepackage{../../common}
\begin{document}

Evri�imi bir daha tan�mlayal�m, 

$$f * g \equiv \int _{-\infty}^{\infty} f(\tau)g(t-\tau) \ud\tau $$

Evri�im s�raba��ms�zd�r, yani $f * g = g * f$, o zaman 

$$f * g \equiv \int _{-\infty}^{\infty} f(t-\tau)g(\tau) \ud\tau $$

ifadesi de do�rudur. 

$$ f * g = \sum _{-\infty}^{\infty} f_{t-i} g_i  $$

$$ f * g = \sum _{0}^{\infty} f_{t-i} g_i  $$

Ornek

\begin{minted}[fontsize=\footnotesize]{python}
import scipy.signal
f = [1,2,3,4,5,6]
g = [5,4,3,2,1]
print scipy.signal.convolve(a,b)
\end{minted}

\begin{verbatim}
[ 5 14 26 40 55 70 50 32 17  6]
\end{verbatim}

Ornek

[1, sf. 365]

\begin{minted}[fontsize=\footnotesize]{python}
import scipy.signal
d = 1/6. * np.array([1.0,1.0,1.0,1.0,1.0,1.0])
print scipy.signal.convolve(d,d) 
print scipy.signal.convolve(d,d) * 36.
\end{minted}

\begin{verbatim}
[ 0.02777778  0.05555556  0.08333333  0.11111111  0.13888889  0.16666667
  0.13888889  0.11111111  0.08333333  0.05555556  0.02777778]
[ 1.  2.  3.  4.  5.  6.  5.  4.  3.  2.  1.]
\end{verbatim}








Kaynaklar 

[1] Strang, {\em Computational Science and Engineering}

\end{document}
