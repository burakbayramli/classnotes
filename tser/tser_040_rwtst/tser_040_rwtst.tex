\documentclass[12pt,fleqn]{article}\usepackage{../../common}
\begin{document}
Rasgele Yürüyüş Testleri

Bir zaman serisinin rasgele yürüyüş (RY, random walk) olup olmadığını
anlamak için Genişletilmiş Dickey-Fuller (ADF) testini görmüştük. Şimdi bu
testin hangi teoriye dayandığını göreceğiz, önemli bir bölümünü kendimiz
kodlayacağız, ve \verb!urca! paketi üzerinden bu testin bazı yan ürünlerini
irdelemeyi öğreneceğiz. Alttaki tabloda farklı çeşitlerde rasgele yürüyüş
modellerini görüyoruz, bunlar sırasıyla trendlı RY, kaymalı (drift) RY ve
normal RY olarak tanımlanabilir. Bu modellerin test istatistikleri vardır.

\begin{tabular}{|c|c|c|c|}
\hline 
Sıfır Modeli & Alternatif & Kısıtlama & Test \\
\hline 
$y_t = y_{t-1} + e_t$ & 
$y_t = \alpha + \rho y_{t-1} + e_t$ & 
$\alpha=0,\rho=1$ &
$\phi_1$ \\
\hline   
$y_t = y_{t-1} + e_t$ & 
$y_t = \alpha + \beta t + \rho y_{t-1} + e_t$ & 
$\alpha=0,\beta=0,\rho=1$ &
$\phi_2$ \\
\hline   
$y_t = \alpha + y_{t-1} + e_t$ & 
$y_t = \alpha + \beta t + \rho y_{t-1} + e_t$ & 
$\beta=0,\rho=1$ &
$\phi_3$ \\
\hline   
$y_t =  y_{t-1} + e_t$ & 
$y_t = \alpha + \rho y_{t-1} + e_t$ & 
$\alpha = 0,\rho=1$ &
$\tau_\mu$ \\
\hline   
$y_t =  \alpha + y_{t-1} + e_t$ & 
$y_t = \alpha + \beta t + \rho y_{t-1} + e_t$ & 
$\beta = 0,\rho=1$ &
$\tau_\tau$ \\
\hline   
\end{tabular}


$e_t$ birbirinden bağımsız, aynı dağılımlı Gaussian'dır. Bu testlerden
$\phi_1,\phi_2,\phi_3$ birleşik hipotez testleridir, bu kavramı {\em
  İstatistik, Birleşik Hipotez Testleri} altında görmüştük. Mesela $\phi_2$
için ``kısıtlanmış'' durum sıfır hipotezi altında gösteriliyor,
``kısıtlanmamış'' durum ise alternatif kolonu altında. Reddetmek
istediğimiz hipotez iptal edilen değişkenlerin önemli olmadığı, eğer
F-testi ile bu hipotezi reddedersek, yani önemli olduklarını kanıtlarsak,
üstteki tablodaki sıfır hipotezi reddedilmiş olur.

Peki eğer $\phi_2$ için mesela, hem sıfır hem alternatif hipotez altında
$y_{t-1}$ görülüyor, o zaman rasgele yürüyüş varlığını nasıl reddederiz?
Evet iki tarafta da $y_{t-1}$ var fakat bir tarafta birim kök (unit root)
olarak var, yani $1 \cdot y_{t-1}$ olarak, öteki tarafta bir katsayı
üzerinden modele dahil, ayrıca diğer değişkenler de bu durumda mevcut. Eğer
pür birim kök'e karşı alternatif güçlü çıkar ise o tüm diğer değişkenler
zamann serisinin açıklamak için önemli sonucu çıkar, bu durum rasgele
yürüyüş olmadığına dair güçlü bir işarettir (tabii tamamen yok denemez).

Hesaplar bir tür birleşik hipotezdir demiştik, yani F-test'i olarak
hesaplanırlar,

$$ \phi_i = \frac{(SSE_r - SSE_u) / r}{SSE_u} / (N-k) $$

$r$ kısıtlanan değişken sayısıdır, $\phi_3$ için bu 3 olurdu mesela, SSE
değerleri ise artıkların karesi kullanılarak hesaplanacaktır, $SSE_r$
kısıtlanan durum için, $SSE_u$ kısıtlama olmayan durum için, $N-k$ işe
kısıtlı olmayan modelin serbestlik derecesi.

Örnek olarak Federal Reserve Board tarafından yayınlanan Üretim İndisi
verisinin logaritması alınmış halini kullanacağız, veri 1950 1. çeyrek-1977
4. çeyrek arasını kapsar ve çeyreksel bazlıdır. Endüstriyel Üretim İndisi
bir ekonomik göstergedir, sanayisel, madencilik, elektrik, doğal gaz gibi
sektörlerin üretimini tek bir sayı altında özetlemeye uğraşır. Soru şu,
acaba bu seri rasgele yürüyüş özellikleri taşıyor mu, taşımıyor ise, hangi
alternatifler geçerli? 

\begin{minted}[fontsize=\footnotesize]{python}
import pandas as pd
dfind = pd.read_csv('ind.csv',index_col='date')
dfind = dfind[dfind.index > '1950-01-01']
dfind = dfind[dfind.index < '1977-10-01']
\end{minted}

\begin{minted}[fontsize=\footnotesize]{python}
np.log(dfind.indpro).plot()
plt.title(u'Üretim Indisi')
plt.savefig('tser_rwtst_02.png')
\end{minted}

\includegraphics[height=6cm]{tser_rwtst_02.png}

Farz ediyoruz ki bu zaman serisi şu şekilde temsil edilebiliyor,

$$ 
y_t = \beta_0 + \beta_1 t + \alpha_1 y_{t-1} + \alpha_2 (y_{t-1} - y_{t-2}) + e_t 
\mlabel{1}
$$

Şimdi mesela $\phi_2$ testi için bir regresyon işleteceğiz, bunun için
denklem üzerinde biraz değişiklik yapalım, kısıtlamalar üzerinden
$\beta_0=0,\beta_1=0,\alpha_1=1$ ve iki taraftan $y_{t-1}$ çıkartalım,

$$ 
y_t - y_{t-1} =  \cancel{y_{t-1} - y_{t-1}} + \alpha_2 (y_{t-1} - y_{t-2}) + e_t 
$$

$$ 
\Delta y_t =  \alpha_2 \Delta y_{t-1}  + e_t  
\mlabel{2}
$$

Bu durumda kısıtlanmış regresyon (2)'yi kullanacaktır, kısıtlanmamış ise
(1)'i. Ardından bu iki regresyonun sonucundan gelen hata karelerinin
toplamlarını (SSE) kullanarak bir F-testi hesaplanır.

$\phi_3$ için benzer bir durum, tek fark kısıtlanmış formülde bir kesi hala
var, yani her iki tarafta da $\beta_0$ mevcut.

$$ 
\Delta y_t =  \beta_0 + \alpha_2 \Delta y_{t-1}  + e_t  
\mlabel{3}
$$

Tabii ana formül (1)'i de $\Delta y_t$ bazında göstermek iyi olur, iki
taratan $y_{t-1}$ çıkartalım,

$$ 
y_t-y_{t-1} = \beta_0 + \beta_1 t + (\alpha_1-1) y_{t-1} + \alpha_2 (y_{t-1} - y_{t-2}) + e_t 
$$

$$ 
\Delta y_t = \beta_0 + \beta_1 t + (\alpha_1-1) y_{t-1} + \alpha_2 \Delta y_{t-1} + e_t 
$$

$\alpha_1-1$ tek başına bir katsayı olarak girebilir, hangi sembolü
taşıdığı pek önemli değil.

Bu arada (1,2)'deki semboller tabloda gösterilenlerden biraz değişik
olabilir, fakat işin özü aynıdır. (1) formülü bir gecikme (lag) ile önceki
farklara bağlantı yaratmayı seçti, bunun için bazı sembollerin değişmesi
gerekti.

Regresyon ve F-test hesapları,

\begin{minted}[fontsize=\footnotesize]{python}
import statsmodels.formula.api as smf

dfind['y'] = np.log(dfind.indpro)
dfind['dy'] = dfind.y.diff()
dfind['ylag'] = dfind.y.shift(1)
dfind['dylag'] = dfind.ylag.diff()
dfind['t'] = range(len(dfind))

res1 = smf.ols('dy ~ t + ylag + dylag', data=dfind).fit()
sse1 = np.sum(res1.resid**2)
print list(res1.params)
print 'sse1',sse1

res2 = smf.ols('dy ~ dylag ', data=dfind).fit()
sse2 = np.sum(res2.resid**2)
print list(res2.params)
print 'sse2',sse2

# dikkat kesi yok burada
res3 = smf.ols('dy ~ 0 + dylag', data=dfind).fit()
print list(res3.params)
sse3 = np.sum(res3.resid**2)
print 'sse3',sse3

k = 3
phi2 = (sse3-sse1)/(k*sse1/(len(dfind)-4))
print '\nphi2', phi2
k = 2
phi3 = (sse2-sse1)/(k*sse1/(len(dfind)-4))
print 'phi3', phi3
\end{minted}

\begin{verbatim}
[0.32890563193314926, 0.0012176940580472463, -0.11592339963358432, 0.47618947733620209]
sse1 0.0439920148353
[0.0054652611288141572, 0.41767827251937806]
sse2 0.0490062183592
[0.49810699391267177]
sse3 0.0517366340291

phi2 6.22029276313
phi3 6.04093237744
\end{verbatim}

Kontrol için aynı hesabı \verb!urca! paketi üzerinden işletelim, ki kritik
değerleri de görebilelim. Not: kritik değerler için de F dağılımı
kullanamıyoruz, çünkü artıklarda (residuals) normallik farz edemiyoruz, ki
onların karesi ve toplamı chi kare olsun, ve chi kare bölümler F
ölsün. Kalıntılarda normallik niye farz edilemiyor? Çünkü regresyonda bir
önkabul değişkenlerin de $N(0,\sigma)$ olarak dağıldığıdır, fakat sıfır
hipotezine bakarsak regresyona sokulan değişken zaman serisinin ta kendisi,
ve zaman serisi sıfır hipotezi altında normal dağılmaz ({\d değişimi}
normal dağılır), bu yüzden kalıntılar da normal olamaz. [5] bu standard
olmayan F dağılımı simülasyon ile üretiyor ve kritik değerleri bir tabloda
paylaşıyor. Simülasyonun nasıl yapıldığı [1, sf 204]'te. 

\begin{minted}[fontsize=\footnotesize]{python}
%load_ext rpy2.ipython
%R library(urca)
\end{minted}

\begin{minted}[fontsize=\footnotesize]{python}
series = np.log(dfind.indpro)
%R -i series
%R  adf <- ur.df(series, type = 'trend',lags=1)
%R -o adfout adfout <- summary(adf)
print adfout
\end{minted}

\begin{verbatim}

############################################### 
# Augmented Dickey-Fuller Test Unit Root Test # 
############################################### 

Test regression trend 


Call:
lm(formula = z.diff ~ z.lag.1 + 1 + tt + z.diff.lag)

Residuals:
      Min        1Q    Median        3Q       Max 
-0.068623 -0.007659  0.002293  0.011152  0.051096 

Coefficients:
              Estimate Std. Error t value Pr(>|t|)    
(Intercept)  0.3289056  0.0942984   3.488 0.000715 ***
z.lag.1     -0.1159234  0.0337053  -3.439 0.000840 ***
tt           0.0012177  0.0003568   3.412 0.000918 ***
z.diff.lag   0.4761895  0.0826273   5.763  8.5e-08 ***
---

Residual standard error: 0.02057 on 104 degrees of freedom
Multiple R-squared:  0.2706,	Adjusted R-squared:  0.2496 
F-statistic: 12.86 on 3 and 104 DF,  p-value: 3.262e-07


Value of test-statistic is: -3.4393 6.1029 5.927 

Critical values for test statistics: 
      1pct  5pct 10pct
tau3 -3.99 -3.43 -3.13
phi2  6.22  4.75  4.07
phi3  8.43  6.49  5.47
\end{verbatim}

Sonuçlara göre $\phi_2 = 6.1$, kritik değere göre yüzde 1 önemlilik
(significance) seviyesinde $\phi_2$ hipotezini reddediyoruz, yani rasgele
yürüyüş hipotezini reddediyoruz (daha doğrusu kısıtlamaların önemsiz
olduğunu reddedince RY dolaylı olarak reddedilmiş oluyor). Bu demektir ki
veride bir kayış (drift), birim kök, ya da deterministik bir trend var
($\beta_1 t$ ile gösterilen). Tabii birim kök hala ``tamamen yok'' diyemiyoruz,
bu nüansı önce belirttik.

Şimdi, acaba kayış (drift), yoksa trend varlığından hangisi daha muhtemel
diye görmek için $\phi_3$'e bakıyoruz. Bu test ile sıfır hipotezi üstteki
tabloda 1. model, alternatifi 2. model, aradaki tek fark $\beta_1=0$
olması. Eğer bunu reddedebilirsek trendi kabul etmemiz gerekir. Test değeri
5.9, kritik değere göre yüzde 10 seviyesinden bu hipotezi reddedebiliriz,
ve alternatif tez olan bu zaman serisinin trend durağan (trend stationary)
olduğunu kabul ederiz. Trend durağanlık formüldeki $\beta_1 t$'nin sıfır
olmadığı / etkin olduğunu söylemektedir.

Not: Trend durağanlık nedir? Bir zaman serisi durağan değil ise bu durağan
olmamanın birkaç farklı sebebi olabilir, birim kök durumu bunlardan biri ki bu
durumda RY ortaya çıkıyor. Fakat RY yerine deterministik trend de
durağansızlığın bir sebebi olabiliyor, $\beta_1 t$ kadar bir ek sürekli her
zaman diliminde seriye ekleniyor; bu da durağanlığı bozar. Yani durağanlığın
bozulması hem birinci hem ikinci ya da ikisinin birden varlığı sebebiyle ortaya
çıkabilir [4, sf 270]. Üstteki testler bu iki durumu birbirine alternatif hale
getirip test etmeye uğraşır.

Görüldüğü gibi elle yapılan regresyon ve F-test sonuçları paketin
sonuçlarına oldukça yakın. Bir de $\tau_\tau$ değerini hesaplayalım, bu
testin sonucu 1. regresyondaki $y_{t-1}$'nin katsayısından gelir, daha
doğrusu onun t değeridir, yani regresyon katsayısını standart hatası ile
bölersek sonucu elde ederiz,

\begin{minted}[fontsize=\footnotesize]{python}
print 't degeri', res1.params['ylag'] / res1.bse['ylag']
\end{minted}

\begin{verbatim}
t degeri -3.43932388217
\end{verbatim}

ki üstteki kritik değerlere göre bu değeri yüzde 5 seviyesi -3.43'ten küçük
olduğu için bu hipotezi de reddetmek mümkün, yani birim kök varlığı
reddedilmiştir.

S\&P Fiyat / Kazanç Oranı

Bu örnekte 1871-2002 yılları arasında S\&P Birleşik Hisse İndisi fiyat seviyesi
ile bu indis altında izlenen şirketlerin getirdiği kazancın oranına
bakacağız. Literatürde bu gibi değerleme oranlarının (valuation ratio)
ortalamaya dönüş (mean reversion) davranışı gösterip göstermediği oldukça ilgi
çeken bir konudur, çünkü hisse senetlerinin fiyatlarının ne olacağını tahmin
etmek gibi ilginç uygulama alanları vardır [4, sf. 274]. Mesela Campbell and
Shiller adlı araştırmacılara göre 1990'ların sonlarında görülen çok yüksek
fiyat/getiri oranı hisse senetlerinin düşeceğinin habercisidir, ki böylece
``tarihi normale dönüş'' olacaktır / olmalıdır. Önce bu oranını log'u alınmış
halini grafikleyelim,

\begin{minted}[fontsize=\footnotesize]{python}
import pandas as pd
dfpe = pd.read_csv('pe.dat',sep='\s*')
dfpe['ln_price_earnings'] = np.log(dfpe.PRICE/dfpe.EARNINGS)
dfpe.ln_price_earnings.plot()
plt.savefig('tser_rwtst_01.png')
\end{minted}

\includegraphics[height=6cm]{tser_rwtst_01.png}

Görünüşe göre bir ortalamayı merkez alan / etrafında salınım var gibi
gözüküyor, bazen ortalamaya dönüş uzun yıllar alabiliyor gibi de
duruyor.. Şimdi birim kök varlığını test edelim,

\begin{minted}[fontsize=\footnotesize]{python}
series = dfpe.ln_price_earnings
%R -i series
%R  adf <- ur.df(series, type = 'drift',lags=0)
%R -o adfout adfout <- summary(adf)
print adfout
\end{minted}

\begin{verbatim}

############################################### 
# Augmented Dickey-Fuller Test Unit Root Test # 
############################################### 

Test regression drift 


Call:
lm(formula = z.diff ~ z.lag.1 + 1)

Residuals:
     Min       1Q   Median       3Q      Max 
-0,53078 -0,10211  0,00324  0,12171  0,41194 

Coefficients:
            Estimate Std. Error t value Pr(>|t|)   
(Intercept)  0,33486    0,12790   2,618   0,0099 **
z.lag.1     -0,12452    0,04847  -2,569   0,0113 * 
---

Residual standard error: 0,1777 on 129 degrees of freedom
Multiple R-squared:  0,04867,	Adjusted R-squared:  0,04129 
F-statistic:   6,6 on 1 and 129 DF,  p-value: 0,01134


Value of test-statistic is: -2,569 3,4573 

Critical values for test statistics: 
      1pct  5pct 10pct
tau2 -3,46 -2,88 -2,57
phi1  6,52  4,63  3,81
\end{verbatim}

Kritik değer -2.569, yüzde 10 eşiğinden bile daha büyük bu, demek ki
rasgele yürüyüş tezini reddedemiyoruz. 

Fakat diğer yandan, ve özellikle serinin görünüşünde net şekilde
görülebileceği üzere, serinin durağan-olmadığını da tam
reddedemiyoruz. Hurst hesabı,

\begin{minted}[fontsize=\footnotesize]{python}
import sys; sys.path.append('../tser_mean')
import hurst as h
print h.hurst(dfpe.ln_price_earnings)
\end{minted}

\begin{verbatim}
0.0756699198133
\end{verbatim}

Bu değer 0.5'ten oldukça uzak.

Küresel Isınma Var mı?

İklimin daha sıcaklaşıp sıcaklaşmadığı politikacılar, bilim adamları
arasındaki tartışma konularından biri. Eğer ısınma var ise, ve kaynağı biz
insanlar isek, bu durumu durdurmak için adımlar atılmalı, çünkü yine bilim
adamlarına göre sonuçları kötü olabilir. Isınmanın olup olmadığını kontrol
etmek için GİSS tarafından yayınlanan veriye bakacağız, bu veri 1880-2010
arasında ay bazlı tüm dünyadaki sıcaklık anormalliklerini kaydetmiştir,
anormallik ise 1951-80 periyotunun ortalaması baz alınarak tanımlanmıştır.

\begin{minted}[fontsize=\footnotesize]{python}
import pandas as pd
dfclim = pd.read_csv('climate-giss.csv',index_col=0,parse_dates=True)
\end{minted}

\begin{minted}[fontsize=\footnotesize]{python}
dfclim.Temp.plot()
plt.savefig('tser_rwtst_03.png')
\end{minted}

\includegraphics[height=6cm]{tser_rwtst_03.png}

``Küresel ısınma yok'' diyen arkadaşlar üstteki grafikte görüldüğü gibi
özellikle 20. yüzyılda mevcudiyeti bariz olan yükselişi açıklamak
durumundalar, bunun için sıcaklık ``rasgele yürüyüştür'' diyenler var; bu
teze göre sıcaklık verisi hisse senedi gibidir, ``yukarı da çıkabilir,
aşağı da inebilir, tahmin edilemez bir şekilde dalgalanır''. Bu teze göre
şimdiye kadar olan çıkışın bir inişi de ``olabilir'', herşey rasgeledir
belki de onlar için daha önemlisi ``olanlar bizim (insanlığın) kontrolü
dışındadır'' . Tabii ilk akla gelen soru 100 kusur senedir yükselen nasıl
bir senet ki bu analoji doğru olabilsin..? Matematiksel olarak, yazının
başında gördüğümüz RY testi herhalde faydalı bir bilgi sağlar.

\begin{minted}[fontsize=\footnotesize]{python}
series = dfclim.Temp
%R -i series
%R  adf <- ur.df(series, type = 'trend',selectlags="AIC")
%R -o adfout adfout <- summary(adf)
print adfout
\end{minted}

\begin{verbatim}

############################################### 
# Augmented Dickey-Fuller Test Unit Root Test # 
############################################### 

Test regression trend 


Call:
lm(formula = z.diff ~ z.lag.1 + 1 + tt + z.diff.lag)

Residuals:
    Min      1Q  Median      3Q     Max 
-80.967  -9.868   0.214  10.370  85.279 

Coefficients:
              Estimate Std. Error t value Pr(>|t|)    
(Intercept) -10.578706   1.163642  -9.091   <2e-16 ***
z.lag.1      -0.277517   0.020894 -13.282   <2e-16 ***
tt            0.014425   0.001428  10.105   <2e-16 ***
z.diff.lag   -0.224780   0.024657  -9.116   <2e-16 ***
---

Residual standard error: 16.81 on 1562 degrees of freedom
Multiple R-squared:  0.2205,	Adjusted R-squared:  0.219 
F-statistic: 147.3 on 3 and 1562 DF,  p-value: < 2.2e-16


Value of test-statistic is: -13.2819 58.8243 88.2169 

Critical values for test statistics: 
      1pct  5pct 10pct
tau3 -3.96 -3.41 -3.12
phi2  6.09  4.68  4.03
phi3  8.27  6.25  5.34
\end{verbatim}

Testlere bakınca \verb!tau3! reddedilmiş, $\phi_2,\phi_3$ reddedilmiş. Yani
RY tezi net bir şekilde reddedildi, daha önemlisi trend durağanlığı kabul
edildi. Bir trend var, ve bu trend bariz bir şekilde yukarıya doğru. 

Bu noktada inkarcı arkadaşlar (!) vites değiştirip ``burada ortalamaya
dönüş var'' diyebilirler, eh tabii ki sıcaklık verisinde ortalama dönüş
var, evet Hurst hesabına bakarsak,

\begin{minted}[fontsize=\footnotesize]{python}
import sys; sys.path.append('../tser_mean')
import hurst as h
print h.hurst(dfclim.Temp)
\end{minted}

\begin{verbatim}
0.040611546891
\end{verbatim}

Varyans Oranı

\begin{minted}[fontsize=\footnotesize]{python}
from arch.unitroot import VarianceRatio
vr = VarianceRatio(dfclim.Temp, 12)
print(vr.summary().as_text())
\end{minted}

\begin{verbatim}
     Variance-Ratio Test Results     
=====================================
Test Statistic                -15.768
P-value                         0.000
Lags                               12
-------------------------------------
Computed with overlapping blocks (de-biased)
\end{verbatim}

Fakat bu iki sonuç rasgele yürüyüş reddi aslında, direk ortalamaya dönüşün
kabulü değil! Bunun detayları için Varyans Ortalaması teorisine yakında
bakmak lazım. Yanlız şu da var ki ADF testlerindeki red gözardı edilemez,
ki bu red rasgele yürüyüşü reddedip onu trende göre test etti. Üstüne
üstlük kritik değerler çok yüksek seviyelerde asıldı, ki bu yüzden trendi
kabul etmek zorunda kaldık. İşin meteorolojik tarafı da önemli; bilimcilere
göre iklimdeki birkaç derecelik kalıcı değişimin bile büyük etkileri
olacağını duyuyoruz. Yanlış sonuca varmanın bedeli ağır olur.

Kaynaklar 

[1] Enders, {\em Applied Econometric Time Series}

[2] Pfaff, {\em Analysis of Integrated and Cointegrated Time Series with R}

[3] FRED, {\em Industrial Production Index}, \url{https://research.stlouisfed.org/fred2/graph/?g=1bHY}

[4] Verbeek, {\em A Guide to Modern Econometrics}

[5] Dickey D., Fuller W., {\em Likelihood Ratio Statistics for Autoregressive
Time Series with Unit Root}, 1981

\end{document}
