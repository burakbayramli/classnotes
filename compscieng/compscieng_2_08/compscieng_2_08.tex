\documentclass[12pt,fleqn]{article}\usepackage{../../common}
\begin{document}
Ders 2.08

[Lineer problemler atlandı]

Şimdi gayrı lineer problemlere gelelim. Alttaki model denkleme bakalım,

$$
u_t + u u_x = 0
\mlabel{1}
$$

Bu denklem ile daha önce gördüğümüz alttaki denklem arasındaki fark bariz,

$$
u_t = c u_x
$$

Üstteki yatay iletim (advection) denkleminde sabit bir hız var, $c$. Ama iki
üstteki durumda hız $-u$, ya da $c$'nin yerine $-u$ koymuş oluyoruz.  O gayrı
lineer denklemi analiz etmek istiyoruz, daha önce olduğu gibi analitik olarak
çözmek isteriz, eğer mümkünse bir formüle erişmek isteriz.. Karakteristik
çizgiler bağlamında neler olduğuna bakmak isteriz.. Sonuçta tek bir uzay
değişkeni ve tek bir denklem var, buradan karakteristiklere bakarak işin
özünü görmek mümkün olmalı.

Ana denkleme eşdeğer olan bir form görelim,

$$
\frac{\partial u}{\partial t} +
\frac{\partial }{\partial x}
\underbrace{ \left( \frac{u^2}{2} \right)}_{f(u)} = 0
\mlabel{2}
$$

Hatta üstteki forma daha ``doğru'' form ismi verilebilir, eğer onu açsak (1)'e
erişirdik fakat üstteki denklemdeki parantez içindeki kısmın fiziksel bir anlamı
var, o kısma akış (flux) ismi veriliyor.

Fakat göreceğiz ki bu diferansiyel denklem aynı noktada iki tane çözüm ortaya
çıkartabiliyor, ve onlardan birini seçmemiz gerekiyor. Diğer bir deyişle çözüm
süreksiz (discontinuous) hale gelebiliyor. Mükemmel pürüzsüz bir başlangıç
fonksiyonu bir süre sonra süreksiz oluyor. Sürekli başlıyoruz, çözüm
karakteristik çizgiler üzerinde sabit, fakat iki karakteristik çizgi birbiriyle
çakıştığında ne olur? Daha önce çakışma olmadı çünkü ana denklem o türde
değildi. Burada mümkün!

O durumu alta çizdim,

\includegraphics[width=15em]{compscieng_2_08_01.png}

Bakıyoruz sol taraftaki o kısımda başlangıç değerleri 1, onlar sabit $c$
durumunda olduğu gibi problemsiz sağ yukarı doğru gidiyorlar. Ama sağ kısma
bakarsak, $u_0=0$ durumu için, bu karakteristik çizgilerle bağlantılı hız
sıfır, orada dalga hareket etmiyor yani, karakteristik orada direk yukarı
çıkıyor, çünkü hiçbir şey olmuyor, $u$ değeri olduğu haliyle yukarı taşınıyor,
sağa, sola gidiş yok. Ama o orta kısımda bir şeyler oluyor, karalanmış
kısımdan bahsediyorum. Burada neler olduğunu iyi anlamamız gerekiyor.

Cebire girmeden belki grafikleyerek bir şeyler anlayabilirim, orta bölgedeki
karakteristik çizgilerin eğimi sıfır ile bir arasında, yani, şimdi $u_0(x)$'i
$t=0$ için grafiklersem,

\includegraphics[width=15em]{compscieng_2_08_02.png}

Sol kısımda sabit 1'de gidiyorum, sağ kısımda sabit 0'da gidiyorum. Arada
lineer bir düşüş var. Üstteki grafikte değer 1, $1-x$, 0 değerlerinde..
Bu değerler iki üstteki grafikte yatay eksenın bölümlerine tekabül ediyor.
İki üstte $x,t$ uzayındayız dikkat, bir üstte ise $x,u$ uzayında.

Yani $t=1$ noktasina kadar karakteristikler tum hikayeyi anlatiyorlar.  $u$'nun
nerede sifir oldugunu, nerede 1 oldugunu biliyorum. $t=1$ noktasina kadar
diferansiyel denklem icin durum iyi. Mesela $t=1/2$'de durum nedir? 

\includegraphics[width=10em]{compscieng_2_08_03.png}

Bu durumda dalga biraz daha ilerlemiş durumdadır, iniş daha dik hale gelmiştir.
$t=1$'de tabii ki grafik tam dik hale gelmiş olacaktır.

Fakat $t=1$ sonrası için düşünürsek şimdi, aynı noktadan birden fazla
karakteristik geçiyor olacak. Orada 0'dan mı yoksa 1'den gelen değerleri mi baz
almak lazım? O çakışma bölgesinde neler oluyor? İşler karıştı çünkü benim
karakteristik takip etme kuralım bana iki tane sonuç verdi. Bize bir cevap
lazım, daha önemlisi, fiziksel olarak anlamlı bir cevap lazım.

Ortada bir sok dalgasi, sok cizgisi var, fakat onun hakkinda arakteristiklerden
iyi bilgi gelmiyor. Merak ettigimiz sok cizgisi neye benzer, gidis yolu nedir,
egimi nedir? Onu ne kontrol eder? Ne yazik ki diferansiyel denklem bize burada
dert oluyor. Olanlari anlamak icin denklemin entegral formuna gitmek daha dogru
olabilir.

$$
\frac{\ud}{\ud t} \int_{x_L}^{x_R}
u \ud x + \left[ f(u_{right}) - f(u_{left})  \right]
$$


Tabii entegral alınca her şeyi daha pürüzsüz hale getirmiş oluyorum, ama aynı
anda muhafaza fiziksel kanununa sadık kalmış oluyorum. Formülün nereden
geldiğini görmek zor değil, (2)'nin $x$ üzerinden entegral alınmış hali aslında,
sol terimdeki zamansal türev hala orada, ama ikinci terimdeki yer türevi
yok olacak, ve entegral sınırları olarak bir noktadan diğerine diyelim şimdilik,
bunlar $x_{left}$, ve $x_{right}$, akis fonksiyonu icin $u_{left}$ ve $u_{right}$.

Bu entegral muhafaza kanunu ne diyor? Entegral derken (2)'deki entegral terimden
bahsediyorum, o entegralin muhafazasından bahsediyoruz, diyor ki o entegralde
değişim var ise, o değişim sağdan dışarı çıkan ya da soldan içeri giren akışla
mümkün olabilir / onların toplamıdır. Tüm bunlar sıfıra eşit olduğu için
bir muhafaza beyani oluyor bunlar. 

















[devam edecek]
  
\end{document}
