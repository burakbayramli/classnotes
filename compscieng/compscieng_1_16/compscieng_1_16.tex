\documentclass[12pt,fleqn]{article}\usepackage{../../common}
\begin{document}
Ders 1-16

Makaskirişler - 2. Bölüm

Bugünün en önemli iki işi var, biri $A$ matrisini ortaya çıkartmak, gerilme
(strain), ya da esneme (stretching) matrisi yani.. Çubuklar ne kadar esner?
$A$'da her çubuk için bir satır olacak. $A$ biraz çetrefil hale halde, yaylar
yerine çubuklar var, bu daha fazla kolon demek çünkü iki boyutta her düğüm için
iki bilinmeyen ekleniyor. Bu ortamda $A^T C A$'ye en iyi bakış açısı çubuk
bazında. İkincisi eğlenceli kısım, bazı makaşkırış örnekleri göstereceğim,
bunlardan bazıları deforme olabilen, stabil olmayan türden olabilecek.
Bu sistemleri analiz ederken muhakkak $Au = 0$ sistemini hesaplatabiliriz,
fakat biz aynı anda  mühendislik bakış açısını da geliştirmek istiyoruz.

$A^T C A$ çarpımının çubuk bazlı olmasına döneyim, bu çarpımı satır çarpı $C$
öğesi çarpı kolon olarak görmenin faydalarından bahsetmiştik. Her satır çarpı
kolon bana bir matris veriyordu ve bu matris tek bir çubuğa tekabül eden sonuç
oluyordu, ve çarpımların hepsi toplanınca $A^T C A$ elde edilmiş olur.

O zaman her $A$ satirinda kac tane sifir olmayan oge vardir? Her satir cubugu
temsil ediyorsa ve her ucta iki degisken var ise cevap dort. Sifir olan degerler
ayni cubuga tekabul etmeyen pim noktalari oldugu icin bu satirda olmayacaklar,
degerleri sifir.

[atlandi]





















\end{document}




















