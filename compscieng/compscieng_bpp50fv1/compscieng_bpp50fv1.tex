\documentclass[12pt,fleqn]{article}\usepackage{../../common}
\begin{document}
Sınırlı Hacim (Finite Volume) Yöntemi - 1

Üç boyutlu kütle muhafazası üzerinden süreklilik formül [2]'de işlendi.  Şimdi
tek boyutlu ortamda muhafaza kanunlarını işleyeceğiz, gaz dinamiği, genel
aerodinamik konularında bu yaklaşım faydalı olacak. Çözmeye çalışılacak
problemler muhafaza kanunları içeren hiperbolik sistemler (hyperbolic systems of
conservation laws) olarak anılır. Bu tür sistemler zamana bağlı çoğunlukla gayrı
lineer kısmı türevsel denklemlerdir, ve aslında basit yapıları vardır. Tek
yersel boyutta bu denklemler şuna benzer [3, sf. 1],

$$
\frac{\partial }{\partial t} u(x,t) + 
\frac{\partial }{\partial x} f(u(x,t)) = 0
\mlabel{1}
$$

Daha önce [1]'de Burgers'in denklemini görmüştük, 

$$
u_t + uu_x = 0
\mlabel{2}
$$

Bu denklem (1) ışığında düşünülebilir, eğer $f(u) = \frac{1}{2}u^2$ tanımlarsak,
(1) formülü, yani $u_t + f(u)_x = 0$, (2) ile aynıdır.



[devam edecek]

Kaynaklar

[1] Bayramlı, {\em Hesapsal Bilim, Hesapsal Sıvı Dinamiğine Giriş}

[2] Bayramlı, {\em Fizik, Sıvılar, 1}

[3] Leveque, {\em Numerical Methods for Conservation Laws}

\end{document}
