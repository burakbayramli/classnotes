\documentclass[12pt,fleqn]{article}\usepackage{../../common}
\begin{document}
Sınırlı Hacim (Finite Volume) Yöntemi - 1

Üç boyutlu kütle muhafazası üzerinden süreklilik formül [2]'de işlendi.  Şimdi
tek boyutlu ortamda muhafaza kanunlarını işleyeceğiz, gaz dinamiği, genel
aerodinamik konularında bu yaklaşım faydalı olacak. Çözmeye çalışılacak
problemler muhafaza kanunları içeren hiperboliç sistemler (hyperboliç systems of
conservation laws) olarak anılır. Bu tür sistemler zamana bağlı çoğunlukla gayrı
lineer kısmı türevsel denklemlerdir, ve aslında basit yapıları vardır. Tek tek
yersel boyutta bu denklemler suna benzer [3, sf. 1],

$$
\frac{\partial }{\partial t} u(x,t) + 
\frac{\partial }{\partial x} f(u(x,t)) = 0
$$

Daha önce [1]'de Burgers'in denklemini görmüştük, 

$$
u_t + uu_x = 0
$$

Bu denklem iki üstteki form ışığında düşünülebilir, eğer $f(u) = \frac{1}{2}u^2$
tanımlarsak, iki üstteki $u_t + f(u)_x = 0$ denklemi bir üstteki denklem ile
aynıdır.



[devam edecek]

Kaynaklar

[1] Bayramlı, {\em Hesapsal Bilim, Hesapsal Sıvı Dinamiğine Giriş}

[2] Bayramlı, {\em Fizik, Sıvılar, 1}

[3] Leveque, {\em Numerical Methods for Conservation Laws}

\end{document}
