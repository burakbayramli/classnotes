\documentclass[12pt,fleqn]{article}\usepackage{../../common}
\begin{document}
Ders 2.3

Her hesapsal yöntemin doğruluğu ve stabilitesini bilmek isteriz. En basit
başlangıç değer problemi (initial value problem -İVP-) ile başlayalım,

$$
\frac{\partial u}{\partial t} =
c \frac{\partial u}{\partial x}
$$

Buna tek yön dalga denklemi diyebiliriz, iki yönlü dalga denklemi için üstteki
formülde ikinci türevlerin olması gerekirdi, o tür denklemde dalgalar iki yöne
de giderdi. Üstteki tek yöne dalga gönderiyor, basit, temiz bir denklem, birinci
derece, hız bağlamında sabit katsayılı. Başlangıç değer problemi için başlangıç
değeri $u(x,0)$ ile verilmiş olsun, ve benim ilgilendiğim $u(x,t)$ çözümü.





\end{document}

