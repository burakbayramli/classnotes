\documentclass[12pt,fleqn]{article}\usepackage{../../common}
\begin{document}
Izgara Denklemleri

Bu denklemlere gelmeden önce Galerkin, ve şekil fonksiyonları (shape function)
konusuna bakalım.

Alttaki gibi bir denklem düşünelim,

$$
E I \frac{\ud^4 y}{\ud X_1^4} = q
\mlabel{1}
$$

Biraz düzenleme sonrası

$$
E I \frac{\ud^4 y}{\ud X_1^4} - q = 0
$$

elde ederim. Amacım öyle bir yaklaşık $y$, ya da $y_{approx}$ diyelim, bulmak ki
üstteki denklemi çözebileyim. Bunu $y$ yerine onu yaklaşık temsil edebilen bir
diğer fonksiyonu geçirerek yapabilirim. Bir polinom bu işi görebilir; Pek çok
diğer yöntemin kullandığı tipik bir polinom vardır,

$$
y_{approx} = a_0 + a_1 X_1 + a_2 X_1^2 
$$

diye gider, aslında daha genel olarak olan her terimde ``bir katsayı çarpı
$X_1$'in bir tür fonksiyonu'' gibi bir toplam kullanmak daha iyi olabilir,
bu formda,

$$
y_{approx} = a_0 \phi_0(X_1) + a_1 \phi_1(X_1) + a_2 \phi_2(X_1) 
$$

Daha kısa olarak

$$
y_{approx} = \sum_{i=0}^{n} a_i \phi_i(X) 
$$

Dikkat $\phi_i(X)$ içinde $X$ var bu $X = X_1,X_2,..,X_n$ anlamında, cebirsel
olarak her $\phi$ fonksiyonuna $X$ geçildiğini düşünebiliriz ama her $\phi_i$
tüm $X$ öğelerini kullanmayabilir; üstteki polinom örneğinde mesela $\phi_1$
fonksiyonu sadece $X_1$'i kullanarak bir hesap yapar, diğerleri diğer şekillerde.

Not, $y_{approx}$ gerekli (essential) sınır şartlarını yerine getirmelidir.

Şekil Fonksiyonları (Shape Functions)

Diyelim ki bir çubuğa bakıyorum ve onun üzerinde iki tane düğüm tanımladım,
düğümlerden biri $X_1 = x_1$ noktasında diğeri $X_2 = x_2$ noktasında.

\includegraphics[width=15em]{compscieng_bpp45fem2_05.jpg}

Ve yine diyelim ki bu iki düğümdeki yer değişimi $u_i$ değerlerini biliyorum,
eldeki örnek için $u_1$ ve $u_2$, kabaca alttaki gibi olsun,

\includegraphics[width=15em]{compscieng_bpp45fem2_06.jpg}

Şekle gelelim; eğer bu iki düğüm üzerinden bir lineer bağlantı kullanmak
istiyorsam yani iki düğüm arasında aradeğerleme yapacak fonksiyon lineer olsun
diyorsam, yapılacak olan bariz aslında,

\includegraphics[width=15em]{compscieng_bpp45fem2_07.jpg}

Aradeğerleme $u_e$ bu şekilde. Peki o fonksiyonda bilinmeyen $a_0,a_1$ nasıl
bulunacak? Biz nihai sonuç olarak bu katsayılarla ilgilenmiyoruz, bizi tek
ilgilendiren yer değişim fonksiyonu, bunu belirtmiştik. O zaman üstteki
fonksiyonu $u_1,u_2$ temelli olarak tekrar yazabilir miyiz acaba? Eğer $x_1$
noktasında fonksiyon değeri $u_1$, $x_2$ noktasında $u_2$ ise,

$$
u(x_1) = a_0 + a_1 x_1 = u_1
$$

$$
u(x_2) = a_0 + a_1 x_2 = u_2
$$

İki bilinmeyen var, iki denklem var, çözüm [1, Ders 2]

$$
a_0 = \frac{u_2 x_1 - u_1 x_2}{L}, \qquad a_1 = \frac{u_2 - u_1}{L}
$$

$L$ kırmızı ile gösterilen parçanın uzunluğu sadece, yani $x_2 - x_1$.

Üstteki $a_0,a_1$ değerlerini nasıl bulduğumuzu merak edenler için

$$
u(x_1) = a_0 + a_1 x_1 = u_1, \quad
u(x_2) = a_0 + a_1 x_2 = u_2
$$

ile başlarız, ikinci formülden birinciyi çıkartırsak,

$$
(x_2 - x_1) a_1 = u_2 - u_1 \to a_1 = \frac{u_2 - u_1}{L}
$$

ki $L = x_2 - x_1$

$a_0$'i bulmak için birinci formüldeki $a_1$'i alıp, yani
$a_1 = \frac{u_1 - a_0}{x_1}$, ikinciye sokuyoruz,


$$
a_0 + \frac{\frac{u_1 - a_0}{x_1}}{x_1} x_2 = u_2
$$

$$
x_1 a_0 + u_1 x_2 - a_0 x_2 = u_2 x_2
$$

$$
u_1 x_2 - u_2x_1 = -a_0 x_1 + a_0 x_2
$$

$$
= a_0 (x_2 - x_1) = a_0 L
$$

$$
a_0 = \frac{u_1 x_2 - u_1 x_2}{L}
$$

Devam edelim.

$a_0,a_1$ degerlerini $u_e$ icine koyunca,

$$
u = \frac{u_2 x_1 - u_1 x_2}{L} - \frac{u_2 - u_1}{L} X_1
$$

Biraz daha cebirsel değiştirme sonrası

$$
u = \frac{(x_2 - X_1)}{L} u_1 + \frac{(X_1 - x_1)}{L} u_2
$$

Bu son değişimi yaptık çünkü bu formda dikkat edersek denklem daha önce
gördüğümüz Galerkin deneme fonksiyonlarına benziyor,

$$
u = u_1 \phi_1(X_1) + u_2 \phi_2 (X_1)
$$

kalıbında görüldüğü gibi. Deneme fonksiyonlarında $u_1,u_2$ sabit değerlerdi, bu
bölümde gördüğümüz $u_i$ değerleri de öyle aslında. Bilinen $u_i$ değerlerini
yaklaşık temsile uğraşıyoruz, gerçi çözüm mekaniği içinde o $u$ değerleri de
hesaplanıyor fakat bu belli formülasyonlar için onların bilindiği
farzedilebilir.

Şimdi $\phi_1,\phi_2$ fonksiyonları bizim önceden seçtiğimiz fonksiyonlardı,
polinom seçtiğimizde $X_1,X_1^2$ gibi hesaplar kullandık. Üstteki türetim
sonrası $N_1,N_2$ var, ve ulaştığımız fonksiyonlar

$$
N_1 = \frac{(x_2 - X_1)}{L}, \quad N_2 = \frac{(X_1 - x_1)}{L} 
$$

Bu şekil fonksiyonları neye benziyor acaba? Alttaki gibi,

\includegraphics[width=25em]{compscieng_bpp45fem2_08.jpg}

Bu iki fonksiyonu üst üste koyduğumuzda (superimpose) yani topladığımızda,
sonucun mavi çizgiyi aynen vereceğini görebilirdik.

İlginç bir diğer özellik, eğer mesela $N_2$'ye tek başına bakarsam, onun ilk
düğümde 0 değerinde, ikinci düğümde 1 değerinde olduğunu görebiliriz.  Eğer
$N_3$ olsa bu şekil de birinci ve ikinci düğümde sıfır olurdu. Diğer yandan her
şekil fonksiyonu ait olduğu düğümde 1 değerindedir. $N_1$ birinci düğümde 1
değerinde, $N_2$ ikinci düğümde 1 değerinde, vs. Bu şekil fonksiyonları öyle
tasarlanmıştır.

Izgara





[devam edecek]

Kaynaklar

[1] Petitt, {\em Finite Element Method Theory}, University of Alberta,
    \url{https://www.youtube.com/watch?v=2iUnfPRk6Ro&list=PLLSzlda_AXa3yQEJAb5JcmsVDy9i9K_fi}

[2] Bayramli, {\em Fizik, Materyel Mekanigi 7}

\end{document}
